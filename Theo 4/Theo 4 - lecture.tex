\documentclass[11pt]{scrartcl} % optionaler Paramter: Freiraum nach Absatz: halfparskip

% deutsche Umlaute nutzbar machen
\usepackage[utf8x]{inputenc}
% deutsche Silbentrannung, Standardfloskeln nutzen
\usepackage[ngerman]{babel}

\usepackage[utf8x]{inputenc}
\usepackage[T1]{fontenc}

% mathematische Erweiterungspakete der AMS (American Mathematical Society)
\usepackage{amsmath,amssymb,amstext}

% Fusszeilen
\usepackage[automark]{scrpage2}
\pagestyle{scrheadings}
\clearscrheadfoot
\ifoot[]{\author}
\ofoot[]{\pagemark}


\title{Vorlesungsmitschrift: Theoretische Physik 4 - statistische Physik bei Prof. Alber}
\author{Jan Krause}
\date{\today{}, Darmstadt}

\begin{document}

\maketitle
\tableofcontents

 
% TODO Die folgende Zeile soll eine unterstrischene Überschrift one Nummerierung sein
 Prinzipien der Thermodynamik
% \label{sec:prinzipien}

 \begin{itemize}
  \item th.dyn. Gleichgewichtszustände (werden angenommen von:)
  \item th.dyn. Sys. (beschreiben)
  \item ``natürliche Zustandsänderungen
 \end{itemize}



 \section{Grundprinzipien/-konzepte der Thermodynamik}
 \label{sec:grundprinzipien}

 \subsection{Thermodynamische Systeme}
 \label{sec:grundprinzipien-thermodynamische-systeme}
 \begin{itemize}
  \item wohldefinierte Menge einer physikalischen Substanz
  \item Skalierbarkeit (beliebige Teilnungs- und Vereinigungsprozesse sind möglich)
 \end{itemize}
 
 \subsection{Thermodynamische Gleichgewichtszustände}
 \label{sec:grundprinzipien-thermodynmische-gleichgewichtszustaende}
 
 

  

 
 
 
 
 
 
 
 
 
 
 
 
 
 
 
 
 
 
 
 
 
 
 
 
\end{document}
