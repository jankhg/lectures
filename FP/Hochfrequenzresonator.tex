\documentclass[10pt,article,colorback,accentcolor=tud9d]{scrartcl}
\usepackage[utf8]{inputenc} %?%
\usepackage[T1]{fontenc}
\usepackage{lmodern} %cool beans%
\usepackage[english,ngerman]{babel} %deutsche Schriftzeichen%
\usepackage{amsmath} %für Formeln%
\usepackage{float} %Für Position von Abbildungen%
\usepackage[colorlinks,
pdfpagelabels,
pdfstartview = FitH,
bookmarksopen = true,
bookmarksnumbered = true,
linkcolor = black,
plainpages = false,
hypertexnames = false,
citecolor = black] {hyperref} %Für verlinktes Inhaltsverzeichnis% 
\usepackage{ulem} %unterstreichen und durchstreichen
\usepackage{cite} %für bibtex
\usepackage{url} %für bibtex
%\usepackage{harvard}%für bibtex
\usepackage{pdfsync}%für sumatra
\usepackage{placeins}%Floatbarrier
\title{Hochfrequenzresonator}
\subtitle{F-Praktikum von Jan Krause und Philipp Dijkstal}
%\subsubtitle{email: \textaccent{philipp.dijkstal@web.de, jankhg@gmail.com}}
%\institution{Fachbereich Physik \\ Institut für Kernphysik}
\begin{document}

\maketitle

\tableofcontents

\newpage
\bibliographystyle{abbrv}
\section{Einleitung}
Ein Hochfrequenzresonator (im Folgenden HFR) ist ein metallisches Gefäß, dessen Hohlraumgeometrie man sich zunutze machen kann,
um einen Resonator zu realisieren. Im allgemeinen Fall lassen sich die Resonanzfrequenzen eines HFR nicht analytisch bestimmen.
Für einfache Geometrien, wie z.B. einen Würfel oder einen Zylinder, ist dies jedoch möglich. Im Falle eines wie hier betrachteten
Zylinders bedient man sich dabei der so genannten Besselfunktionen 1. Art.

Besonders in der Beschleunigerphysik spielen Hochfrequenzresonatoren eine zentrale Rolle, wenn es darum geht, geladene Teilchen
zu beschleunigen.

\section{Grundlagen}

\subsection{Zylinderresonator}
Im vorliegenden Versuch betrachten wir einen Zylinderresonator, dessen Resonanzfrequenzen sich wie bereits erwähnt mit Hilfe der
Besselfunktionen 1. Art analytisch bestimmen lassen.

\begin{align*}
&m: &\text{azimutale Quantenzahl}\\
&n: &\text{radiale Quantenzahl}\\
&p: &\text{axiale Quantenzahl}\\
&j_{m,n}: &\text{n-te Nullstelle der Besselfunktion} J_m
\end{align*}


\begin{equation}
 f_{mnp}^{TM} = \frac{c}{2\pi} \sqrt{\left(\frac{j_{m,n}}{R}\right)^2 + \left(\frac{p\pi}{h}\right)^2}
\end{equation}

\begin{equation}
 f_{mnp}^{TE} = \frac{c}{2\pi} \sqrt{\left(\frac{j'_{m,n}}{R}\right)^2 + \left(\frac{p\pi}{h}\right)^2}
\end{equation}

mit den Besselfunktionen
\begin{equation}
 J_m(x) = \sum_{k=0}^{\infty} \frac{(-1)^k}{k! \Gamma(k+m+1)} \left(\frac{x}{2}\right)^{2k+m}
\end{equation}





\section{Auswertung}

\end{document}