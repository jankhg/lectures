\documentclass[11pt]{scrartcl} % optionaler Paramter: Freiraum nach Absatz: halfparskip

% deutsche Umlaute nutzbar machen
\usepackage[utf8x]{inputenc}

\usepackage[utf8x]{inputenc}
\usepackage[T1]{fontenc}

% deutsche Silbentrannung, Standardfloskeln nutzen
\usepackage[ngerman]{babel}

% mathematische Erweiterungspakete der AMS (American Mathematical Society)
\usepackage{amsmath,amssymb,amstext}


\begin{document}
 
 
 
 
 % VL 2 - 2013-04-18
 \section{Formalismus der QM}
 \label{sec:formalismus}
 
 \begin{itemize}
  \item mathematisches Handwerkszeug bereitstellen
  
  \begin{itemize}
   \item lin. Algebra
   \item später: Analysis, Funktionentheorie
  \end{itemize}
  
  Grundelemente: Vektoren

 \end{itemize}
 
 
 
 \subsection{Vektoren und Hilbertraum}
 
 \begin{itemize}
  \item Grundelemente: Vektoren
  \item Schreibweise: Dirac-Notation -> Ket-Symbol: $| \_ > $
  
  % TODO Pfeil nach oben
  Bsp.: $|0>, |1>, |n>, |nlm>, | Pfeil nach oben >$
  
  \item Vektorraum: Ket-Vektoren sind Elemente eines linearen komplexen
  Vektorraumes V, auf dem Addition und Multiplikation mit komplexen Zahlen def.
  
  \item Eigenschaften:
  
  \begin{itemize}
   \item Multiplikation:
   \begin{itemize}
    \item Abgeschlossenheit: $|\alpha> + |\beta> \in V$
    \item Distributivität:
    \begin{enumerate}
     \item $c ( |\alpha> + |beta> ) = c|\alpha> + c|\beta>, c,d \in V$
    \end{enumerate}

    
   \end{itemize}

  \end{itemize}

  
 \end{itemize}


 
 
 
 
 
 
 \begin{itemize}
  \item Dimension des Vektorraums V
  
   maximale Zahl lin. unabh. Vektoren
   
  
  \item Satz von n lin. unabh. Vektoren in einem Vektorraum der Dim. n.
  
  
  \item Basisentwicklung:
  
  Jeder Vektor $|\alpha>$ eines n-dim. Vektorraumes lässt sich eindeutig durch eine Lin.komb. von n Basisvektoren
  $ \{ |1>,...|n> \} $ mit Koeff. $ C_1,...,C_n $ darstellen:
  
  % TODO ist das Folgende richtig?
  \begin{equation} |\alpha> = \sum_{i=1}^{n} C_i |i> \end{equation}
 \end{itemize}

 
 
 \subsubsection{Skalarprodukt und Hilbertraum}
 
 \begin{itemize}
  \item Skalarprodukt ordnet zwei Vektoren $ |\alpha>, |\beta> $ eine komplexe Zahl zu.
  
  typische Schreibweisen:
  
  $ \vector{a} \vector{b} $
  
  In Dirac-Notation:
  
  $ <\alpha|\beta> = <\alpha||\beta> $
  
  \item Eigenschaften
  
  \begin{itemize}
   \item Schief-Symmetrie: $ <\alpha|\beta> = ( <\beta|\alpha> )^* = <\beta|\alpha>^* $
   
   \item Positiv Semidefinit: $ <\alpha|\alpha> >= 0 $
   
   (wobei: $ <\alpha|\alpha> = 0 <=> |\alpha> = |0> $ )
   
   \item Linearität im Ket: $ <\alpha| ( C |\beta> + D | \gamma> ) = C < \alpha|\beta> + D <\alpha|\gamma> $
  \end{itemize}

 \end{itemize}

 
 
 
 
 
 
 
 
\end{document}
