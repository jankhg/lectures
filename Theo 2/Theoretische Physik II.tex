\documentclass[10pt,article,colorback,accentcolor=tud9d]{scrartcl}
\usepackage[utf8]{inputenc} %?%
\usepackage[T1]{fontenc}
\usepackage{lmodern} %cool beans%
\usepackage[english,ngerman]{babel} %deutsche Schriftzeichen%
\usepackage{amsmath} %für Formeln%
\usepackage{float} %Für Position von Abbildungen%
\usepackage[singlespacing]{setspace}
\usepackage{nccmath}
\usepackage[colorlinks,
pdfpagelabels,
pdfstartview = FitH,
bookmarksopen = true,
bookmarksnumbered = true,
linkcolor = black,
plainpages = false,
hypertexnames = false,
citecolor = black] {hyperref} %Für verlinktes Inhaltsverzeichnis% 
\usepackage[a4paper, left=2cm, right=2cm, top=2cm]{geometry}%Ränder
\usepackage{amssymb}%Für R reele zahlen
\title{Theoretische Physik II}
\subtitle{Prof. Robert Roth - SS 2013}
\author{Jan Krause und Philipp Dijkstal}
%\subtitle{Professor: Dr. Gernot Alber\\
%Mitschrift von Philipp Dijkstal}
%\subsubtitle{email: \textaccent{philipp.dijkstal@web.de}}
%\institution{Fachbereich Physik}
\begin{document}
\maketitle
\tableofcontents
\newpage
\noindent Vorlesung 1 vom 16.04.2013 -Gedöns fehlt noch\\
\begin{flushright}Vorlesung 2 vom 18.04.2013\end{flushright}
\section{Formalismus der Quantenmechanik}
$\rightarrow$ mathematisches Handwerkszeug bereitstellen\\
\begin{itemize}
\item Lineare Algebra
\item später: Analysis, Funktionentheorie
\end{itemize}
\subsection{Vektoren und Hilbert-Raum}
\begin{itemize}
\item Grundelement: Vektoren
\item Schreibweise: Dirac-Notation\\
$\rightarrow$ Ket-Symbole bzw. Kets\\
$\left| \quad \rangle  \right.$ $\rightarrow$ Bezeichnungen innerhalb Symbol\\
Bsp.: $\left| 0\rangle \right.,\left|1\rangle \right.,\left|n\rangle \right.,\left|nlm\rangle \right.,
\left| \uparrow\rangle \right.$
\item Vektorraum: Ket-Vektoren sind Elemente eines linearen kompletten 
Vektorraums $(\nu)$\\
$\left|1\rangle \right.,\left|2\rangle \right.,...,\left|\alpha\rangle \right.,...,\left|\beta\rangle 
\right.,... \in \nu$\\
auf dem Addition zweier Vektoren 2 Multiplikationen mit $\mathbb{C}$-Zahl 
definiert
\item Eigenschaften der Addition und Multiplikation einer Zahl aus $\mathbb{C}$
\begin{fleqn}
\begin{itemize}
\item Abgeschlossenheit: 
\begin{equation} 
\left| \alpha \rangle \right.+\left|\beta\rangle \right. \in \nu
\end{equation}
\item Distributivität: 
\begin{equation}
\begin{aligned}
C(\left|\alpha\rangle \right. +\left|\beta\rangle \right.) = C\left|\alpha\rangle \right. +C\left|
\beta\rangle \right.\\
(C+D)\left|\alpha\rangle \right. = C\left|\alpha\rangle \right. +D\left|\alpha\rangle \right.
\end{aligned}
\end{equation}
\item Assoziativität: 
\begin{equation}
C(D\left|\alpha\rangle \right.)=(CD)\left|\alpha\rangle \right.
\end{equation}
\item Kommutativität der Addition: 
\begin{equation}
\left|\alpha\rangle \right. + \left|\beta\rangle \right. =\left|\beta\rangle \right. + \left|\alpha
\rangle \right.
\end{equation}
\item Assoziativität der Addition: 
\begin{equation}
\left|\alpha\rangle \right. + (\left|\beta\rangle \right. + \left|\gamma\rangle \right.) = (\left|
\alpha\rangle \right. + \left|\beta\rangle \right.)+\left|\gamma\rangle \right.
\end{equation}
\item Nullvektor: 
\begin{equation}
\begin{aligned}
\left|\alpha\rangle \right. +\left|0\rangle \right. = \left|\alpha\rangle \right.\\
0 \left|\alpha\rangle \right. = \left|0\rangle \right.\\
C \left|0\rangle \right. = \left|0\rangle \right.
\end{aligned}
\end{equation}
\item Inverses Element bezüglich Addition:
\begin{equation}
\begin{aligned}
\left|\alpha\rangle \right. + \left|{\alpha}_{inv}\rangle \right. = \left|0\rangle \right.\\
\left|{\alpha}_{inv}\rangle \right. = -\left|\alpha\rangle \right.
\end{aligned}
\end{equation}
\end{itemize}
\end{fleqn}
\item Lineare Unabhängigkeit\\
Der Satz
\begin{fleqn}
\begin{equation} \nonumber
{\left|1\rangle \right.,\left|2\rangle \right.,...,\left|n\rangle \right.}
\end{equation}
ist genau dann linear unabhängig, wenn
\begin{equation}
\sum {C}_i \left|i\rangle \right. = \left|0\rangle \right.
\end{equation}
\end{fleqn}
nur für ${C}_1, {C}_2,...,{C}_n=0$\\
erfüllt ist.
\item Dimension des Vektorraums $\nu$ maximale Zahl linear unabhängiger 
Vektroen. 
\item Basisentwicklung: Jeder Vektor $\left|\alpha\rangle \right.$ eines n-dim. 
Vektorraums lässt sich eindeutig durch Linearkombination von n Basisvektoren \\
${\left|1\rangle \right.,\left|2\rangle \right.,...,\left|n\rangle \right.}$ mit Koeffizienten ${C}
_1,...{C}_n$\\
darstellen:\\
$\left|\alpha\rangle \right. = \sum {C}_i \left|i\rangle \right.$
\end{itemize}
\subsection{Skalarprodukt und Hilbertraum}
\begin{itemize}
\item Skalarprodukt ordnet zwei Vektoren $\left|\alpha\rangle \right.,\left|\beta\rangle 
\right.$ eine $\mathbb{C}$-Zahl zu.\\
Typische Schreibweisen:
\begin{fleqn}
\begin{equation}
\vec{a}\vec{b} \quad (\vec{a}\vec{b})
\end{equation}
In Diracnotation:
\begin{equation}
\langle \alpha\left|\right.\beta\rangle  \ =\langle \alpha \left|\right|\beta\rangle 
\end{equation}
$\rightarrow$ bra(c)ket
\item Eigenschaften
\begin{itemize}
\item Schiefsymmetrie:
\begin{equation}
\begin{aligned}
\langle \alpha\left|\right.\beta\rangle  \ =(\langle \beta\left|\right.\alpha\rangle )^*
\end{aligned}
\end{equation}
\item Positive Semidefinit: 
\begin{equation}
\langle \alpha\left|\right.\alpha\rangle  \ \ \geq \ 0
\end{equation}
(wobei: $\langle \alpha\left|\right.\alpha\rangle  \ \ = 0 \Leftrightarrow \left|
\alpha\right.\rangle  \ \ = \ \left|0\right.\rangle $)
\item Kombination von Schief-Symmetrie und Linearität im Ket:
\begin{equation}
\begin{aligned}
\left|\right.\Psi\rangle  \ = C\left|\right.\beta\rangle +D\left|\right.\gamma\rangle \\
\langle \alpha\left|\right.\Psi\rangle  \ =C\langle \alpha\left|\right.\beta\rangle  +D\langle \alpha\left|\right
.\gamma\rangle \\
\langle \Psi\left|\right.\alpha\rangle  \ =\langle \alpha\left|\right.\Psi\rangle ^* \\
 =(C \langle \alpha\left|\right.\beta\rangle  + D\langle \alpha\left|\right.\gamma\rangle )^*\\
=C^*(\langle \alpha\left|\right.\beta\rangle )^* + ^*(\langle \alpha\left|\right.\gamma\rangle )^*\\
=C^*\langle \beta\left|\right.\alpha\rangle  +D^*\langle \gamma\left|\right.\alpha\rangle 
\end{aligned}
\end{equation}
\end{itemize}
\end{fleqn}
\item Daher interpretiert man den ersten Faktor im Skalarprodukt als "`
anderen Typ"' von Vektor: Bra-Vektoren
\item Jedem Ket-Vektor $\left|\right.\alpha\rangle $ ist über duale Korrespondenz 
ein Bra-Vektor zugeordnet\\
$\left|\right.\alpha\rangle  \ \ \rightarrow \ \ \langle \alpha\left|\right.$\\
Entscheidende Eigenschaft:
\begin{fleqn}
\begin{equation}
\begin{aligned}
\left|\right.\Psi\rangle  \ = C \left|\right.\alpha\rangle  +D\left|\right.\beta\rangle \\
\longleftrightarrow\\
\langle \Psi\left|\right.=C^*\langle \alpha\left|\right. +D^* \langle \beta\left|\right.
\end{aligned}
\end{equation}
\end{fleqn}
\item Die Reihenfolge von Bra und Ket im Skalarprodukt ist wichtig!
\item Manchmal auch:
\begin{fleqn}
\begin{equation}
\langle \alpha\left|\right.=(\left|\right.\alpha\rangle )^+ \leftarrow \text{hermetisch 
Adjungiert}
\end{equation}
\end{fleqn}
\item Begriffe im Zusammenhang mit Skalarprodukt:
\begin{itemize}
\item Orthogonalität: Zwei Vektoren $\left|\right.\alpha\rangle , \left|\right.\beta\rangle 
$ sind orthogonal, wenn
\begin{fleqn}
\begin{equation}
\langle \alpha\left|\right.\beta\rangle  \ =0
\end{equation}
\end{fleqn}
\item Norm: Die Norm eines Vektors $\left|\right.\alpha\rangle $ ist 
\begin{fleqn}
\begin{equation}
\left|\right|\left|\right.\alpha\rangle \left|\right|=\sqrt{\langle \alpha\left|\right.
\alpha\rangle }
\end{equation}
\end{fleqn}
Ein Vektor ist normiert, wenn
\begin{fleqn}
\begin{equation}
\left|\right|\left|\right.\alpha\rangle \left|\right| =1 \quad \langle \alpha\left|\right.
\alpha\rangle 
\end{equation}
\end{fleqn}
\end{itemize}
\item Orhonormierte Basis: Basis
\begin{fleqn}
\begin{equation} \nonumber
{\left|\right.1\rangle ,\left|\right.2\rangle ,...,\left|\right.i\rangle ,...,\left|\right.j\rangle ,...}
\end{equation}
heißt orthognonal wenn
\begin{equation}
\langle i\left|\right. j\rangle  \ ={\delta}_{ij}
\end{equation}
\end{fleqn}
\end{itemize}
\begin{flushright}Vorlesung 3 - 23.04.2013\end{flushright}
\section{Lineare Operatoren}
\begin{itemize}
\item Operatoren sind formales Werkzeug, um Vektoren zu manipulieren. \\
\begin{fleqn}
$\rightarrow$4 Operatoren beschreiben Abbildung von einem Ket $\left|\right. 
\alpha \rangle  \in \mathbb{H}$ auf einen anderen Ket  $\left|\right. \alpha' \rangle  \in 
\mathbb{H}$
\begin{equation}
  \left|\right. \alpha' \rangle  \ = \hat{A} \left|\right. \alpha \rangle 
\end{equation}
$\rightarrow$ Operatoren wirken immer nach rechts\\
$\rightarrow$ Hütchen zur Unterscheidung von Zahlen, Matrizen,...
\item In QM sind (fast) ausschließlich lineare Operatoren relevant.
\begin{equation}
\begin{aligned}
  \hat{A} (C\left|\right. \alpha \rangle  + D\left|\right. \beta \rangle )&\\
  &= \hat{A} C\left|\right. \alpha \rangle  + \hat{A} D \left|\right. \alpha \rangle \\
  &= C \hat{A} \left|\right. \alpha \rangle  + D \hat{A} \left|\right. \beta \rangle 
\end{aligned}
\end{equation}
\item Assoziativität und Distributivität 
\begin{equation}
\begin{aligned}
&(\hat{A} + \hat{B}) \left|\right. \alpha \rangle  \ = \hat{A} \left|\right. \alpha \rangle 
 + \hat{B} \left|\right. \alpha \rangle \\
&(\hat{A} \hat{B})\left|\right. \alpha \rangle  \ = \hat{A} (\hat{B} \left|\right. 
\alpha \rangle )
\end{aligned}
\end{equation}
\item Rechnen mit isolierten Operatoren z.B.
\begin{equation}
\begin{aligned}
  &\hat{A} + \hat{B} = \hat{B} + \hat{A} \\
  &(\hat{A} + \hat{B}) + \hat{C}= \hat{A} +(\hat{B}+ \hat{C})
\end{aligned}
\end{equation}
Für Produkte von Operatoren
\begin{equation}
\begin{aligned}
  &\hat{A}(\hat{B} \hat{C}) = (\hat{A} \hat{B})\hat{C}
  &\hat{A}(\hat{B} + \hat{C}) = \hat{A}\hat{B} + \hat{B}\hat{C}
\end{aligned}
\end{equation}
\item Vorsicht: Multiplikation ist nicht multiplikativ
\begin{equation}
\hat{A}\hat{B} \not= \hat{B}\hat{A}
\end{equation}
$\rightarrow$ Reihenfolge ist wesentlich
\item Spezielle Operatoren
  \begin{itemize}
  \item Eins-Operator 
   \begin{equation}
  \hat{1}: \quad \hat{1}\left|\right. \alpha \rangle  \ = \left|\right. \alpha \rangle 
  \end{equation}
  \item Inverser Operator 
  \begin{equation}
  \begin{aligned}
  &\hat{A}^{-1}: {\hat{A}}^{-1} \hat{A}\left|\right. \alpha \rangle  \ = \left|\right
. \alpha \rangle \\
  \text{oder} \quad &\hat{1} = {\hat{A}}^{-1}\hat{A} = \hat{A} {\hat{A}}^{-1}
  \end{aligned}
  \end{equation}
  \end{itemize}
\item Spezieller Operator: dyadisches Produkt
  \begin{equation}
  \left|\right. \beta' \rangle \langle \beta\left|\right.
  \end{equation}
  Anwendung auf Ket
  \begin{equation}
  (\left|\right. \beta' \rangle \langle \beta\left|\right.)\left|\right.\alpha\rangle  \ = \left|
\right.\beta'\rangle \langle \beta\left|\right.\alpha\rangle  \in \mathbb{C}
  \end{equation}
  $\rightarrow$ Wichtig z.B. für Spektraldarstellung von Operatoren
\item Matrixelemente von Operatoren\\
  Betrachte Transformation von $\left|\right. \beta \rangle $ mittels $\hat{A}$
  \begin{equation}
  \left|\right. \beta'\rangle  \ = \hat{A}\left|\right. \beta\rangle 
  \end{equation}
  Skalarprodukt mit $\langle \alpha \left|\right.$
  \begin{equation}
  \langle \alpha\left|\right.\beta'\rangle  \ = \underbrace{\langle \alpha\left|\right.\hat{A}\left
|\right.\beta\rangle }_{\text{Matrixelement}}
  \end{equation}
\item Erwartungswert eines Operators: Diagonales Matrixelement
  \begin{equation}
  \langle \hat{A}\rangle _\alpha = \langle \alpha \left|\right.\hat{A} \alpha\rangle 
  \end{equation}
\end{fleqn}
\end{itemize}
\textbf{Hermetische Adjunktion}
\begin{fleqn}
\begin{itemize}
\item Übertragung der hermiteschen adjungierten für Matrizen auf Operatoren
  \begin{equation}
  \langle \alpha\left|\right.\hat{A}\left|\right.\beta\rangle  \ \xrightarrow{herm. adj.} (\langle 
\beta \left|\right.\hat{A}\left|\right.\alpha\rangle )^*
  \end{equation}
  Def. herm. adj. Operator
  \begin{equation}
  \langle \alpha\left|\right.{\hat{A}}^{\dagger} \left|\right. \beta\rangle  \ =(\langle \beta\left|\right.
\hat{A}\left|\right.\alpha\rangle )^*
  \end{equation}
\item Aus Definition folgen Rechenregeln
  \begin{equation}
  \begin{aligned}
  &(C\hat{A})^{\dagger}=C^*{\hat{A}}^{\dagger}\\
  &({\hat{A}}^{\dagger})^{\dagger}=\hat{A}\\
  &(\hat{A} + \hat{B})^{\dagger} )= \hat{A}^{\dagger} + \hat{B}^{\dagger}\\
  &(\hat{A}\hat{B})^{\dagger}=\hat{B}^{\dagger}\hat{A}^{\dagger}
  \end{aligned}
  \end{equation}
\item \onehalfspacing Universalrezept für herm. Adj.\singlespacing 
  1) Zyklische Umkehr der Reihenfolge\\
  2) Komplex konjugieren der $\mathbb{C}$-Zahl\\
  3) Ersetzung der Bras durch Kets u.U.\\
  4) hermitesche Adjungierte der Einzeloperatoren\\
  Bsp.:
  \begin{equation}
  \begin{aligned}
  &(\langle \alpha\left|\right.\hat{A}^{\dagger}+C\hat{A}\hat{B}\left|\right.\beta\rangle )^{\dagger}\\
  &= (\left|\right.\beta\rangle )^{\dagger} (\hat{A}^{\dagger}+C\hat{A}\hat{B})^{\dagger}(\langle \alpha\left|\right
.)^{\dagger}\\
  &= \langle \beta\left|\right. \hat{A} +C^*\hat{B}^{\dagger} \hat{A}^{\dagger} \left|\right.\alpha\rangle 
  \end{aligned}
  \end{equation}
  Achtung: Ausdrücke wie $\alpha\rangle \hat{A}$ oder $ \hat{A}\langle \alpha\left|\right.$ 
sind sinnnlos!
\item Duale Korrespondenz
  \begin{equation}
  \begin{aligned}
  &\left|\right.\alpha'\rangle &=\hat{A}\left|\right.\alpha\rangle \\
  & \quad \quad \quad \updownarrow&\\
  &\langle \alpha'\left|\right.&= \ \langle \alpha\left|\right.\hat{A}^{\dagger}
  \end{aligned}
  \end{equation}
\end{itemize}

\begin{flushright}
Vorlesung 4 - 25.04.2013
\end{flushright}

\subsection{Hermitesche und Unitäre Operatoren}
\begin{itemize}
  \item Für hermitesche operatoren gilt
    \begin{equation}
    \hat{A}^{\dagger}=\hat{A}
    \end{equation}
    selbst-adjungiert
  \item Erwartungswerte von hermiteschen Operatoren sind reell.
    \begin{equation}
    \langle \alpha\left.\right|\hat{A}\left.\right|\alpha\rangle  \ =  \left(\langle \alpha \left.\right| \hat{A}^{\dagger} \left.\right| \alpha \rangle \right)^* = \left(\langle \alpha \left.\right|\hat{A}\left.\right|\alpha\rangle \right)^*
    \end{equation}
    $\Rightarrow$ $\langle \alpha\left.\right|\hat{A}\left.\right|\alpha\rangle  \in \mathbb{R}$
  \item Für unitäre Operatoren gilt
    \begin{equation}
    \hat{U}^{\dagger}=\hat{U}^{-1} \quad \text{bzw} \quad \hat{U}^{\dagger}\hat{U}=\hat{1}
    \end{equation}
  \item Anwendung eines unitären Operatoren ändert Normierung nicht
    \begin{equation} \nonumber
     \left.\right|\alpha'\rangle  \ = \hat{U}\left.\right|\alpha\rangle  \quad \langle \alpha'\left.\right|= \ \langle \alpha\left.\right|\hat{U}^{\dagger}
    \end{equation}
    \begin{equation}
    \begin{aligned}
    \langle \alpha'\left.\right|\alpha'\rangle  \ =& \ \langle \alpha\left.\right|\underbrace{\hat{U}^{\dagger}\hat{U}}_{\hat{1}}\left.\right|\alpha\rangle \\
    &= \ \langle \alpha\left.\right|\hat{1}\left.\right|\alpha\rangle 
    \end{aligned}
    \end{equation}
    \end{itemize}
\subsection{Funktionen von Operatoren}
\begin{itemize}
  \item Funktionen von Operatoren sind nur über Reihendarstellungen definiert z.B.
    \begin{equation}
    F(\hat{A})=\sum^\infty_{n=0}a_n\hat{A}^n
    \end{equation}
  \item Bsp: Exponentialfunktion
    \begin{equation}
    exp(c\hat{A})= \sum^\infty_{n=0}\frac{c^n}{n!}\hat{A}^n
    \end{equation}
  \item Durch Nicht-Kommutativität des Produkts von Operatoren gelten viele Rechenregeln für spezielle Funktionen \textbf{nicht} mehr z.B.
  \begin{equation}
  exp(\hat{A}+\hat{B})\neq exp(\hat{A}) +exp(\hat{B}), \quad \text{wenn} \quad \hat{A}\hat{B} \neq \hat{B}\hat{A}
  \end{equation}
\end{itemize}
\subsection{Eingenwertproblem \& Darstellungen}
\begin{itemize}
  \item Gegeben sei Operator $\hat{A}$. Ket-Vektoren $\left.\right|a\rangle $ heiße Eigenvektor zum Eigenwert a, wenn
    \begin{equation}
    \hat{A}\left.\right|a\rangle  \ = a\left.\right|a\rangle 
    \end{equation}
    \textcolor[rgb]{1,0,0}{Die Eigenwerte sind im Allgemeinen komplexe Zahlen.}
  \item Spektrum des Operators: Gesamtheit der EW und EV.
  \item Diskrete EW / diskretes Spektrum oder kontinuierliche EW / kont. Spektrum oder Kombination
  \item Diskrete Spektren: endlich oder abzählbar unendlich viele EV\\
    $\rightarrow$ kompatibel mit mathematischem Hilbertraum
  \item kontinuierliche Spektrum: überabzählbar unendlich viele EV\\
    $\rightarrow$ erweiterter Hilbertraum
  \item Entartung: Mehrere Eigenzustände zum selben EW\\
    $\rightarrow$ Entartungsindex\footnote{Komma nicht nötig} $\left.\right|a,n\rangle $
\end{itemize}
\subsection{Eingenwertproblem hermitescher Operatoren}
Besondere Eigenschaften für EW-Problem hermitescher Operatoren
  \begin{itemize}
    \item Eigenwerte hermitescher Operatoren sind reell\\
      $\rightarrow$ Beweis über reellen Erwartungswert und 
      \begin{equation}
      a=\frac{\langle a\left.\right|\hat{A}\left.\right|a\rangle }{\langle a\left.\right|a\rangle }
      \end{equation}
    \item Eigenvektoren zu verschiedenen EW sind orthogonal\\ \\
      a) Ohne Entartung:
         \begin{equation}
        \label{eq:ortho}
          \langle a\left.\right|a'\rangle  = 0 \quad \text{wenn} \quad a \neq a'
          \end{equation}
          $\rightarrow$ Beweis über $(a-a') \ \langle a\left.\right|a'\rangle  = 0$\\ \\
      b) Mit Entartung:\\
         - Wie (a) für EV aus verschiedenen ent. Unterräumen\\
         - Innerhalb des ent. Unterraums muss explizit orthogonalisiert werden.
    \item Eigenvektoren bilden eine vollst. orthonormierte Basis\\
      - Orthogonalität aus \ref{eq:ortho}\\
      - Normierung ist für diskrete Spektren einfach, für kontinuierliche Spektren ist Vorsicht geboten.\\
      - Vollständigkeit: bewiesen für diskrete Spektren aber für kontinuierliche Spektren ist Vollständigkeit mathematisch noch nicht gezeigt. $\rightarrow$ Postulat!
  \end{itemize}
\textbf{Diskerete Eigenbasen}
\begin{itemize}
  \item Eigenwertproblem eines herm. Op. $\hat{A}$ mit diskretem Spektrum
    \begin{equation}
    \hat{A}\left.\right|a_i\rangle  \ = a_i\left.\right|a_i\rangle  \quad i:\text{Zählindex}
    \end{equation}
  \item Vorrausgesetzt EV sind normiert, dann gilt zusammen mit Orthogonalität 
    \begin{equation}
    \langle a_i\left.\right|a_j\rangle  \ = \delta_{ij}
    \end{equation}
  \item Vollständigkeit: Entwicklung eines beliebigen Vektors $\left.\right|\beta\rangle $ in Eigenbasis $\left.\right|a_i\rangle $:
    \begin{equation}
    \left.\right|\beta\rangle  \ = \sum_i C_i \left.\right|a_i \rangle 
    \end{equation}
    - Multiplikation $\langle a_j\left.\right|$
      \begin{equation}
      \begin{aligned}
      \langle a_j\left.\right|\beta\rangle  \ &= \ \langle a_j\left.\right|\left(\sum_i C_i\left.\right|a_i\rangle \right)\\
      &=\sum_iC_i\langle a_j\left.\right|a_i\rangle \\
      &=\sum_iC_i \delta_{ij}\\
      &=C_j
      \end{aligned}
      \end{equation}
    - Einsetzen in Entwicklung
    \begin{equation}
    \begin{aligned}
      \left.\right|\beta\rangle  \ &= \sum_i\left(\langle a_i\left.\right|\beta\rangle \right) \left.\right|a_i\rangle \\
      &=\sum_i\left.\right|a_i\rangle \langle a_i\left.\right|\beta\rangle \\
      &=\left(\sum_i\left.\right|a_i\rangle \langle a_i\left.\right|\right)\left.\right|\beta\rangle \\
      &=\hat{1}\left.\right|\beta\rangle 
    \end{aligned}
    \end{equation}
  \item Zerlegung des $\hat{1}$-Operators
    \begin{equation}
    \hat{1}= \sum_i \left.\right|a_i\rangle \langle a_i\left.\right|
    \end{equation}
    $\rightarrow$ Sehr nützliches Rechenwerkzeug
  \item Beispiel:für "`triviale"' Rechnung:
    \begin{equation}
    \begin{aligned}
    \hat{A} &= \hat{1}\hat{A}\hat{1}\\
    &=\left(\sum_i\left.\right|a_i\rangle \langle a_i\left.\right|\right)\hat{A}\left(\sum_j\left.\right|a_j\rangle \langle a_j\left.\right|\right)\\
    &=\sum_{i,j}\left.\right|a_i\rangle \langle a_i\left.\right|\hat{A}\left.\right|a_j\rangle \langle a_j\left.\right|
    \end{aligned}
    \end{equation}
    $\rightarrow$ Nutze Eigenwertberechnung
    \begin{equation}
    \hat{A}\left.\right|a_j\rangle =a_j\left.\right|a_j\rangle  \nonumber
    \end{equation}
    \begin{equation}
    \begin{aligned}
    &\sum_{i,j}\left.\right|a_i\rangle \langle a_i\left.\right|\hat{A}\left.\right|a_j\rangle \langle a_j\left.\right|\\
    =&\sum_{i,j}a_j\left.\right|a_i\rangle \underbrace{\langle a_i\left.\right|a_j\rangle }_{\delta_{ij}}\langle a_j\left.\right|\\
    =&\sum_ia_i\left.\right|a_i\rangle \langle a_i\left.\right|\\
    =&\sum_i\left.\right|a_i\rangle a_i\langle a_i\left.\right|
    \end{aligned}
    \end{equation}
    $\rightarrow$ Spektraldarstellung des Operators $\hat{A}$
  
\end{itemize}

\begin{flushright}
Vorlesung 5 - 30.03.2013
\end{flushright}
\textbf{Darstellung}\\
$\rightarrow$ normale Anschauung versagt.
\begin{itemize}
	\item Wir unterscheiden zwiscen abstrakten Hilbertraumobjekten (Ket, Bra, Operatoren), die über Rechenregeln definiert sind, und den Darstellungen dieser Objekte.(analogon zu Spalten- und Zeilenvektoren und Matrizen)
  \item Erinnerung: Koordinaten für Darstellung eines Vektors im 3D Raum(kartesisch, Kugelkoordinaten)
  \item Beispiel: Eigenwertproblem eines abstrakten Operators $\hat{B}$.
    \begin{equation}
    \hat{B}\left.\right|b\rangle =b\left.\right|b\rangle
    \end{equation}
    Übertragung in Darstellung durch Eigenbasis von $\hat{A}$, d.h. $\left\{\left.\right|a_i\rangle\right\}$ mit
    \begin{equation}
    \begin{aligned}
    &\langle a_i\left.\right|a_j\rangle=\delta_{ij}\\
    &\hat{1}=\sum_i\left.\right|a_i\rangle\langle a_i\left.\right|
    \end{aligned}
    \end{equation}
    Konstruktion der Darstellung
    \begin{equation}
    \begin{aligned}
      \hat{B}\hat{1}\left.\right|b\rangle=&b\left.\right|b\rangle\\
      \hat{B}\left(\sum_i\left.\right|a_i\rangle\langle a_i\left.\right|\right)\left.\right|b\rangle=&b\left.\right|b\rangle\\
      \langle a_j\left.\right|\hat{B}\left(\sum_i\left.\right|a_i\rangle\langle a_i\left.\right|\right)\left.\right|b\rangle’&\langle a_j\left.\right|b\left.\right|b\rangle\\
      \sum_i\langle a_j\left.\right|\hat{B}\left.\right|a_i\rangle\langle a_i\left.\right|b\rangle =&b\langle a_j\left.\right|b\rangle      
    \end{aligned}
    \end{equation}
  In "`Matrixnotation"'$\Rightarrow$ Matrix-Eigenwertproblem %fehlt
\item Praktische Rechnung wird letztendlich in eine geschickt gewählten Darstellung ausgeführt.
\end{itemize}
\textbf{Darstellungswechsel}
\begin{itemize}
	\item Betrachte zwei Eigenbasen $\{\left.\right|a_i\rangle\}$ und $\{\left.\right|b_i\rangle\}$
  \item Allgemeiner Vektor $\left.\right|\gamma\rangle$ sein in $\{\left.\right|a_i\rangle\}$ Basis dargestellt, d.h. $\langle a_i\left.\right|\gamma\rangle$ sind bekannt.
  \item Wie bekommt man daraus die Entwicklungskoeffizienten in $\{\left.\right|b_i\rangle\}$?
  \begin{equation}
  \begin{aligned}
    \langle b_i \left.\right|\gamma\rangle&=\langle b_i\left.\right|\hat{1}\left.\right|\gamma\rangle\\
    &=\langle b_i \left.\right|\left(\sum_j\left.\right|a_j\rangle\langle a_j\left.\right|\right)\left.\right|\gamma\rangle\\
    &=\sum_j\langle b_i\left.\right|a_j\rangle\langle a_j\left.\right|\gamma\rangle
  \end{aligned}
  \end{equation}
  \item Transformationsmatrix $U_{ij} =\langle b_ia_j\rangle$ ist unitär
    \begin{equation}
      U^{\dagger}U=\hat{1}
     \end{equation}
     \begin{equation}
      \begin{aligned}
      \sum_jU^{\dagger}_{ij}U_{jk}&=\sum_j\langle b_j\left.\right|a_i\rangle^{\dagger}\langle b_j\left.\right|a_k\rangle\\
      U_{jk}^{\dagger}&=\sum_j\langle a_j\left.\right|b_j\rangle\langle b_j\left.\right|a_k\rangle\\
      &=\langle a_j\left.\right|\hat{1}\left.\right|a_k\rangle\\
      &=\langle a_j\left.\right|a_k\rangle\\
      &=S_{ij}
      \end{aligned}
    \end{equation}
\end{itemize}
\textbf{Kontinuierliche Eigenbasen}
\begin{itemize}
	\item Betrachte hermiteschen Operator $\hat{B}$ mit kontinuierlichem Spektrum, d.h.
    \begin{equation}
    \hat{B}\left.\right|b\rangle =b\left.\right|\rangle
    \end{equation}
    mit kontinuierlichen reellen EW.
  \item Orthogonalität gilt wie bei diskreten Basen
    \begin{equation}
      \langle b\left.\right|b'\rangle=0 \quad \text{für} \quad b\neq b'
    \end{equation}
  \item Komplizierter werden Normierung und Vollständigkeit
  \item Vollständigkeit: Analogie zur diskreten Basis $\{\left.\right|a_i\rangle\}$
    \begin{equation}
      \left.\right|\gamma\rangle =\sum_i\left.\right|a_i\rangle \langle a_i\left.\right|\gamma\rangle
    \end{equation}
    Übergang zu kontinuierlicher Basis $\{\left.\right|b\rangle\} \rightarrow$\\
    Übergang von Summe $\sum_i$ nach Integral $\int db$
    \begin{equation}
    \left.\right|\gamma\rangle = \int db \left.\right|b\rangle \underbrace{\langle b\left.\right|\gamma\rangle}_{\text{kont. Funktion von b}}
    \label{eq:integral}
    \end{equation}
  \item Zerlegung der $\hat{1}$ in kontinuierlicher Basis
    \begin{equation}
      \hat{1} = \int db \left.\right|b\rangle\langle b\left.\right|
    \end{equation}
  \item Normierung: Multipliziere (\ref{eq:integral}) von links mit $\langle b'\left.\right|$
    \begin{equation}
      \langle b'\left.\right|\gamma\rangle =\int db \langle b'\left.\right|b\rangle\langle b \left.\right|\gamma\rangle
    \end{equation}
    vergleiche mit
    \begin{equation}
    \begin{aligned}
    &f(x') =\int dx \delta(x-x')f(x)\\
    &\Rightarrow \langle b'\left.\right|b\rangle = \delta(b'-b)
    \end{aligned}
    \end{equation}
    Orthonormierungsbedingung für kontinuierliche Basen
  \item physikalische Interpretationen wird durch $\delta$-Distributionen schwieriger
  \item Spektraldarstellung für $\hat{B}$ mit konstanter Eigenbasis
    \begin{equation}
    \hat{b}=\langle b\left.\right|b\left.\right|b\rangle %seltsam
    \end{equation}
\end{itemize}


\subsection{Kommutatoren und simultane Eigenbasen}
\begin{itemize}
	\item physikalische Auswirkungen der Nicht-Vertauschbarkeit des Operatorprodukts formalisieren 
  \item Kommutation zweier Operatoren $\hat{A}$ und $\hat{B}$
    \begin{equation}
    \left[\hat{A},\hat{B}\right]=\hat{A}\hat{B}-\hat{B}\hat{A}
    \end{equation}
  \item Kommutierende Operatoren
    \begin{equation}
    \Leftrightarrow \left[\hat{A},\hat{B}\right]=0
    \end{equation}
  \item Rechenregeln
    \begin{equation}
    \begin{aligned}
      &\left[\hat{A},\hat{B}\right]=-\left[\hat{B},\hat{A}\right]\\
      &\left[\hat{A},\hat{B}\right]^\dagger=\left[\hat{B}^\dagger,\hat{A}^\dagger\right]\\
      &\left[\hat{A},\hat{B}+\hat{C}\right] =\left[\hat{A},\hat{B}\right]+\left[\hat{A},\hat{C}\right]\\
      &\left[\hat{A},\hat{B}\hat{C}\right]=\left[\hat{A},\hat{B}\right]\hat{C}+\hat{B}\left[\hat{A},\hat{C}\right]\\
      &\left[\hat{A},\left[\hat{B},\hat{C}\right]\right]+
        \left[\hat{C},\left[\hat{A},\hat{B}\right]\right]+
        \left[\hat{B},\left[\hat{C},\hat{A}\right]\right]=0
        \quad\text{(Jacobi-Identität)}
    \end{aligned}
    \end{equation}
  \item triviale Kommutation
    \begin{equation}
    \begin{aligned}
      &\left[\hat{A},\hat{1}\right]=0\\
      &\left[\hat{A},\hat{A}\right]=0
    \end{aligned}
    \end{equation}
\end{itemize}


\begin{flushright}
Vorlesung 6 - 2.05.13
\end{flushright}
\textbf{Simultane Eigenbasen}
  \begin{itemize}
    \item Betrachte zwei kommutierende hermetische Operatoren $\hat{A},\hat{B}$
      \begin{equation}
        \left[\hat{A},\hat{B}\right]=0
      \end{equation}
      Es existiert eine simultane Eigenbasis $\left\{\left.\right|ab\rangle\right\}$
      \begin{equation}
      \begin{aligned}
      \hat{A}\left.\right|ab\rangle&=a\left.\right|ab\rangle\\
      \hat{B}\left.\right|ab\rangle&=b\left.\right|ab\rangle
      \end{aligned}
      \end{equation}
    \item Beweis: Angenommen $\left[\hat{A},\hat{B}\right]=0$ und betrachte 
      \begin{equation}
      \hat{A}\left.\right|a\rangle=a\left.\right|a\rangle
      \end{equation}
      Multipliziere mit $\hat{B}$
      \begin{equation}
      \begin{aligned}
      \hat{B}\hat{A}\left.\right|a\rangle=&\hat{B}a\left.\right|a\rangle\\
      \hat{A}\left(\hat{B}\left.\right|a\rangle\right)&=a\left(\hat{B}\left.\right|a\rangle\right)
      \end{aligned}
      \end{equation}
        $\Rightarrow$ $\left(\hat{B}\left.\right|a\rangle\right)$ erfüllt noch immer EW-Relation für $\hat{A}$
    \item Für nicht entartetes a-Spektrum:\\
      $\left(\hat{B}\left.\right|a\rangle\right)$ muss proportional zu $\left.\right|a\rangle$ sein:
      \begin{equation}
      \Rightarrow \hat{B}\left.\right|a\rangle=b\left.\right|a\rangle
      \end{equation}
      Daher sind $\left.\right|a\rangle$ simultane EV zu $\hat{A}$ und $\hat{B}$ $\rightarrow$ $\left.\right|ab\rangle$
    \item Bei entarteten a-Spektrum:\\
      Entartungsindex i in $\hat{A}$ EW-Relation
      \begin{equation}
      \hat{A}\left.\right|ai\rangle =a\left.\right|ai\rangle
      \end{equation}
      $\rightarrow \left(\hat{B}\left.\right|ai\rangle\right)$ ist weiterhin EV zu $\hat{A}$ aber nicht notwendig proportional zu $\left.\right|ai\rangle$\\
      $\rightarrow$ Wähle orhonormierte Basis im entarteten Unterraum so, dass $\hat{B}\left.\right|ai\rangle$ prop. zu $\left.\right|ai\rangle$, d.h. wir lösen EW von $\hat{B}$ im ent. Unterraum
    \item Wenn keine weiteren Entartungen auftreten, dann ist simultane EB eindeutig charakterisiert durch 
      \begin{equation}
      \begin{aligned}
        \hat{A}\left.\right|ab\rangle&=a\left.\right|ab\rangle\\
        \hat{B}\left.\right|ab\rangle&=b\left.\right|ab\rangle
      \end{aligned}
      \end{equation}
      $\rightarrow$ Entartung vom $\hat{A}$-Spektrum durch Hinzunahme von $\hat{B}$ eliminiert oder gehoben.
    \item Vollständiger Satz von kommutierenden Operatoren:\\
      Für einen Satz von Operatoren $\left\{\hat{A},\hat{B},\hat{C},...\right\}$, die paarweise miteinander kommutieren
      \begin{equation}
      \left[\hat{A},\hat{B}\right]=\left[\hat{A},\hat{C}\right],...=0
      \end{equation}
      zu dem sich kein weiterer komm. Operator finden lässt, gibt es eine simultane Eigenbasis, 
      \begin{equation}
      \begin{aligned}
        \hat{A}\left.\right|a,b,c...\rangle&=a\left.\right|a,b,c...\rangle\\
        \hat{B}\left.\right|a,b,c...\rangle&=b\left.\right|a,b,c...\rangle
      \end{aligned}
      \end{equation}
      die keine Entartung aufweist.
  \end{itemize}
\textbf{Unschärferelationen}
  \begin{itemize}
    \item Für nicht kommutierende Operatoren $\hat{A},\hat{B}$ mit 
      $$
      \left[\hat{A},\hat{B}\right]\neq 0
      $$
      existiert keine simultane Eigenbasis
    \item Trotzdem wichtige Aussage zur Unschärfe möglich
    \item Unschärfe oder Varianz des EW von A mit $\left.\right|\gamma\rangle$
      \begin{equation}
      \begin{aligned}
        \Delta A_\gamma&=\sqrt{\langle\gamma\left.\right|\left(\hat{A}-\langle\gamma\left|\hat{A}\right|\gamma\rangle\right)^2\left.\right|\gamma\rangle}\\
        &=\sqrt{\langle\gamma\left|\hat{A}^2\right|\gamma\rangle-\langle\left|\hat{A}\right|\gamma\rangle^2}\\
        &=\sqrt{\langle\hat{H}^2\rangle_\gamma-\langle\hat{A}\rangle_\gamma^2}
      \end{aligned}        
      \end{equation}
      mit $\hat{A}=\langle\gamma\left|\hat{A}\right|\gamma\rangle$
    \item Für das Produkt der Unschärfe $\Delta A_j$ und $\Delta B_j$ gilt verallgemeinerte Unschärferelation
      \begin{equation}
      \Delta A_\gamma \Delta B_\gamma \geq \frac{1}{2}\left|\langle\gamma\left|\left[ \hat{A},\hat{B}\right]\right|\gamma\rangle\right|
      \end{equation}
    \item Beweis er Schwarzsche Ungleichung
      \begin{equation}
      \langle\alpha\left.\right|\alpha\rangle\langle\beta\left.\right|\beta\rangle \geq \left|\langle\alpha\left|\right.\beta\rangle\right|^2
      \end{equation}
      Definition von 
      \begin{equation}
      \begin{aligned}
        \left.\right|\alpha\rangle=&\left(\hat{A}-\langle\hat{A}_\gamma\right)\left.\right|\gamma\rangle\\
        \left.\right|\beta\rangle=&\left(\hat{B}-\langle\hat{B}_\gamma\right)\left.\right|\gamma\rangle
      \end{aligned}
      \end{equation}
      Damit 
      \begin{equation}
      \Delta A_\gamma^2\Delta B_\gamma^2 \geq \left|\langle \left(\hat{A}-\langle\hat{A}\rangle_\gamma\right)\left(\hat{B}-\langle\hat{B}\rangle_\gamma\right)\rangle\right|^2
      \end{equation}
      Nutze Darstellung des Produkts zweier Operatoren über Kommutator $+$ Antikommutator
      \begin{equation}
      \begin{aligned}
        \hat{C}\hat{D}&=\frac{1}{2}\left[\hat{C},\hat{D}\right]+\frac{1}{2}\underbrace{\left\{\hat{C},\hat{D}\right\}}_{\text{Antikommutator}}\\
        &=\frac{1}{2i}\left(i\left[\hat{C},\hat{D}\right]\right)+\frac{1}{2}\left\{\hat{C},\hat{D}\right\} \quad \quad \text{; macht $\left[\hat{C},\hat{D}\right]$ hermitesch}
      \end{aligned}
      \end{equation}
      Mit $\hat{C}=\hat{A}-\langle\hat{A}\rangle_\gamma$, $\hat{D}=\hat{B}-\langle\hat{B}\rangle_\gamma$
      \begin{equation}
      \begin{aligned}
        &\left|\langle\left(\hat{A}\langle\hat{A}\rangle_\gamma\right)\left(\hat{B}\langle\hat{B}\rangle_\gamma\right)\rangle_\gamma\right|^2\\
        &=\left|\langle\hat{C},\hat{D}\rangle_\gamma\right|^2\\
        &=\left|\langle\frac{1}{2i}\left(i\left[\hat{C},\hat{D}\right]\right)+\frac{1}{2}\left\{\hat{C},\hat{D}\right\}\rangle_\gamma\right|^2\\
        &=\left|\frac{1}{2i}\langle i\left[\hat{C},\hat{D}\right]\rangle_\gamma+\frac{1}{2}\langle\left\{\hat{C},\hat{D}\right\}\rangle_\gamma\right|^2\\
        &=\frac{1}{4}\left|\langle\left[\hat{C},\hat{D}\right]\rangle_\gamma\right|^2+\frac{1}{4}\left|\langle\left\{\hat{C},\hat{D}\right\}\rangle_\gamma\right|^2
      \end{aligned}
      \end{equation}
    \item Antikommutatorbeitrag $(\geq 0)$ wird auf rechter Seite der Schwarzschen Ungleichung weggelassen
      \begin{equation}
      \begin{aligned}
        \Delta A_\gamma^2 \Delta B_\gamma^2 &\geq \frac{1}{4}\left|\langle\left[\hat{C},\hat{D}\right]\rangle_\gamma\right|^2\\
        &=\frac{1}{4}\left|\langle\left[\hat{A},\hat{B}\right]\rangle_\gamma\right|^2\\
        \Rightarrow \Delta A_\gamma \Delta B_\gamma &\geq \frac{1}{2}\left|\langle\left[\hat{A},\hat{B}\right]\rangle\right|
      \end{aligned}
      \end{equation}
  \end{itemize}

\section{Postulate der Quantenmechanik}
\subsection{Die Postulate}
  \begin{itemize}
    \item Mathematischer Formalismus mit physikalischer Größe koppeln
      \begin{itemize}
        \item Wie lässt sich der Zustand eines Quantensystems zu einem festen Zeitpunkt t beschreiben?
        \item Wie sind Observablen formal repräsentiert?
        \item Wie wird die Dynamik eines Quantensystems beschrieben?
      \end{itemize}
  \end{itemize}
  
  \textbf{1. Postulat: Zustand eines Systems}
    Der Zustand eines Systems zum \textbf{Zeitpunkt t} ist durch einen \textbf{normierten} Vektor $\left.\right|\Psi,t\rangle$ im \textbf{Hilbertraum} beschrieben. Der Zustandsvektor $\left.\right|\Psi,t\rangle$ enthält die \textbf{komplette Information} über das System.\\
  \textbf{2. Postulat: Observablen}\\
    Jede physikalische Observable wird durch einen hermiteschen linearen Operator auf dem Hilbertraum beschrieben. 
    \begin{itemize}
      \item klare Trennung zwischen $\underbrace{\text{Zustand}}_{\text{ket-Vektoren}}$ des Systems und $\underbrace{\text{Observablen}}_{\text{Operatoren}}$
      \item Observablen, die ein klassisches Analogon besitzen (z.B. Ort, Impuls, Drehimpuls, kin. Energie, Gesamtenergie...) werden über Korrespondenzregeln definiert. \\
        Einer Observable, die durch eine Funktion $F(\vec{x},\vec{p})$ in der klassischen Physik beschrieben wird, wird ein hermitescher Operator zugeordnet, der sich aus Einsetzung ergibt. 
        $$
        \vec{x} \rightarrow \hat{\vec{x}} \quad, \quad \vec{p} \rightarrow \hat{\vec{p}}
        $$
        Zwischen $\hat{\vec{x}}$ und $\hat{\vec{p}}$ gilt die fundamentale Kommutatorrelation
        \begin{equation}
        \left[\hat{\vec{x}}_i, \hat{\vec{p}}_j\right]=i\hbar\delta_{ij}
        \end{equation}
      \item Manchmal führt Einsetzung auf einen nicht hermiteschen Operator, z. B. die Radialkomponente des Impulses
        \begin{equation}
        \begin{aligned}
          \text{klassisch:} \quad &\vec{x}\vec{p}\\
          \text{QM:} \quad &\frac{1}{2}\left(\hat{\vec{x}}\hat{\vec{p}}+\hat{\vec{p}}\hat{\vec{x}}\right)
        \end{aligned}
        \end{equation}
        $\rightarrow$ explizite "`Hermitesierung"' notwendig
      \item Bsp.: Energie der Observable:
        \begin{equation}
        \begin{aligned}
          \text{klassisch: Hamiltonfunktion} \quad H(\vec{x},\vec{p})&=\frac{\vec{p}}{2m} + V(\vec{x})\\
          \text{QM: Hamiltonoperator} \quad \hat{H}(\hat{\vec{x}},\hat{\vec{o}})&=\frac{\hat{\vec{p}}^2}{2m}+V(\hat{\vec{x}})
        \end{aligned}
        \end{equation}
      \item Analogie des fundamentalen Kommutators mit Poisson-Klammern der klassischen Mechanik
        \begin{equation}
         \left\{x_i,p_j\right\}=\delta_{ij}
        \end{equation}
        mit 
        \begin{equation}
        \left\{A,B\right\}:=\sum_j\left(\frac{\partial A}{\partial x_i}\frac{\partial B}{\partial p_i}-\frac{\partial A}{\partial p_i}\frac{\partial B}{\partial x_i}\right)
        \end{equation}
      \item Observablen ohne klassische Analogien (z.B. Spin) müssen im Rahmen der QM konstruiert werden
    \end{itemize}
  \textbf{3. Postulat: ideale Messung}\\
    Bei einer idealen Messung einer Observablen $\hat{A}$ an einem System im Zustand $\left.\right|\Psi,t\rangle$ kann nur einer der (diskreten) Eigenwerte des Operators $\hat{A}$ resultieren.\\
    Der Zustand des Systems unmittelbar nach der Messung ist durch den Eigenvektor zum gemessenen Eigenwert gegeben.
    \begin{itemize}
      \item Direkter Zusammenhang zwiwchen EW-Spektrum des Operators der Observable und dem wirklichen Messergebnis
        \begin{itemize}
          \item diskretes Spektrum: nur spezielle, diskrete Messergebnisse sind möglich. $\rightarrow$ Quantisierung
          \item kontinuierliche Spektren oder bei Entartung ein kleiner Unterraum, welcher sich aus Messunsicherheit bei kont. Observablen oder Entartung ergibt, ist nach der Messung als Information über den Zustand verfügbar
          \item Messung ist ein ideales Werkzeug zur Präparation von Zuständen
        \end{itemize}
    \end{itemize}
    Nachtrag: 
    \begin{equation}
      \hat{\vec{x}}=\left(\begin{array}{c} \hat{x}_1 \\ \hat{x}_2 \\ \hat{x}_3 \end{array}\right) \quad, \quad \hat{\vec{x}}\hat{\vec{p}}=\hat{x}_1\hat{p}_1 + \hat{x}_2\hat{p}_2 + \hat{x}_3\hat{p}_3 \quad, \quad \hat{\vec{x}}\left.\right|\hat{x}\rangle=\vec{x}\left.\right|\vec{x}\rangle
    \end{equation}
    \begin{equation}
    \left(\begin{array}{c} \hat{x}_1 \\ \hat{x}_2 \\ \hat{x}_3 \end{array}\right)\left.\right|\vec{x}\rangle=\left(\begin{array}{c} x_1 \\ x_2 \\ x_3 \end{array}\right)\left.\right|\vec{x}\rangle
    \end{equation}
    \begin{equation}
    \underbrace{\hat{x}_1}_{\text{Operator}}\left.\right|x_1x_2x_3\rangle=\underbrace{x_1}_{\text{Eigenwert}}\left.\right|x_1x_2x_3\rangle
    \end{equation}
  \textbf{4. Postulat: Wahrscheinlichkeitsaussage}\\
    Für ein Spektrum im Zustand $\left.\right|\Psi,t\rangle$ liefert das Skalarprodukt mit den Eigenzuständen der Observable eien Wahrscheinlichkeitsamplitude für die Messung der zugeordneten, zugehörigen Eigenwerte.
\end{fleqn}
\end{document}