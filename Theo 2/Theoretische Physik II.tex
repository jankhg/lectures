\documentclass[10pt,article,colorback,accentcolor=tud9d]{scrartcl}
\usepackage[utf8]{inputenc} %?%
\usepackage[T1]{fontenc}
\usepackage{lmodern} %cool beans%
\usepackage[english,ngerman]{babel} %deutsche Schriftzeichen%
\usepackage{amsmath} %für Formeln%
\usepackage{float} %Für Position von Abbildungen%
\usepackage[singlespacing]{setspace}
\usepackage{nccmath}
\usepackage{ulem} % Unterstreichungen
\usepackage[colorlinks,
pdfpagelabels,
pdfstartview = FitH,
bookmarksopen = true,
bookmarksnumbered = true,
linkcolor = black,
plainpages = false,
hypertexnames = false,
citecolor = black] {hyperref} %Für verlinktes Inhaltsverzeichnis% 
\usepackage[a4paper, left=2cm, right=2cm, top=2cm]{geometry}%Ränder
\usepackage{amssymb}%Für R reelle zahlen
\usepackage{enumitem}% small item distance
\setlist[1]{itemsep=-2pt}

\title{Theoretische Physik II}
\subtitle{Prof. Robert Roth - SS 2013}
\author{Jan Krause und Philipp Dijkstal}
\date{ }
%\subsubtitle{email: \textaccent{philipp.dijkstal@web.de}}
%\institution{Fachbereich Physik}
\begin{document}
\maketitle
\tableofcontents
\newpage
\begin{flushright}Vorlesung 2 - 18.04.2013\end{flushright}
\section{Formalismus der Quantenmechanik}
$\rightarrow$ mathematisches Handwerkszeug bereitstellen
\begin{itemize}
\item Lineare Algebra
\item später: Analysis, Funktionentheorie
\end{itemize}
\subsection{Vektoren und Hilbertraum}
\begin{itemize}
\item Grundelement: Vektoren
\item Schreibweise: Dirac-Notation\\
$\hookrightarrow$ Ket-Symbole bzw. Kets\\
$\left| \quad \rangle  \right.$ $\rightarrow$ Bezeichnungen innerhalb Symbol\\
Bsp.: $\left| 0\rangle \right.,\left|1\rangle \right.,\left|n\rangle \right.,\left|nlm\rangle \right.,
\left| \uparrow\rangle \right.$
\item Vektorraum: Ket-Vektoren sind Elemente eines linearen kompletten 
Vektorraums $(\nu)$\\
$\left|1\rangle \right.,\left|2\rangle \right.,...,\left|\alpha\rangle \right.,...,\left|\beta\rangle 
\right.,... \in \nu$\\
auf dem Addition zweier Vektoren 2 Multiplikationen mit $\mathbb{C}$-Zahl 
definiert
\item Eigenschaften der Addition und Multiplikation einer Zahl aus $\mathbb{C}$
 
\begin{itemize}
\item Abgeschlossenheit: 
\begin{equation} 
\left| \alpha \rangle \right.+\left|\beta\rangle \right. \in \nu
\end{equation}
\item Distributivität: 
\begin{equation}
\begin{aligned}
C(\left|\alpha\rangle \right. +\left|\beta\rangle \right.) = C\left|\alpha\rangle \right. +C\left|
\beta\rangle \right.\\
(C+D)\left|\alpha\rangle \right. = C\left|\alpha\rangle \right. +D\left|\alpha\rangle \right.
\end{aligned}
\end{equation}
\item Assoziativität: 
\begin{equation}
C(D\left|\alpha\rangle \right.)=(CD)\left|\alpha\rangle \right.
\end{equation}
\item Kommutativität der Addition: 
\begin{equation}
\left|\alpha\rangle \right. + \left|\beta\rangle \right. =\left|\beta\rangle \right. + \left|\alpha
\rangle \right.
\end{equation}
\item Assoziativität der Addition: 
\begin{equation}
\left|\alpha\rangle \right. + (\left|\beta\rangle \right. + \left|\gamma\rangle \right.) = (\left|
\alpha\rangle \right. + \left|\beta\rangle \right.)+\left|\gamma\rangle \right.
\end{equation}
\item Nullvektor: 
\begin{equation}
\begin{aligned}
\left|\alpha\rangle \right. +\left|0\rangle \right. = \left|\alpha\rangle \right.\\
0 \left|\alpha\rangle \right. = \left|0\rangle \right.\\
C \left|0\rangle \right. = \left|0\rangle \right.
\end{aligned}
\end{equation}
\item Inverses Element bezüglich Addition:
\begin{equation}
\begin{aligned}
\left|\alpha\rangle \right. + \left|{\alpha}_{inv}\rangle \right. = \left|0\rangle \right.\\
\left|{\alpha}_{inv}\rangle \right. = -\left|\alpha\rangle \right.
\end{aligned}
\end{equation}
\end{itemize}
 
\item Lineare Unabhängigkeit\\
Der Satz
 
\begin{equation} \nonumber
{\left|1\rangle \right.,\left|2\rangle \right.,...,\left|n\rangle \right.}
\end{equation}
ist genau dann linear unabhängig, wenn
\begin{equation}
\sum {C}_i \left|i\rangle \right. = \left|0\rangle \right.
\end{equation}
 
nur für ${C}_1, {C}_2,...,{C}_n=0$\\
erfüllt ist.
\item Dimension des Vektorraums $\nu$ maximale Zahl linear unabhängiger 
Vektoren. 
\item Basisentwicklung: Jeder Vektor $\left|\alpha\rangle \right.$ eines n-dim. 
Vektorraums lässt sich eindeutig durch Linearkombination von n Basisvektoren \\
${\left|1\rangle \right.,\left|2\rangle \right.,...,\left|n\rangle \right.}$ mit Koeffizienten ${C}
_1,...{C}_n$\\
darstellen:\\
$\left|\alpha\rangle \right. = \sum {C}_i \left|i\rangle \right.$
\end{itemize}
\subsection{Skalarprodukt und Hilbertraum}
\begin{itemize}
\item Skalarprodukt ordnet zwei Vektoren $\left|\alpha\rangle \right.,\left|\beta\rangle 
\right.$ eine $\mathbb{C}$-Zahl zu.\\
Typische Schreibweisen:
 
\begin{equation}
\vec{a}\vec{b} \quad (\vec{a}\vec{b})
\end{equation}
In Diracnotation:
\begin{equation}
\langle \alpha\left|\right.\beta\rangle  \ =\langle \alpha \left|\right|\beta\rangle 
\end{equation}
$\hookrightarrow$ bra(c)ket
\item Eigenschaften
\begin{itemize}
\item Schiefsymmetrie:
\begin{equation}
\begin{aligned}
\langle \alpha\left|\right.\beta\rangle  \ =(\langle \beta\left|\right.\alpha\rangle )^*
\end{aligned}
\end{equation}
\item Positive Semidefinit: 
\begin{equation}
\langle \alpha\left|\right.\alpha\rangle  \ \ \geq \ 0
\end{equation}
wobei: $\langle \alpha\left|\right.\alpha\rangle  \ \ = 0 \Leftrightarrow \left|
\alpha\right.\rangle  \ \ = \ \left|0\right.\rangle $
\item Kombination von Schief-Symmetrie und Linearität im Ket:
\begin{equation}
\begin{aligned}
\left|\right.\psi\rangle  \ = C\left|\right.\beta\rangle +D\left|\right.\gamma\rangle \\
\langle \alpha\left|\right.\psi\rangle  \ =C\langle \alpha\left|\right.\beta\rangle  +D\langle \alpha\left|\right
.\gamma\rangle \\
\langle \psi\left|\right.\alpha\rangle  \ =\langle \alpha\left|\right.\psi\rangle ^* \\
 =(C \langle \alpha\left|\right.\beta\rangle  + D\langle \alpha\left|\right.\gamma\rangle )^*\\
=C^*(\langle \alpha\left|\right.\beta\rangle )^* + ^*(\langle \alpha\left|\right.\gamma\rangle )^*\\
=C^*\langle \beta\left|\right.\alpha\rangle  +D^*\langle \gamma\left|\right.\alpha\rangle 
\end{aligned}
\end{equation}
\end{itemize}
 
\item Daher interpretiert man den ersten Faktor im Skalarprodukt als "`
anderen Typ"' von Vektor: Bra-Vektoren
\item Jedem Ket-Vektor $\left|\right.\alpha\rangle $ ist über duale Korrespondenz 
ein Bra-Vektor zugeordnet\\
$\left|\right.\alpha\rangle  \ \ \rightarrow \ \ \langle \alpha\left|\right.$\\
Entscheidende Eigenschaft:
 
\begin{equation}
\begin{aligned}
\left|\right.\psi\rangle  \ &= C \left|\right.\alpha\rangle  +D\left|\right.\beta\rangle \\
&\Updownarrow\\
\langle \psi\left|\right.&=C^*\langle \alpha\left|\right. +D^* \langle \beta\left|\right.
\end{aligned}
\end{equation}
 
\item Die Reihenfolge von Bra und Ket im Skalarprodukt ist wichtig!
\item Manchmal auch:
 
\begin{equation}
\langle \alpha\left|\right.=(\left|\right.\alpha\rangle )^\dagger \leftarrow \text{hermitesch 
Adjungiert}
\end{equation}
 
\item Begriffe im Zusammenhang mit Skalarprodukt:
\begin{itemize}
\item Orthogonalität: Zwei Vektoren $\left|\right.\alpha\rangle , \left|\right.\beta\rangle 
$ sind orthogonal, wenn
 
\begin{equation}
\langle \alpha\left|\right.\beta\rangle  \ =0
\end{equation}
 
\item Norm: Die Norm eines Vektors $\left|\right.\alpha\rangle $ ist 
 
\begin{equation}
\left|\right|\left|\right.\alpha\rangle \left|\right|=\sqrt{\langle \alpha\left|\right.
\alpha\rangle }
\end{equation}
 
Ein Vektor ist normiert, wenn
 
\begin{equation}
\left|\right|\left|\right.\alpha\rangle \left|\right| =1 \quad \langle \alpha\left|\right.
\alpha\rangle 
\end{equation}
 
\end{itemize}
\item Orhonormierte Basis: Basis
 
\begin{equation} \nonumber
{\left|\right.1\rangle ,\left|\right.2\rangle ,...,\left|\right.i\rangle ,...,\left|\right.j\rangle ,...}
\end{equation}
heißt orthognonal wenn
\begin{equation}
\langle i\left|\right. j\rangle  \ ={\delta}_{ij}
\end{equation}
 
\end{itemize}
\begin{flushright}Vorlesung 3 - 23.04.2013\end{flushright}
\section{Lineare Operatoren}
\begin{itemize}
\item Operatoren sind formales Werkzeug, um Vektoren zu manipulieren. \\
$\hookrightarrow$ 4 Operatoren beschreiben Abbildung von einem Ket $\left|\right. 
\alpha \rangle  \in \mathbb{H}$ auf einen anderen Ket  $\left|\right. \alpha' \rangle  \in 
\mathbb{H}$
\begin{equation}
  \left|\right. \alpha' \rangle  \ = \hat{A} \left|\right. \alpha \rangle 
\end{equation}
$\hookrightarrow$ Operatoren wirken immer nach rechts\\
$\hookrightarrow$ Hütchen zur Unterscheidung von Zahlen, Matrizen,...
\item In QM sind (fast) ausschließlich lineare Operatoren relevant.
\begin{equation}
\begin{aligned}
  \hat{A} (C\left|\right. \alpha \rangle  + D\left|\right. \beta \rangle )&\\
  &= \hat{A} C\left|\right. \alpha \rangle  + \hat{A} D \left|\right. \alpha \rangle \\
  &= C \hat{A} \left|\right. \alpha \rangle  + D \hat{A} \left|\right. \beta \rangle 
\end{aligned}
\end{equation}
\item Assoziativität und Distributivität 
\begin{equation}
\begin{aligned}
&(\hat{A} + \hat{B}) \left|\right. \alpha \rangle  \ = \hat{A} \left|\right. \alpha \rangle 
 + \hat{B} \left|\right. \alpha \rangle \\
&(\hat{A} \hat{B})\left|\right. \alpha \rangle  \ = \hat{A} (\hat{B} \left|\right. 
\alpha \rangle )
\end{aligned}
\end{equation}
\item Rechnen mit isolierten Operatoren z.B.
\begin{equation}
\begin{aligned}
  &\hat{A} + \hat{B} = \hat{B} + \hat{A} \\
  &(\hat{A} + \hat{B}) + \hat{C}= \hat{A} +(\hat{B}+ \hat{C})
\end{aligned}
\end{equation}
Für Produkte von Operatoren
\begin{equation}
\begin{aligned}
  &\hat{A}(\hat{B} \hat{C}) = (\hat{A} \hat{B})\hat{C}
  &\hat{A}(\hat{B} + \hat{C}) = \hat{A}\hat{B} + \hat{B}\hat{C}
\end{aligned}
\end{equation}
\item Vorsicht: Multiplikation ist nicht kommutativ
\begin{equation}
\hat{A}\hat{B} \not= \hat{B}\hat{A}
\end{equation}
$\hookrightarrow$ Reihenfolge ist wesentlich
\item Spezielle Operatoren
  \begin{itemize}
  \item Eins-Operator 
   \begin{equation}
  \hat{1}: \quad \hat{1}\left|\right. \alpha \rangle  \ = \left|\right. \alpha \rangle 
  \end{equation}
  \item Inverser Operator 
  \begin{equation}
  \begin{aligned}
  &\hat{A}^{-1}:\quad {\hat{A}}^{-1} \hat{A}\left|\right. \alpha \rangle  \ = \left|\right
. \alpha \rangle \\
  &\text{oder} \quad \hat{1} = {\hat{A}}^{-1}\hat{A} = \hat{A} {\hat{A}}^{-1}
  \end{aligned}
  \end{equation}
  \end{itemize}
\item Spezieller Operator: dyadisches Produkt
  \begin{equation}
  \left|\right. \beta' \rangle \langle \beta\left|\right.
  \end{equation}
  Anwendung auf Ket
  \begin{equation}
  (\left|\right. \beta' \rangle \langle \beta\left|\right.)\left|\right.\alpha\rangle  \ = \left|
\right.\beta'\rangle \langle \beta\left|\right.\alpha\rangle  \in \mathbb{C}
  \end{equation}
  $\hookrightarrow$ Wichtig z.B. für Spektraldarstellung von Operatoren
\item Matrixelemente von Operatoren\\
  Betrachte Transformation von $\left|\right. \beta \rangle $ mittels $\hat{A}$
  \begin{equation}
  \left|\right. \beta'\rangle  \ = \hat{A}\left|\right. \beta\rangle 
  \end{equation}
  Skalarprodukt mit $\langle \alpha \left|\right.$
  \begin{equation}
  \langle \alpha\left|\right.\beta'\rangle  \ = \underbrace{\langle \alpha\left|\right.\hat{A}\left
|\right.\beta\rangle }_{\text{Matrixelement}}
  \end{equation}
\item Erwartungswert eines Operators: Diagonales Matrixelement
  \begin{equation}
  \langle\hat{A}\rangle _\alpha = \langle \alpha \left|\right.\hat{A}| \alpha\rangle 
  \end{equation}
 
\end{itemize}
\textbf{Hermitesche Adjunktion}
 
\begin{itemize}
\item Übertragung der hermiteschen adjungierten für Matrizen auf Operatoren
  \begin{equation}
  \langle \alpha\left|\right.\hat{A}\left|\right.\beta\rangle  \ \xrightarrow{\text{herm. adj.}} (\langle 
\beta \left|\right.\hat{A}\left|\right.\alpha\rangle )^*
  \end{equation}
  Definition hermitesch adjungierter Operator
  \begin{equation}
  \langle \alpha\left|\right.{\hat{A}}^{\dagger} \left|\right. \beta\rangle  \ =(\langle \beta\left|\right.
\hat{A}\left|\right.\alpha\rangle )^*
  \end{equation}
\item Aus Definition folgen Rechenregeln
  \begin{equation}
  \begin{aligned}
  &(C\hat{A})^{\dagger}=C^*{\hat{A}}^{\dagger}\\
  &({\hat{A}}^{\dagger})^{\dagger}=\hat{A}\\
  &(\hat{A} + \hat{B})^{\dagger} )= \hat{A}^{\dagger} + \hat{B}^{\dagger}\\
  &(\hat{A}\hat{B})^{\dagger}=\hat{B}^{\dagger}\hat{A}^{\dagger}
  \end{aligned}
  \end{equation}
\item  Universalrezept für hermitesche Adjunktion\\
  1) Zyklische Umkehr der Reihenfolge\\
  2) Komplex konjugieren der $\mathbb{C}$-Zahl\\
  3) Ersetzung der Bras durch Kets u.U.\\
  4) hermitesche Adjungierte der Einzeloperatoren\\
  Bsp.:
  \begin{equation}
  \begin{aligned}
  &(\langle \alpha\left|\right.\hat{A}^{\dagger}+C\hat{A}\hat{B}\left|\right.\beta\rangle )^{\dagger}\\
  &= (\left|\right.\beta\rangle )^{\dagger} (\hat{A}^{\dagger}+C\hat{A}\hat{B})^{\dagger}(\langle \alpha\left|\right
.)^{\dagger}\\
  &= \langle \beta\left|\right. \hat{A} +C^*\hat{B}^{\dagger} \hat{A}^{\dagger} \left|\right.\alpha\rangle 
  \end{aligned}
  \end{equation}
  Achtung: Ausdrücke wie $|\alpha\rangle \hat{A}$ oder $ \hat{A}\langle \alpha\left|\right.$ 
sind sinnnlos!
\item Duale Korrespondenz
  \begin{equation}
  \begin{aligned}
  \left|\right.\alpha'\rangle &=\hat{A}\left|\right.\alpha\rangle \\
    &\updownarrow\\
  \langle \alpha'\left|\right.&= \ \langle \alpha\left|\right.\hat{A}^{\dagger}
  \end{aligned}
  \end{equation}
\end{itemize}

\begin{flushright}
Vorlesung 4 - 25.04.2013
\end{flushright}

\subsection{Hermitesche und Unitäre Operatoren}
\begin{itemize}
  \item Für hermitesche Operatoren gilt
    \begin{equation}
    \hat{A}^{\dagger}=\hat{A}
    \end{equation}
    selbst-adjungiert
  \item Erwartungswerte von hermiteschen Operatoren sind reell.
    \begin{equation}
    \langle \alpha|\hat{A}|\alpha\rangle  \ =  \left(\langle \alpha | \hat{A}^{\dagger} | \alpha \rangle \right)^* = \left(\langle \alpha |\hat{A}|\alpha\rangle \right)^*
    \end{equation}
    $\Rightarrow$ $\langle \alpha|\hat{A}|\alpha\rangle  \in \mathbb{R}$
  \item Für unitäre Operatoren gilt
    \begin{equation}
    \hat{U}^{\dagger}=\hat{U}^{-1} \quad \text{bzw.} \quad \hat{U}^{\dagger}\hat{U}=\hat{1}
    \end{equation}
  \item Anwendung eines unitären Operatoren ändert Normierung nicht
    \begin{equation} \nonumber
     |\alpha'\rangle  \ = \hat{U}|\alpha\rangle  \quad \langle \alpha'|= \ \langle \alpha|\hat{U}^{\dagger}
    \end{equation}
    \begin{equation}
    \begin{aligned}
    \langle \alpha'|\alpha'\rangle  \ =& \ \langle \alpha|\underbrace{\hat{U}^{\dagger}\hat{U}}_{\hat{1}}|\alpha\rangle \\
    &= \ \langle \alpha|\hat{1}|\alpha\rangle 
    \end{aligned}
    \end{equation}
    \end{itemize}
\subsection{Funktionen von Operatoren}
\begin{itemize}
  \item Funktionen von Operatoren sind nur über Reihendarstellungen definiert z.B.
    \begin{equation}
    F(\hat{A})=\sum^\infty_{n=0}a_n\hat{A}^n
    \end{equation}
  \item Bsp: Exponentialfunktion
    \begin{equation}
    \exp(c\hat{A})= \sum^\infty_{n=0}\frac{c^n}{n!}\hat{A}^n
    \end{equation}
  \item Durch Nicht-Kommutativität des Produkts von Operatoren gelten viele Rechenregeln für spezielle Funktionen \textbf{nicht} mehr z.B.
  \begin{equation}
  \exp(\hat{A}+\hat{B})\neq \exp(\hat{A}) \cdot\exp(\hat{B}), \quad \text{wenn} \quad \hat{A}\hat{B} \neq \hat{B}\hat{A}
  \end{equation}
\end{itemize}
\subsection{Eingenwertproblem \& Darstellungen}
\begin{itemize}
  \item Gegeben sei Operator $\hat{A}$. Ket-Vektoren $|a\rangle $ heiße Eigenvektor zum Eigenwert a, wenn
    \begin{equation}
    \hat{A}|a\rangle  \ = a|a\rangle 
    \end{equation}
    Die Eigenwerte sind im Allgemeinen komplexe Zahlen.
  \item Spektrum des Operators: Gesamtheit der EW und EV.
  \item Diskrete EW / diskretes Spektrum oder kontinuierliche EW / kontinuierliches Spektrum oder Kombination
  \item Diskrete Spektren: endlich oder abzählbar unendlich viele EV\\
    $\hookrightarrow$ kompatibel mit mathematischem Hilbertraum
  \item kontinuierliche Spektrum: überabzählbar unendlich viele EV\\
    $\hookrightarrow$ erweiterter Hilbertraum
  \item Entartung: Mehrere Eigenzustände zum selben EW\\
    $\hookrightarrow$ Entartungsindex\footnote{Komma nicht nötig} $|a,n\rangle $
\end{itemize}
\subsection{Eingenwertproblem hermitescher Operatoren}
Besondere Eigenschaften für EW-Problem hermitescher Operatoren
  \begin{itemize}
    \item Eigenwerte hermitescher Operatoren sind reell\\
      $\hookrightarrow$ Beweis über reellen Erwartungswert und 
      \begin{equation}
      a=\frac{\langle a|\hat{A}|a\rangle }{\langle a|a\rangle }
      \end{equation}
    \item Eigenvektoren zu verschiedenen EW sind orthogonal\\ \\
      a) Ohne Entartung:
         \begin{equation}
        \label{eq:ortho}
          \langle a|a'\rangle  = 0 \quad \text{wenn} \quad a \neq a'
          \end{equation}
          $\hookrightarrow$ Beweis über $(a-a') \ \langle a|a'\rangle  = 0$\\ \\
      b) Mit Entartung:\\
         - Wie (a) für EV aus verschiedenen ent. Unterräumen\\
         - Innerhalb des ent. Unterraums muss explizit orthogonalisiert werden.
    \item Eigenvektoren bilden eine vollst. orthonormierte Basis\\
      - Orthogonalität aus (\ref{eq:ortho})\\
      - Normierung ist für diskrete Spektren einfach, für kontinuierliche Spektren ist Vorsicht geboten.\\
      - Vollständigkeit: bewiesen für diskrete Spektren aber für kontinuierliche Spektren ist Vollständigkeit mathematisch noch nicht gezeigt. $\rightarrow$ Postulat!
  \end{itemize}
\textbf{Diskrete Eigenbasen}
\begin{itemize}
  \item Eigenwertproblem eines herm. Op. $\hat{A}$ mit diskretem Spektrum
    \begin{equation}
    \hat{A}|a_i\rangle  \ = a_i|a_i\rangle  \quad i:\text{Zählindex}
    \end{equation}
  \item Vorausgesetzt EV sind normiert, dann gilt zusammen mit Orthogonalität 
    \begin{equation}
    \langle a_i|a_j\rangle  \ = \delta_{ij}
    \end{equation}
  \item Vollständigkeit: Entwicklung eines beliebigen Vektors $|\beta\rangle $ in Eigenbasis $|a_i\rangle $:
    \begin{equation}
    |\beta\rangle  \ = \sum_i C_i |a_i \rangle 
    \end{equation}
    - Multiplikation $\langle a_j|$
      \begin{equation}
      \begin{aligned}
      \langle a_j|\beta\rangle  \ &= \ \langle a_j|\left(\sum_i C_i|a_i\rangle \right)\\
      &=\sum_iC_i\langle a_j|a_i\rangle \\
      &=\sum_iC_i \delta_{ij}\\
      &=C_j
      \end{aligned}
      \end{equation}
    - Einsetzen in Entwicklung
    \begin{equation}
    \begin{aligned}
      |\beta\rangle  \ &= \sum_i\left(\langle a_i|\beta\rangle \right) |a_i\rangle \\
      &=\sum_i|a_i\rangle \langle a_i|\beta\rangle \\
      &=\left(\sum_i|a_i\rangle \langle a_i|\right)|\beta\rangle \\
      &=\hat{1}|\beta\rangle 
    \end{aligned}
    \end{equation}
  \item Zerlegung des $\hat{1}$-Operators
    \begin{equation}
    \hat{1}= \sum_i |a_i\rangle \langle a_i|
    \end{equation}
    $\rightarrow$ Sehr nützliches Rechenwerkzeug
  \item Beispiel:für "`triviale"' Rechnung:
    \begin{equation}
    \begin{aligned}
    \hat{A} &= \hat{1}\hat{A}\hat{1}\\
    &=\left(\sum_i|a_i\rangle \langle a_i|\right)\hat{A}\left(\sum_j|a_j\rangle \langle a_j|\right)\\
    &=\sum_{i,j}|a_i\rangle \langle a_i|\hat{A}|a_j\rangle \langle a_j|
    \end{aligned}
    \end{equation}
    $\rightarrow$ Nutze Eigenwertberechnung
    \begin{equation}
    \hat{A}|a_j\rangle =a_j|a_j\rangle  \nonumber
    \end{equation}
    \begin{equation}
    \begin{aligned}
    &\sum_{i,j}|a_i\rangle \langle a_i|\hat{A}|a_j\rangle \langle a_j|\\
    &=\sum_{i,j}a_j|a_i\rangle \underbrace{\langle a_i|a_j\rangle }_{\delta_{ij}}\langle a_j|\\
    &=\sum_ia_i|a_i\rangle \langle a_i|\\
    &=\sum_i|a_i\rangle a_i\langle a_i|
    \end{aligned}
    \end{equation}
    $\rightarrow$ Spektraldarstellung des Operators $\hat{A}$
  
\end{itemize}

\begin{flushright}
Vorlesung 5 - 30.04.2013
\end{flushright}
\textbf{Darstellung}\\
$\hookrightarrow$ normale Anschauung versagt.
\begin{itemize}
	\item Wir unterscheiden zwischen abstrakten Hilbertraumobjekten (Ket, Bra, Operatoren), die über Rechenregeln definiert sind, und den Darstellungen dieser Objekte.(Analogon zu Spalten- und Zeilenvektoren und Matrizen)
  \item Erinnerung: Koordinaten für Darstellung eines Vektors im 3D Raum(kartesisch, Kugelkoordinaten)
  \item Beispiel: Eigenwertproblem eines abstrakten Operators $\hat{B}$.
    \begin{equation}
    \hat{B}|b\rangle =b|b\rangle
    \end{equation}
    Übertragung in Darstellung durch Eigenbasis von $\hat{A}$, d.h. $\left\{|a_i\rangle\right\}$ mit
    \begin{equation}
    \begin{aligned}
    &\langle a_i|a_j\rangle=\delta_{ij}\\
    &\hat{1}=\sum_i|a_i\rangle\langle a_i|
    \end{aligned}
    \end{equation}
    Konstruktion der Darstellung
    \begin{equation}
    \begin{aligned}
      \hat{B}\hat{1}|b\rangle=&b|b\rangle\\
      \hat{B}\left(\sum_i|a_i\rangle\langle a_i|\right)|b\rangle=&b|b\rangle\\
      \langle a_j|\hat{B}\left(\sum_i|a_i\rangle\langle a_i|\right)|b\rangle'=&\langle a_j|b|b\rangle\\
      \sum_i\langle a_j|\hat{B}|a_i\rangle\langle a_i|b\rangle =&b\langle a_j|b\rangle      
    \end{aligned}
    \end{equation}
  In "`Matrixnotation\footnote{Indizes laufen in Pfeilrichtung}"'$\Rightarrow$ Matrix-Eigenwertproblem 
  \begin{equation}
    j \downarrow \left(\begin{array}{c} \rightarrow i \\ \langle a_j|\hat{B}|a_i\rangle \\ \left.\right.\end{array}\right)\cdot 
    \left(\begin{array}{c} i \downarrow \\ \langle a_i|b\rangle \\ \left.\right.\end{array}\right)=
    \left(\begin{array}{c} j \downarrow \\ \langle a_j|b\rangle \\ \left.\right.\end{array}\right)
  \end{equation}
\item Praktische Rechnung wird letztendlich in eine geschickt gewählten Darstellung ausgeführt.
\end{itemize}
\textbf{Darstellungswechsel}
\begin{itemize}
	\item Betrachte zwei Eigenbasen $\{|a_i\rangle\}$ und $\{|b_i\rangle\}$
  \item Allgemeiner Vektor $|\gamma\rangle$ sein in $\{|a_i\rangle\}$ Basis dargestellt, d.h. $\langle a_i|\gamma\rangle$ sind bekannt.
  \item Wie bekommt man daraus die Entwicklungskoeffizienten in $\{|b_i\rangle\}$?
  \begin{equation}
  \begin{aligned}
    \langle b_i |\gamma\rangle&=\langle b_i|\hat{1}|\gamma\rangle\\
    &=\langle b_i |\left(\sum_j|a_j\rangle\langle a_j|\right)|\gamma\rangle\\
    &=\sum_j\langle b_i|a_j\rangle\langle a_j|\gamma\rangle
  \end{aligned}
  \end{equation}
  \item Transformationsmatrix $U_{ij} =\langle b_ia_j\rangle$ ist unitär
    \begin{equation}
      U^{\dagger}U=\hat{1}
     \end{equation}
     \begin{equation}
      \begin{aligned}
      \sum_jU^{\dagger}_{ij}U_{jk}&=\sum_j\langle b_j|a_i\rangle^{\dagger}\langle b_j|a_k\rangle\\
      U_{jk}^{\dagger}&=\sum_j\langle a_j|b_j\rangle\langle b_j|a_k\rangle\\
      &=\langle a_j|\hat{1}|a_k\rangle\\
      &=\langle a_j|a_k\rangle\\
      &=S_{ij}
      \end{aligned}
    \end{equation}
\end{itemize}
\textbf{Kontinuierliche Eigenbasen}
\begin{itemize}
	\item Betrachte hermiteschen Operator $\hat{B}$ mit kontinuierlichem Spektrum, d.h.
    \begin{equation}
    \hat{B}|b\rangle =b|b\rangle
    \end{equation}
    mit kontinuierlichen reellen EW.
  \item Orthogonalität gilt wie bei diskreten Basen
    \begin{equation}
      \langle b|b'\rangle=0 \quad \text{für} \quad b\neq b'
    \end{equation}
  \item Komplizierter werden Normierung und Vollständigkeit
  \item Vollständigkeit: Analogie zur diskreten Basis $\{|a_i\rangle\}$
    \begin{equation}
      |\gamma\rangle =\sum_i|a_i\rangle \langle a_i|\gamma\rangle
    \end{equation}
    Übergang zu kontinuierlicher Basis $\{|b\rangle\}$\\ 
    $\hookrightarrow$ Übergang von Summe $\sum_i$ nach Integral $\int db$
    \begin{equation}
    |\gamma\rangle = \int db |b\rangle \underbrace{\langle b|\gamma\rangle}_{\text{kont. Funktion von b}}
    \label{eq:integral}
    \end{equation}
  \item Zerlegung der $\hat{1}$ in kontinuierlicher Basis
    \begin{equation}
      \hat{1} = \int db\ |b\rangle\langle b|
    \end{equation}
  \item Normierung: Multipliziere (\ref{eq:integral}) von links mit $\langle b'|$
    \begin{equation}
      \langle b'|\gamma\rangle =\int db\ \langle b'|b\rangle\langle b |\gamma\rangle
    \end{equation}
    vergleiche mit
    \begin{equation}
    \begin{aligned}
    &f(x') =\int dx\ \delta(x-x')f(x)\\
    &\Rightarrow \langle b'|b\rangle = \delta(b'-b)
    \end{aligned}
    \end{equation}
    Orthonormierungsbedingung für kontinuierliche Basen
  \item physikalische Interpretationen wird durch $\delta$-Distributionen schwieriger
  \item Spektraldarstellung für $\hat{B}$ mit konstanter Eigenbasis
    \begin{equation}
    \hat{b}=\langle b|b|b\rangle %seltsam
    \end{equation}
\end{itemize}


\subsection{Kommutatoren und simultane Eigenbasen}
\begin{itemize}
	\item physikalische Auswirkungen der Nicht-Vertauschbarkeit des Operatorprodukts formalisieren 
  \item Kommutation zweier Operatoren $\hat{A}$ und $\hat{B}$
    \begin{equation}
    \left[\hat{A},\hat{B}\right]=\hat{A}\hat{B}-\hat{B}\hat{A}
    \end{equation}
  \item Kommutierende Operatoren
    \begin{equation}
    \Leftrightarrow \left[\hat{A},\hat{B}\right]=0
    \end{equation}
  \item Rechenregeln
    \begin{equation}
    \begin{aligned}
      &\left[\hat{A},\hat{B}\right]=-\left[\hat{B},\hat{A}\right]\\
      &\left[\hat{A},\hat{B}\right]^\dagger=\left[\hat{B}^\dagger,\hat{A}^\dagger\right]\\
      &\left[\hat{A},\hat{B}+\hat{C}\right] =\left[\hat{A},\hat{B}\right]+\left[\hat{A},\hat{C}\right]\\
      &\left[\hat{A},\hat{B}\hat{C}\right]=\left[\hat{A},\hat{B}\right]\hat{C}+\hat{B}\left[\hat{A},\hat{C}\right]\\
      &\left[\hat{A},\left[\hat{B},\hat{C}\right]\right]+
        \left[\hat{C},\left[\hat{A},\hat{B}\right]\right]+
        \left[\hat{B},\left[\hat{C},\hat{A}\right]\right]=0
        \quad\text{(Jacobi-Identität)}
    \end{aligned}
    \end{equation}
  \item triviale Kommutation
    \begin{equation}
    \begin{aligned}
      &\left[\hat{A},\hat{1}\right]=0\\
      &\left[\hat{A},\hat{A}\right]=0
    \end{aligned}
    \end{equation}
\end{itemize}


\begin{flushright}
Vorlesung 6 - 02.05.13
\end{flushright}
\textbf{Simultane Eigenbasen}
  \begin{itemize}
    \item Betrachte zwei kommutierende hermetische Operatoren $\hat{A},\hat{B}$
      \begin{equation}
        \left[\hat{A},\hat{B}\right]=0
      \end{equation}
      Es existiert eine simultane Eigenbasis $\left\{|ab\rangle\right\}$
      \begin{equation}
      \begin{aligned}
      \hat{A}|ab\rangle&=a|ab\rangle\\
      \hat{B}|ab\rangle&=b|ab\rangle
      \end{aligned}
      \end{equation}
    \item Beweis: Angenommen $\left[\hat{A},\hat{B}\right]=0$ und betrachte 
      \begin{equation}
      \hat{A}|a\rangle=a|a\rangle
      \end{equation}
      Multipliziere mit $\hat{B}$
      \begin{equation}
      \begin{aligned}
      \hat{B}\hat{A}|a\rangle=&\hat{B}a|a\rangle\\
      \hat{A}\left(\hat{B}|a\rangle\right)&=a\left(\hat{B}|a\rangle\right)
      \end{aligned}
      \end{equation}
        $\Rightarrow$ $\left(\hat{B}|a\rangle\right)$ erfüllt noch immer EW-Relation für $\hat{A}$
    \item Für nicht entartetes a-Spektrum:\\
      $\left(\hat{B}|a\rangle\right)$ muss proportional zu $|a\rangle$ sein:
      \begin{equation}
      \Rightarrow \hat{B}|a\rangle=b|a\rangle
      \end{equation}
      Daher sind $|a\rangle$ simultane EV zu $\hat{A}$ und $\hat{B}$ $\rightarrow$ $|ab\rangle$
    \item Bei entartetem a-Spektrum:\\
      Entartungsindex i in $\hat{A}$ EW-Relation
      \begin{equation}
      \hat{A}|ai\rangle =a|ai\rangle
      \end{equation}
      $\hookrightarrow \left(\hat{B}|ai\rangle\right)$ ist weiterhin EV zu $\hat{A}$ aber nicht notwendig proportional zu $|ai\rangle$\\
      $\hookrightarrow$ Wähle orthonormierte Basis im entarteten Unterraum so, dass $\hat{B}|ai\rangle$ prop. zu $|ai\rangle$, d.h. wir lösen EW von $\hat{B}$ im ent. Unterraum
    \item Wenn keine weiteren Entartungen auftreten, dann ist simultane EB eindeutig charakterisiert durch 
      \begin{equation}
      \begin{aligned}
        \hat{A}|ab\rangle&=a|ab\rangle\\
        \hat{B}|ab\rangle&=b|ab\rangle
      \end{aligned}
      \end{equation}
      $\hookrightarrow$ Entartung vom $\hat{A}$-Spektrum durch Hinzunahme von $\hat{B}$ eliminiert oder gehoben.
    \item Vollständiger Satz von kommutierenden Operatoren:\\
      Für einen Satz von Operatoren $\left\{\hat{A},\hat{B},\hat{C},...\right\}$, die paarweise miteinander kommutieren
      \begin{equation}
      \left[\hat{A},\hat{B}\right]=\left[\hat{A},\hat{C}\right],...=0
      \end{equation}
      zu dem sich kein weiterer komm. Operator finden lässt, gibt es eine simultane Eigenbasis, 
      \begin{equation}
      \begin{aligned}
        \hat{A}|a,b,c...\rangle&=a|a,b,c...\rangle\\
        \hat{B}|a,b,c...\rangle&=b|a,b,c...\rangle
      \end{aligned}
      \end{equation}
      die keine Entartung aufweist.
  \end{itemize}
\textbf{Unschärferelationen}
  \begin{itemize}
    \item Für nicht kommutierende Operatoren $\hat{A},\hat{B}$ mit 
      $$
      \left[\hat{A},\hat{B}\right]\neq 0
      $$
      existiert keine simultane Eigenbasis
    \item Trotzdem wichtige Aussage zur Unschärfe möglich
    \item Unschärfe oder Varianz des EW von A mit $|\gamma\rangle$
      \begin{equation}
      \begin{aligned}
        \Delta A_\gamma&=\sqrt{\langle\gamma|\left(\hat{A}-\langle\gamma\left|\hat{A}\right|\gamma\rangle\right)^2|\gamma\rangle}\\
        &=\sqrt{\langle\gamma\left|\hat{A}^2\right|\gamma\rangle-\langle\left|\hat{A}\right|\gamma\rangle^2}\\
        &=\sqrt{\langle\hat{A}^2\rangle_\gamma-\langle\hat{A}\rangle_\gamma^2}
      \end{aligned}        
      \end{equation}
      mit $\hat{A}=\langle\gamma|\hat{A}|\gamma\rangle$
    \item Für das Produkt der Unschärfe $\Delta A_j$ und $\Delta B_j$ gilt verallgemeinerte Unschärferelation
      \begin{equation}
      \Delta A_\gamma \Delta B_\gamma \geq \frac{1}{2}\left|\langle\gamma\left|\left[ \hat{A},\hat{B}\right]\right|\gamma\rangle\right|
      \end{equation}
    \item Beweis der Schwarzschen Ungleichung
      \begin{equation}
      \langle\alpha|\alpha\rangle\langle\beta|\beta\rangle \geq \left|\langle\alpha\left|\right.\beta\rangle\right|^2
      \end{equation}
      Definition von 
      \begin{equation}
      \begin{aligned}
        |\alpha\rangle=&\left(\hat{A}-\langle\hat{A}_\gamma\right)|\gamma\rangle\\
        |\beta\rangle=&\left(\hat{B}-\langle\hat{B}_\gamma\right)|\gamma\rangle
      \end{aligned}
      \end{equation}
      Damit 
      \begin{equation}
      \Delta A_\gamma^2\Delta B_\gamma^2 \geq \left|\langle \left(\hat{A}-\langle\hat{A}\rangle_\gamma\right)\left(\hat{B}-\langle\hat{B}\rangle_\gamma\right)\rangle\right|^2
      \end{equation}
      Nutze Darstellung des Produkts zweier Operatoren über Kommutator $+$ Antikommutator
      \begin{equation}
      \begin{aligned}
        \hat{C}\hat{D}&=\frac{1}{2}\left[\hat{C},\hat{D}\right]+\frac{1}{2}\underbrace{\left\{\hat{C},\hat{D}\right\}}_{\text{Antikommutator}}\\
        &=\frac{1}{2i}\left(i\left[\hat{C},\hat{D}\right]\right)+\frac{1}{2}\left\{\hat{C},\hat{D}\right\} \quad \quad \text{; macht $\left[\hat{C},\hat{D}\right]$ hermitesch}
      \end{aligned}
      \end{equation}
      Mit $\hat{C}=\hat{A}-\langle\hat{A}\rangle_\gamma$, $\hat{D}=\hat{B}-\langle\hat{B}\rangle_\gamma$
      \begin{equation}
      \begin{aligned}
        &\left|\langle\left(\hat{A}\langle\hat{A}\rangle_\gamma\right)\left(\hat{B}\langle\hat{B}\rangle_\gamma\right)\rangle_\gamma\right|^2\\
        &=\left|\langle\hat{C},\hat{D}\rangle_\gamma\right|^2\\
        &=\left|\langle\frac{1}{2i}\left(i\left[\hat{C},\hat{D}\right]\right)+\frac{1}{2}\left\{\hat{C},\hat{D}\right\}\rangle_\gamma\right|^2\\
        &=\left|\frac{1}{2i}\langle i\left[\hat{C},\hat{D}\right]\rangle_\gamma+\frac{1}{2}\langle\left\{\hat{C},\hat{D}\right\}\rangle_\gamma\right|^2\\
        &=\frac{1}{4}\left|\langle\left[\hat{C},\hat{D}\right]\rangle_\gamma\right|^2+\frac{1}{4}\left|\langle\left\{\hat{C},\hat{D}\right\}\rangle_\gamma\right|^2
      \end{aligned}
      \end{equation}
    \item Antikommutatorbeitrag $(\geq 0)$ wird auf rechter Seite der Schwarzschen Ungleichung weggelassen
      \begin{equation}
      \begin{aligned}
        \Delta A_\gamma^2 \Delta B_\gamma^2 &\geq \frac{1}{4}\left|\langle\left[\hat{C},\hat{D}\right]\rangle_\gamma\right|^2\\
        &=\frac{1}{4}\left|\langle\left[\hat{A},\hat{B}\right]\rangle_\gamma\right|^2\\
        \Rightarrow \Delta A_\gamma \Delta B_\gamma &\geq \frac{1}{2}\left|\langle\left[\hat{A},\hat{B}\right]\rangle\right|
      \end{aligned}
      \end{equation}
  \end{itemize}

\section{Postulate der Quantenmechanik}
\subsection{Die Postulate}
 Mathematischer Formalismus mit physikalischer Größe koppeln
      \begin{itemize}
        \item Wie lässt sich der Zustand eines Quantensystems zu einem festen Zeitpunkt t beschreiben?
        \item Wie sind Observablen formal repräsentiert?
        \item Wie wird die Dynamik eines Quantensystems beschrieben?
      \end{itemize}
  \textbf{1. Postulat: Zustand eines Systems}\\
    Der Zustand eines Systems zum \textbf{Zeitpunkt t} ist durch einen \textbf{normierten} Vektor $|\psi,t\rangle$ im \textbf{Hilbertraum} beschrieben. Der Zustandsvektor $|\psi,t\rangle$ enthält die \textbf{komplette Information} über das System.
    
    \begin{flushright}
    Vorlesung 7 - 07.05.2013
    \end{flushright}
    
  \noindent\textbf{2. Postulat: Observablen}\\
    Jede physikalische Observable wird durch einen hermiteschen linearen Operator auf dem Hilbertraum beschrieben. 
    \begin{itemize}
      \item klare Trennung zwischen $\underbrace{\text{Zustand}}_{\text{ket-Vektoren}}$ des Systems und $\underbrace{\text{Observablen}}_{\text{Operatoren}}$
      \item Observablen, die ein klassisches Analogon besitzen (z.B. Ort, Impuls, Drehimpuls, kin. Energie, Gesamtenergie...) werden über Korrespondenzregeln definiert. \\
        Einer Observable, die durch eine Funktion $F(\vec{x},\vec{p})$ in der klassischen Physik beschrieben wird, wird ein hermitescher Operator zugeordnet, der sich aus Einsetzung ergibt. 
        $$
        \vec{x} \rightarrow \hat{\vec{x}} \quad, \quad \vec{p} \rightarrow \hat{\vec{p}}
        $$
        Zwischen $\hat{\vec{x}}$ und $\hat{\vec{p}}$ gilt die fundamentale Kommutatorrelation
        \begin{equation}
        \left[\hat{\vec{x}}_i, \hat{\vec{p}}_j\right]=i\hbar\delta_{ij}
        \end{equation}
      \item Manchmal führt Einsetzung auf einen nicht hermiteschen Operator, z. B. die Radialkomponente des Impulses
        \begin{equation}
        \begin{aligned}
          \text{klassisch:} \quad &\vec{x}\vec{p}\\
          \text{QM:} \quad &\frac{1}{2}\left(\hat{\vec{x}}\hat{\vec{p}}+\hat{\vec{p}}\hat{\vec{x}}\right)
        \end{aligned}
        \end{equation}
        $\rightarrow$ explizite "`Hermitesierung"' notwendig
      \item Bsp.: Energie der Observable:
        \begin{equation}
        \begin{aligned}
          \text{klassisch: Hamiltonfunktion} \quad H(\vec{x},\vec{p})&=\frac{\vec{p}}{2m} + V(\vec{x})\\
          \text{QM: Hamiltonoperator} \quad \hat{H}(\hat{\vec{x}},\hat{\vec{o}})&=\frac{\hat{\vec{p}}^2}{2m}+V(\hat{\vec{x}})
        \end{aligned}
        \end{equation}
      \item Analogie des fundamentalen Kommutators mit Poisson-Klammern der klassischen Mechanik
        \begin{equation}
         \left\{x_i,p_j\right\}=\delta_{ij}
        \end{equation}
        mit 
        \begin{equation}
        \left\{A,B\right\}:=\sum_j\left(\frac{\partial A}{\partial x_i}\frac{\partial B}{\partial p_i}-\frac{\partial A}{\partial p_i}\frac{\partial B}{\partial x_i}\right)
        \end{equation}
      \item Observablen ohne klassische Analogien (z.B. Spin) müssen im Rahmen der QM konstruiert werden
    \end{itemize}
  \textbf{3. Postulat: ideale Messung}\\
    Bei einer idealen Messung einer Observablen $\hat{A}$ an einem System im Zustand $|\psi,t\rangle$ kann nur einer der (diskreten) Eigenwerte des Operators $\hat{A}$ resultieren.\\
    Der Zustand des Systems unmittelbar nach der Messung ist durch den Eigenvektor zum gemessenen Eigenwert gegeben.
    \begin{itemize}
      \item Direkter Zusammenhang zwischen EW-Spektrum des Operators der Observable und dem wirklichen Messergebnis
        \begin{itemize}
          \item diskretes Spektrum: nur spezielle, diskrete Messergebnisse sind möglich. $\rightarrow$ Quantisierung
          \item kontinuierliche Spektren oder bei Entartung ein kleiner Unterraum, welcher sich aus Messunsicherheit bei kont. Observablen oder Entartung ergibt, ist nach der Messung als Information über den Zustand verfügbar
          \item Messung ist ein ideales Werkzeug zur Präparation von Zuständen
        \end{itemize}
    \end{itemize}
    Nachtrag: 
    \begin{equation}
      \hat{\vec{x}}=\left(\begin{array}{c} \hat{x}_1 \\ \hat{x}_2 \\ \hat{x}_3 \end{array}\right) \quad, \quad \hat{\vec{x}}\hat{\vec{p}}=\hat{x}_1\hat{p}_1 + \hat{x}_2\hat{p}_2 + \hat{x}_3\hat{p}_3 \quad, \quad \hat{\vec{x}}|\hat{x}\rangle=\vec{x}|\vec{x}\rangle
    \end{equation}
    \begin{equation}
    \left(\begin{array}{c} \hat{x}_1 \\ \hat{x}_2 \\ \hat{x}_3 \end{array}\right)|\vec{x}\rangle=\left(\begin{array}{c} x_1 \\ x_2 \\ x_3 \end{array}\right)|\vec{x}\rangle
    \end{equation}
    \begin{equation}
    \underbrace{\hat{x}_1}_{\text{Operator}}|x_1x_2x_3\rangle=\underbrace{x_1}_{\text{Eigenwert}}|x_1x_2x_3\rangle
    \end{equation}
  \textbf{4. Postulat: Wahrscheinlichkeitsaussage}\\
    Für ein Spektrum im Zustand $|\psi,t\rangle$ liefert das Skalarprodukt mit den Eigenzuständen der Observable eine Wahrscheinlichkeitsamplitude für die Messung der zugeordneten, zugehörigen Eigenwerte.
    \begin{equation}
      P_\psi(a_i)=\left|\langle a_i|\psi\rangle\right|^2 \quad \text{für System im Zustand} \quad |\psi\rangle
    \end{equation}

\begin{flushright}
Vorlesung 8 - 14.05.13
\end{flushright}
 
\subsection{Darstellung und Wellenfunktion}
Ortsdarstellung (OD)\\
kontinuierliche Eigenbasis des Ortsoperators $\hat{\vec{x}}$.
$$
\hat{\vec{x}}=\left(\begin{array}{c} \hat{x}_1 \\ \hat{x}_2 \\ \hat{x}_3 \end{array}\right)
$$
$$
\hat{\vec{x}}|\vec{x}\rangle=\hat{x}|\vec{x}\rangle
$$
\begin{equation}
\langle\vec{x}|\vec{x}\rangle=\hat{\delta}(\vec{x}-\vec{x})
\end{equation}
Zerlegung des Eins-Operators
$$
\hat{1}=\int\ dx \left|\vec{x}\rangle\langle\vec{x}\right|
$$
\begin{equation}
|\psi,t\rangle=\int\ dx\left|\vec{x}\rangle\langle\vec{x}\right|\psi,t\rangle
\label{eq:WellenfunktionOrtsdarstellung}
\end{equation}
Annahme: Teilchen mit Masse m in klassischem Potential $V(\vec{x})$
\begin{equation}
\xrightarrow{\text{2. Post.}} \quad \hat{H}=\frac{\hat{\vec{p}}^2}{2m}+V(\vec{x})
\label{eq:Hamiltonoperator}
\end{equation}
OD eines allg. Operators\footnote{Wichtiger Unterschied zwischen Operator im Hilbertraum und im 3D-Raum (Ortsdarstellung)} $\hat{A}$
$$
\langle\vec{x}\left|\right.\hat{A}|\alpha\rangle \quad \quad |\alpha\rangle \quad \text{beliebig}
$$
OD des Ortsoperators $\hat{\vec{x}}$
$$
\langle\vec{x}|\hat{\vec{x}}|\alpha\rangle=\vec{x}\langle\vec{x}|\alpha\rangle
$$
OD einer Funktion von $\vec{x}$
$$
\langle\vec{x}|V(\vec{x})|\alpha\rangle=V(\vec{x})\langle\vec{x}|\alpha\rangle
$$
OD Impulsoperator\\
keine einfache EW Relation, aber wir wissen
\begin{align}
[\hat{x}_i,\hat{p}_j]&=i\hbar\delta_{ij} \nonumber\\ \nonumber
\langle\vec{x}|\hat{\vec{p}}|\alpha\rangle&=-i\hbar\vec{\nabla}\langle\vec{x}|\alpha\rangle\\ \nonumber
\langle\vec{x}|\hat{p}_i|\alpha\rangle&=i\hbar\frac{\partial}{\partial_i}\langle\vec{x}|\alpha\rangle\\ \nonumber
\langle\vec{x}|[\hat{x}_i,\hat{p}_j]|\alpha\rangle&=\langle\vec{x}|\hat{x}_i,\hat{p}_j|\alpha\rangle-\langle\vec{x}|\hat{x}_j,\hat{p}_i|\alpha\rangle\\ 
&=x_i\langle\vec{x}_i,\hat{p}_j|\alpha\rangle+i\hbar\frac{\partial}{\partial x_i}x_i\langle\hat{x}|\alpha\rangle\nonumber\\
&=i\hbar\delta_{ij}\langle\vec{x}|\alpha\rangle
\end{align}
OD des $\hat{H}$-Operators 
\begin{equation}
\langle\vec{x}|\hat{H}|\alpha\rangle=\left[\frac{-\hbar^2}{2m}\vec{\nabla}^2+V(\vec{x})\right]\langle\vec{x}|\alpha\rangle
\label{eq:Hamiltonoperator Ortsdarstellung}
\end{equation}
Eigenwertproblem von $\hat{H}$
\begin{align}
&\hat{H}|E_i\rangle=E_i|E_i\rangle\nonumber\\
&\xrightarrow{OD}\left(\frac{-\hbar^2}{2m}\cdot \Delta+V(\vec{x})\right)\langle\vec{x}|\alpha\rangle=E_i\langle\vec{x}_i|E_i\rangle
\end{align}
$\hookrightarrow$ Wellenfunktion des Energieeigenzustandes in OD\\
$\hookrightarrow$ stationäre Schrödingergleichung in OD\\\\
\textbf{5. Postulat}: Allgemeine Schrödingergleichung
\begin{equation}
i\hbar\frac{\partial}{\partial t}|\psi,t\rangle=\hat{H}|\psi,t\rangle
\label{eq:Allgemeine Schrödingergleichung}
\end{equation}
Allgemeine Schrödingergleichung in OD
\begin{equation}
i\hbar\frac{\partial}{\partial t}\langle\vec{x}|\psi,t\rangle=\left(-\frac{\hbar^2}{2m} \Delta+V(\vec{x})\right)\langle\vec{x}|\psi,t\rangle
\label{eq:Allgemeine Schrödingergleichung in OD}
\end{equation}
$\hookrightarrow$ einfache DGL, stark vereinfacht\\
abstrakte Schrödingergleichung ist viel allgemeiner als diese DGL (1 Teilchen ohne innere Freiheitsgrade im klassischen Potential $V(\vec{x})$.\\\\
\textbf{Impulsdarstellung}\\
kont. Eigenbasis des Impulsoperators $\hat{\vec{p}}$
\begin{align}
&\hat{\vec{p}}|\vec{p}\rangle=\vec{p}|\vec{p}\rangle\\
\text{mit} \quad &\hat{1}=\int \mathrm{d}^3p\ |\vec{p}\rangle\langle\vec{p}|\\
&\langle\vec{p}|\vec{p}'\rangle=\delta^3(\vec{p}-\vec{p}')
\end{align}
Wellenfunktion in ID
\begin{align}
&\tilde{\psi}(\vec{p},t)=\langle\vec{p}|\psi,t\rangle\\
&\left|\tilde{\psi}(\vec{p},t)\right|^2\ : \quad \text{Wahrscheinlichkeitsdichte für den Impuls $\vec{p}$}
\end{align}
Orts- und Impulsraumwellenfunktionen enthalten gleiche Information, lediglich in Unterschiedlichen Darstellungen von $|\psi,t\rangle$\\
Durch Einsetzen eines Eins-Operators können wir zwischen Darstellungen wechseln:
\begin{equation}
\psi(\vec{x},t)=\langle\vec{x}|\psi,t)=\int d^3p\langle\vec{x}|\vec{p}\rangle\underbrace{\langle\vec{p}|\psi,t\rangle}_{\psi(\vec{p},t)} \quad (*)
\end{equation}
\begin{align}
&\phi_{\vec{p}}(\vec{x})=\langle\vec{x}|\vec{p}\rangle\quad \text{OD des Impulszustandes}\\
&\langle\vec{x}|\hat{\vec{p}}|\vec{p}\rangle=\vec{p}\langle\vec{x}|\vec{p}\rangle\Leftrightarrow -i\hbar\vec{\nabla}\langle\vec{x}|\vec{p}\rangle=\vec{p}\langle\vec{x}|\vec{p}\rangle\quad\text{einfache DGL}\\
&\phi_{\vec{p}}(\vec{x})=N_p\exp\left(\frac{i}{\hbar}\vec{x}\vec{p}\right), \quad N_p=\frac{1}{(2\pi\hbar)^{3/2}}\\
&\delta^3(\vec{p}'-\vec{p})=\left(\frac{1}{2\pi\hbar}\right)^3\int \mathrm{d}^3x\ \exp\left(\frac{i}{\hbar}(\vec{p}'-\vec{p})\vec{x}\right)
\end{align}
$\phi_{\vec{p}}(\vec{x})$ ist eine komplexe ebene Welle mit konstantem Betragsquadrat
\begin{equation}
\left|\phi_{\vec{p}}(\vec{x})\right|^2=\left(\frac{1}{2\pi\hbar}\right)^3
\end{equation}
Einsetzen in ($*$):
\begin{equation}
\psi(\vec{x},t)=\frac{1}{(2\pi\hbar)^{3/2}}\int\mathrm{d}^3p\ \exp\left(\frac{i}{\hbar}\vec{x}\vec{p}\right)\tilde{\psi}(\vec{x},t)\quad \text{Fouriertransformation}
\end{equation}
Umkehrtransformation
\begin{equation}
\tilde{\psi}(\vec{p},t)=\frac{1}{(2\pi\hbar)^{3/2}}\int\mathrm{d}^3x\ \exp\left(-\frac{i}{\hbar}\vec{x}\vec{p}\right)\psi(\vec{x},t)
\end{equation}
\textbf{ID von Impuls und Ortsoperator}
\begin{equation}
\langle\vec{p}|\hat{\vec{p}}|\alpha\rangle=\vec{p}\langle\vec{p}|\alpha\rangle\quad \langle\vec{p}|\hat{\vec{x}}|\alpha\rangle=i\hbar\vec{\nabla}_p\langle\vec{p}|\alpha\rangle \quad \text{mit}\quad \vec{\nabla}_p=\left(\begin{array}{c} \partial/\partial p_1 \\ \partial/\partial p_2 \\ \partial/\partial p_3\end{array}\right)
\end{equation}
\begin{equation}
\hat{H}=\frac{1}{2m}\hat{\vec{p}}^2+V(\vec{x})\xrightarrow{ID} \langle\vec{p}|\hat{H}|\alpha\rangle=\left(\frac{\vec{p}^2}{2m}+V(i\hbar\vec{\nabla}_p)\right)\langle\vec{p}|\alpha\rangle
\end{equation}
Hieraus lässt sich wiederum eine DGL für das EW-Problem von $\hat{H}$ herleiten. Lösung solcher DGL in ID ist meist nicht möglich, da i.A. das Potential $V(\vec{x})$ eine komplexe Funktion des Ortes ist, die in ID zu einem komplexen Differentialoperator wird.

\begin{flushright}
Vorlesung 9 - 16.05.2013
\end{flushright}

\subsection{Zeitentwicklung und stationäre Zustände}
Zeitentwicklung von $|\psi,t\rangle$ durch SGL 
\begin{equation}
i\hbar \frac{\partial}{\partial t}|\psi,t\rangle=\hat{H}|\psi,t\rangle
\label{eq:Schrödingergleichung}
\end{equation}
Diese lineare DGL erster Ordnung lässt sich formal für gegebenen Anfangszustand $|\psi,t_0\rangle$ durch Anwendung eines linearen Zeitentwicklungsoperators $\hat{U}_{(t,t_0)}$ lösen.
\begin{equation}
|\psi,t\rangle=\hat{U}_{(t,t_0)}|\psi,t_0\rangle
\end{equation}
Zeitentwicklungsoperator
\begin{align}
i\hbar\frac{\partial}{\partial t}\hat{U}_{(t,t_0)}|\psi,t_0\rangle&=\hat{H}\hat{U}_{(t,t_0)}|\psi,t_0\rangle \quad \text{mit} \quad |\psi,t_0\rangle \quad \text{beliebig}\\
\rightarrow i\hbar \frac{\partial}{\partial t}\hat{U}_{(t,t_0)}&=\hat{H}\hat{U}_{(t,t_0)}
\end{align}
\subsubsection{Zeitunabhängiger Hamiltonoperator}
\begin{equation}
\hat{U}_{(t,t_0)}=\hat{U}_{(t,-t_0)}=\exp\left(-\frac{i}{\hbar}\hat{H}(t-t_0)\right)
\end{equation}
mit $\hat{U}_{(t_0,t_0)}=\hat{1}$ wenn $\hat{H}^\dagger=\hat{H}$\\
$\rightarrow$ Norm eines Zustandes bleibt während der Zeitentwicklung erhalten\\
Exponentialfunktion mit einem hermiteschen Operator im Exponenten ist typisch für kontinuierliche Transformationen (Translation, Rotation). In diesem Zusammenhang bezeichnet man $\hat{H}$ auch als Generator der zeitlichen Translation.
\subsubsection{Zeitabhängiger Hamiltonoperator}
Nur wenn gilt 
\begin{equation}
\left[\hat{H}_{(t)},\hat{H}_{(t')}\right]=0
\label{eq:Kommutatorrelation2}
\end{equation}
lässt sich die DGL geschlossen lösen.
\begin{equation}
\hat{U}_{(t,t_0)}=\exp\left(-\frac{i}{\hbar}\int_{t_0}^t \ \text{d}t'\hat{H}_{(t')}\right)
\end{equation}
Für (\ref{eq:Kommutatorrelation2}) $\neq$ 0 kann DGL nicht durch Exponentialfunktion gelöst werden.\\
Hier bleibt nur noch iterative Lösung in Form der Dyson-Reihe.
\begin{align}
&i\hbar\frac{\partial}{\partial t} \hat{U}_{(t,t_0)}=\hat{H}_{(t)}\hat{U}_{(t,t_0)}\\
&\hat{U}_{(t,t_0)}=\hat{1}-\frac{i}{\hbar} \ \int^t_{t_0}\ \text{d}t_1\hat{H}_{t_1}\hat{U}_{(t_1,t_0)}
\end{align}
Iteratives Einsetzen von $\hat{U}_{(t,t_0)}$ auf der rechten Seite führt zu Dyson-Reihe.
\subsubsection{Stationäre Zustände}
Der Zeitentwicklungsoperator für zeitunabhängigen $\hat{H}$ und $t_0=0$
\begin{equation}
\hat{U}_{(t)}=\exp\left(-\frac{i}{\hbar}\hat{H}t\right)
\end{equation}
stellt formal Lösung der SGL der Anwendung von $\hat{U}_{(t)}$ dar, ist aber meist ebenso kompliziert wie das direkte Lösen der SGL. Ausnahme bilden Eigenzustände von $\hat{H}$.
\begin{align}
\hat{H}|E_n\rangle&=E_n|E_n\rangle\\
\rightarrow |E_n,t\rangle&=\hat{U}_{(t)}|E_n\rangle=\exp\left(-\frac{i}{\hbar}\hat{H}t\right)|E_n\rangle\\
&=\exp\left(-\frac{i}{\hbar}E_nt\right)|E_n\rangle
\end{align}
Zeitabhängigkeit des Zustandes beschränkt sich auf einen komplexen Phasenfaktor. \\
Frequenz mit der diese Phase des Zustandes rotiert ist
\begin{equation}
w_n=\frac{E_n}{\hbar}
\end{equation}
Bei der Berechnung von beobachtbaren Größen (z.B. Betragsquadrate von Skalarprodukten  oder Erwartungswerten) fällt dieser Phasenfaktor heraus.
\begin{align}
\langle E,t|\hat{A}|E,t\rangle&=\langle E_n |\exp\left(+\frac{i}{\hbar}E_nt\right)\hat{A}\exp\left(-\frac{i}{\hbar}E_nt\right)|E_n\rangle\\
&=\langle E_n|\hat{A}|E_n\rangle
\end{align}
Daher werden EZ von $\hat{H}$ auch als stationäre Zustände bezeichnet, und die entsprechende EW-Gleichung als stationäre Schrödingergleichung.\\
Zeitentwicklung eines beliebigen Zustandes $|\psi,0\rangle$\\
$|E_n\rangle$ bilden eine vollständige orthonormierte Basis
\begin{align}
|\psi,0\rangle&=\sum_n|E_n\rangle\langle E_n|\psi,0\rangle\\
\rightarrow |\psi,t\rangle&=\sum_n\hat{U}_{(t)}|E_n\rangle\langle E_n|\psi,0\rangle\\
&=\sum_n\exp\left(-\frac{i}{\hbar}E_nt\right)\langle E_n |\psi,0\rangle\\
|\psi,t\rangle&=\sum_nc_n(t)|E_n\rangle \quad \text{mit} \\
&c_n(t)=\exp\left(-\frac{i}{\hbar}E_nt\right)\cdot c_n(0)\\
&c_n(0)=\langle E_n|\psi,0\rangle
\end{align}
\subsubsection{Schrödingerbild}
In bisheriger Sichtweise tragen Zustände die Zeitabhängigkeit, formal durch Zeitentwicklungsoperator beschrieben.\\
Zeitabhängigkeit von Observablen, z.B. Erwartungswerte
\begin{equation}
\langle\hat{A}\rangle_{(t)}=\langle\psi,t|\hat{A}|\psi,t\rangle
\end{equation}
resultiert aus Zeitabh- der Zustände, Operatoren sind zeitunabhängig
\subsubsection{Heisenbergbild}
Alternative Interpretation der Zeitentwicklung des Erwartungswertes 
\begin{align}
\langle\hat{A}\rangle_{(t)}&=\langle\psi,t|\hat{A}|\psi,t\rangle\\
&=\langle\psi,0|\hat{U}^\dagger_{(t)}\hat{A}\ \hat{U}_{(t)}|\psi,0\rangle\\
&=\langle\psi,0|\hat{A}_{\hat{H}(t)}|\psi,0\rangle
\end{align}
$\rightarrow$ unitäre Transformation des Operators\\
OP der Observablen entwickeln sich in der Zeit, während die Zustände selbst zeitunabhängig sind. Schrödinger- und Heisenbergbilder sind Darstellungen derselben Physik.
\begin{align}
&\hat{A}_{H(t)}=\hat{U}^\dagger_{(t)}\hat{A}\hat{U}_{(t)}\\
&\frac{\text{d}}{\text{d}t}\hat{A}_{H(t)}=\frac{i}{\hbar}\left[\hat{H},\hat{A}_{H(t)}\right]+\frac{\partial \hat{A}}{\partial t}H(t)
\end{align}

\begin{flushright}
Vorlesung 10 - 21.05.2013
\end{flushright}

\subsection{Observablen und Vorhersagen}
\begin{itemize}
	\item 4. Postulat: Für ideale Messung können Wahrscheinlichkeiten für einen der Eigenwerte der Observablen $\hat{A}$ vorhergesagt werden (diskretes Spektrum).
    \begin{equation}
      P_\psi(a_i)=\left|\langle a_i|\psi\rangle\right|^2 \quad \text{für System im Zustand} \quad |\psi\rangle
    \end{equation}
  \item Experimenteller Zugang: Große Zahl von Messungen der Observablen an Systemen in \textbf{identischen} Zuständen $|\psi\rangle$\\
    $\rightarrow$ Experimentelle Häufigkeitsverteilung\\
    $\rightarrow$ Vergleichbar mit Wahrscheinlichkeitsverteilung $P_\psi(a_i)$ aus der QM-Beschreibung
  \item Bsp.: Doppelspaltexperiment mit Elektronen\\
    Häufigkeitsverteilung der Position der Elektronen auf dem Schirm ist Abb. des Betragsquadrats der Wellenfunktion in Ortsdarstellung
  \item Oft ist nicht komplette Verteilung von Interesse, sondern nur einfache Kenngrößen: Mittelwerte \& Varianz
  \item Mittelwert der Verteilung $P_\psi(a_i)$
    \begin{equation}
    \langle\hat{A}\rangle_\psi=\sum_iP_\psi(a_i) a_i
    \end{equation}
    Im QM-Kontext
    \begin{align}
    \langle\hat{A}\rangle_\psi&=\sum_i\left|\langle a_i|\psi\rangle\right|^2a_i\\
    &=\sum_i\langle\psi|a_i\rangle a_i\langle a_i|\psi\rangle\\
    &=\langle\psi|\hat{A}|\psi\rangle
    \end{align}
    Erwartungswert der Observablen $\hat{A}$ $\Leftrightarrow$ Mittelwert der Wahrscheinlichkeitstheorie 
  \item Analog: Varianz der Verteilung $P_\psi(a_i)$ $\Leftrightarrow$ Unschärfe $\Delta A_\psi$
  \item Spezialfall: $|\psi\rangle=|a_i\rangle$\\
    Erwartungswert: $\langle\psi|\hat{A}|\psi\rangle=\langle a_i|\hat{A}|a_i\rangle=\langle a_i|a_i|a_i\rangle=a_i$\\
    Unschärfe: $\Delta A_{a_i}=0$
\end{itemize}
\textbf{Messung kompatibler Observablen}
\begin{itemize}
	\item Betrachte sequentielle Messung zweier Observablen. Beide Messungen unmittelbar nacheinander, sodass Zeitentwicklung zwischen beiden Messungen irrelevant ist.
  \item Angenommen $\left[\hat{A},\hat{B}\right]=0$\\
    Es existiert simultane Eigenbasis
    \begin{align}
      \hat{A}|a_ib_j\rangle&=a_i|a_ib_j\rangle\\
      \hat{B}|a_ib_j\rangle&=b_j|a_ib_j\rangle
    \end{align}
    Im Falle von diskrete Spektren und keiner Entartung.
  \item Ausgangszustand $|\psi\rangle$ entwickelt in simultaner Eigenbasis
    \begin{equation}
    |\psi\rangle=\sum_{ij}C_{ij}|a_ib_j\rangle
    \end{equation}
  \item 1. Messung von $\hat{A}$. Wahrscheinlichkeitsverteilung $P_\psi(a_i)$
    \begin{align}
      P_\psi(a_i)&=\sum_j\left|\langle a_ib_j|\psi\rangle\right|^2\\
      &=\sum_j\left|\langle a_ib_j|\left(\sum_{i'j'}C_{i'j'}|a_{i'}b_{j'}\rangle\right)\right|^2\\
      &=\sum_j\left|\sum_{i'j'}C_{i'j'}\langle\underbrace{a_ib_j|a_{i'}b_{j'}}_{\delta_{ii'}\delta_{jj'}}\rangle\right|^2\\
      &=\sum_j\left|C_{ij}\right|^2
    \end{align}
  \item 1. Messung: Zustand nach $a_i$ Messung konstruieren\\
    $\rightarrow$ Projektion auf Unterraum zum Eigenwert $a_i$.
  \item Projektionsoperator $\hat{\Pi}(a_i)$
    \begin{equation}
      \hat{\Pi}(a_i)=\sum_j|a_ib_j\rangle\langle a_ib_j|
    \end{equation}
    Verbindung zum $\hat{1}$-Operator 
    \begin{equation}
    \hat{1}=\sum_i\hat{\Pi}(a_i)
    \end{equation}
  \item Eigenschaften von Projektionsoperator 
    \begin{align}
      \hat{\Pi}^\dagger&=\hat{\Pi} \quad \text{hermitesch}\\
      \hat{\Pi}^2&=\hat{\Pi} \quad \text{itempotenz}
    \end{align}
  \item Zustand nach $a_i$-Messung
    \begin{align}
      |\psi'(a_i)\rangle&=\hat{\Pi}(a_i)|\psi\rangle\\
      &=\sum_jC_{ij}|a_ib_j\rangle
    \end{align}
    keine Summation über i
  \item Nachnormierung 
    \begin{equation}
    |\psi(a_i)\rangle=\frac{1}{\left(\sum_j|C_{ij}|^2\right)^{1/2}}\sum_jC_{ij}|a_ib_j\rangle
    \end{equation}
  \item 2. Messung: Wsk. für Messergebnis $b_j$
    \begin{align}
      P_{\psi'(a_i)}(b_j)&=\sum_i\left|\langle a_{i'}b_j|\psi'(a_i)\rangle\right|^2\\
      &=\sum_{i'}\left|\frac{1}{\left(\sum_j |C_{ij}|^2\right)^{1/2}}\sum_{j'}C_{ij'}\underbrace{a_{i'}b_j|a_ib_{j'}\rangle}_{\delta_{ii'}\delta_{jj'}}\right|^2\\
      &=\frac{|C_{ij}|^2}{\sum_{j'}|C_{ij'}|^2}
    \end{align}
  \item Kombinierte Wahrscheinlichkeit für Messung von $a_i$ und $b_j$
    \begin{equation}
    P_\psi(a_i,b_j)=P_\psi(a_i)P_{\psi(a_i)}(b_j)=|C_{ij}|^2
    \end{equation}
  \item 2. Messung: Zustand
    \begin{equation}
    |\psi''(a_ib_j)\rangle=|a_ib_j\rangle
    \end{equation}
  \item Sequentielle Messung weiterer Observablen $\hat{A},\hat{B}$ mit $[\hat{A},\hat{B}]=0$ gilt:
    \begin{itemize}
      \item Wahrscheinlichkeit für bestimmte Kombination von Messergebnissen ($a_i,b_j$)
        \begin{equation}
        P_\psi(a_ib_j)=|C_{ij}|^2=\left|\langle a_ib_j|\psi\rangle\right|^2
        \end{equation}
      \item Zustand nach beiden Messungen
        \begin{equation}
        |\psi''(a_ib_j)\rangle=|a_ib_j\rangle
        \end{equation}
        ist der simultane EZ zu beiden Messergebnissen
      \item Wahrscheinlichkeiten von 2. Zustand sind unabhängig von Reihenfolge der Messungen
      \item Wiederholte Messungen von $\hat{A}$ oder $\hat{B}$ in beliebiger Reihenfolge reproduzieren die ersten Messergebnisse $a_ib_j$.
    \end{itemize}
\end{itemize}

\begin{flushright}
Vorlesung 11 - 23.05.2013
\end{flushright}
\section{Einfache eindimensionale Probleme}
\begin{itemize}
  \item Betrachte Teilchen mit Masse m in einer Dimension im Potential V(x)
  \item Suche Lösungen des EW-Problems des Hamiltonoperators
    \begin{equation}
      \hat{H}=\frac{1}{2m}\hat{p} + V(\hat{x}) \quad \quad \hat{H}|E_n\rangle = E_n|E_n\rangle
    \end{equation}
   \item Wähle OD um das EW-Problem zu lösen [dadurch wird $V(\hat{x})$ einfacher]
     \begin{itemize}
       \item Eigenzustände werden über Wsk.fkt. in OD:
          \begin{equation}
            \psi_{E_n}(x) := \langle x|E_n \rangle
          \end{equation}
       \item Eigenwertproblem von $\hat{H}$ definiert die stationäre SGL in OD:
          \begin{equation}
            \left(-\frac{\hbar}{2m} \frac{\partial^2}{\partial x^2} + V(x) \right)\psi_{E_n}(x) = E_n \psi_{E_n}(x)
          \end{equation}
          $\rightarrow$ homogene lineare DGL 2. Ordnung
     \end{itemize}
   \item Wichtige Nebenbedingung: physikalisch sinnvolle Zustände bzw. Wellenfkt. müssen normiert/normierbar sein:
      \begin{itemize}
       \item für Zustandskets: $\langle \psi | \psi \rangle = 1 $
       \item für Wellenfkt.: $ 1= \langle \psi | \psi \rangle = \int \langle \psi | x \rangle \langle x | \psi \rangle \mathrm{d}x = \int \psi^*(x)\psi(x) \mathrm{d}x = \int |\psi(x)|^2 \mathrm{d}x $
       \item Um normierbar zu sein muss die Wellenfkt. quadratintegrabel und das Integral endlich sein.
      \end{itemize}
   \item Unterscheide zwei Klassen von Lösungen der stationären SGL:
      \begin{enumerate}
       \item gebundene Zustände
          \begin{itemize}
           \item räumlich in endlichem Gebiet lokalisiert, d.h. $ \psi(x) \xrightarrow{|x| \rightarrow \infty} 0 $\\
           $ \hookrightarrow $ Wellenfkt. ist normierbar
           \item Normierbarkeit führt auf diskretes Spektrum
          \end{itemize}
       \item ungebundene Zustände
          \begin{itemize}
           \item Wellenfkt. verschwindet nicht für $ |x| \rightarrow \infty $\\
           $ \hookrightarrow $ Wellenfkt. ist \textbf{nicht} normierbar und damit kein physikalisch erlaubter Zustand
          \end{itemize}
      \end{enumerate}
\end{itemize}


      
\subsection{Freies Teilchen}
\begin{itemize}
 \item Lsg. d. stationären SGL für $ V(x) = 0 $
    \begin{equation}
        -\frac{\hbar^2}{2m} \frac{\partial^2}{\partial x^2} \phi_E(x) = E\phi_E(x)
    \end{equation}
  \item absorbiere Vorfaktor der kin. Energie in einer Umdefinition von E:
    \begin{equation}
      E := \frac{\hbar^2}{2m}k^2 \quad \quad k: \text{Wellenzahl}
    \end{equation}
    \begin{equation}
      \hookrightarrow -\frac{\partial^2}{\partial x^2} \phi_E(x) = k^2 \phi_E(x)
    \end{equation}
    $ \hookrightarrow $ Analogie zu Schwingungs-DGL aus klass. Mechanik\\
    $ \hookrightarrow $ allg. Lsg.:
    \begin{equation}
     \phi_E(x) = A e^{ikx} + B e^{-ikx} = A'\sin{kx} + B'\cos{kx}
    \end{equation}
  \item Alternativ: EW-Problem von $ \hat{p} $ ist schon bekannt. Lsg. f. EW-Problem von $ \hat{p} $:
    \begin{equation}
     \hat{p} |p\rangle = p |p \rangle
    \end{equation}
    \begin{equation}
    % TODO
      \langle x | p \rangle \propto e^{\frac{i}{\hbar}px} = e^{ikx}
    \end{equation}
  \item EW-Problem von $ \hat{p}^2 $ und $ \hat{p}^2 | p \rangle = p^2 | p \rangle \rightarrow $ zweifache Entartung des $ p^2 $ Spektrums
      $ \hookrightarrow $ allg. Form für Eigenzustände zu $ \hat{p}^2 $ : $ A |p \rangle + B|-p \rangle $
  \item kontinuierliches Spektrum/ungebundene Zustände:
    \begin{itemize}
     \item Wellenfkt. $ \phi_E(x) $ ist nicht lokalisiert und nicht normierbar
     \item $ \hookrightarrow \phi_E(x) $ können nicht als physikalisch sinnvolle Systemzustände interpretiert werden
     \item trotzdem nützliche Basis für Beschreibung eines freien Teilchens
    \end{itemize}
\end{itemize}


\subsection{Wellenpakete}
\begin{itemize}
  \item physikalische Perspektive: ``freie Teilchen'' sind typischerweise in endlichem räumlichen Gebiet lokalisiert
  \item Betrachte Superposition von ebene-Welle-Lösungen $ \phi_E(x) $ mit Koeffizienten die durch die Fkt. $C(E) = C(k^2) $ gegeben:
    \begin{equation}
       \phi(x) = \int_{0}^{\infty} C(E)e^{ikx} + C'(E)e^{-ikx} \, \mathrm{d}E = \int_{-\infty}^{\infty} C(k)e^{ikx} \, \mathrm{d}k
    \end{equation}
    $ \hookrightarrow $ Fouriertransformation: Amplitudenfkt. C(k) ist gerade die Wellenfkt. in ID, bezeichnet mit $ \tilde{\phi}(k) $
  \item Normierbarkeit: Wellenfkt. in ID muss quadratintegrabel \& endlich sein:
    \begin{equation}
      \int_{-\infty}^{\infty} |\tilde{\psi}(k)|^2 \, \mathrm{d}k = \int_{-\infty}^{\infty} |C(k)|^2 \, \mathrm{d}k
    \end{equation}
    $ \hookrightarrow $ lässt sich gewährleisten, wenn C(k) im Unendlichen (in k) verschwindet, d.h. lokalisiert in Wellenzahl
  \item Prototyp: Gaußsches Wellenpaket:
    \begin{equation}
      C(k) = \left(\frac{2}{\pi}a^2 \right)^{1/4} e^{-a^2k^2}
    \end{equation}
    $ \hookrightarrow $ Fouriertransformation also Wellenfkt. in OD ist ebenfalls gaußförmig
    \begin{equation}
      \psi(x) = \left(\frac{1}{2\pi a^2} \right)^{1/4} e^{-\frac{x^2}{4a^2}}
    \end{equation}




\end{itemize}




\begin{flushright}
Vorlesung 12 - 28.05.2013
\end{flushright}
\subsection{Stückweise konstante Potentiale}
\begin{itemize}
	\item stückweise konstante Potentiale $\rightarrow$ V(x)
    \begin{equation}
    V(x)=\left\{
    \begin{aligned}
    &\stackrel{:}{.}\\
    &V_{12} \quad \text{für} \quad \xi_1<x<\xi_2\\
    &V_{23} \quad \text{für} \quad \xi_2<x<\xi_3\\
    &\stackrel{:}{.}
    \end{aligned}
    \right.
    \end{equation}
  \item stationäre Schrödingergleichung
    \begin{align}
      &\left(-\frac{\hbar^2}{2m}\frac{\partial^2}{\partial x^2}+V(x)\right)\psi_E(x)=E\psi_E(x)\\
      &-\frac{\hbar^2}{2m}\frac{\partial^2}{\partial x^2}\psi_E(x)=\left[E-V(x)\right]\psi_E(x)\\
      &\frac{\partial^2}{\partial x^2}\psi_E(x)=\frac{2m}{\hbar^2}\left[V(x)-E\right]\psi_E(x)
    \end{align}
  \item Anschaulich ist die zweite Ableitung der Wellenfunktion (Krümmung) mit Differenz $(V(x)-E)$ verknüpft.
  \item $E>V(x)$: Zweite Ableitung $\psi_E''(x)$ hat umgekehrtes Vorzeichen wie $\psi_E(x)$ $\rightarrow$ Krümmung zur Achse.
  \item $E<V(x)$: $\psi_E''(x)$ hat das selbe Vorzeichen wie $\psi_E(x)$ $\rightarrow$ Krümmung von Achse weg.
  \item stückweise konst. Potential in einem der Intervalle mit $V(x)=V_0$
    \begin{equation}
    \frac{\partial^2}{\partial x^2}\psi_E(x)=\frac{2m}{\hbar^2}[V_0-E]\psi_E(x)
    \end{equation}
    Allgemeine Lösung: ebene Welle 
    \begin{equation}
    \psi_E(x)=C\exp\left(\pm\sqrt{\frac{2m}{\hbar^2}[V_0-E]}x\right)
    \end{equation}
  \item Unterscheide $E>V_0, \ E<V_0$\\
    1. $E>V_0$
    \begin{align}
    \psi_E(x)&=Ae^{ikx}+Be^{-ikx}\\
    &=A'\sin(kx)+B'\cos(kx)
    \end{align}
    \begin{equation}
    k=\sqrt{\frac{2m}{\hbar^2}(E-V_0)} \in \mathbb{R}
    \end{equation}
    $\hookrightarrow$ Oszillatorische Lösung\\
    2. $E>V_0$
    \begin{align}
    \psi_E(x)&=Ae^{K x}+Be^{-K x}\\
    &K=\sqrt{\frac{2m}{\hbar^2}(V_0-E)}\in \mathbb{R}
    \end{align}
    $\hookrightarrow$ Exponentieller Anstieg/Abfall
  \item Vorsicht in der Umgebung der Sprungstelle/Unstetigkeiten im Potential $\rightarrow$ Stetigkeitsbedingung
  \item Betrachte $\epsilon$-Umgebung um eine Sprungstelle und integriere (formal) die SGL.
    \begin{align}
    &\int_{\xi-\epsilon}^{\xi+\epsilon}\ dx \ \frac{\partial^2\psi_E(x)}{\partial x^2}=\int_{\xi-\epsilon}^{\xi+\epsilon}\ dx \ \frac{2m}{\hbar^2}(V(x)-E)\psi_E(x)\\
    &\psi_E'(\xi+\epsilon)-\psi_E'(\xi-\epsilon)=\frac{2m}{\hbar^2} \int_{\xi-\epsilon}^{\xi+\epsilon}\ dx \ (V(x)-E) \psi_E(x)
    \end{align}
  \item Aus $\epsilon\rightarrow 0$ Betrachtung:\\
    $\frac{\partial}{\partial x}\psi_E(x)$ und $\psi_E(x)$ sind an der Sprungstelle stetig, wenn V(x) endlich.
\end{itemize}
\subsection{Potentialtopf}
\label{topf}
\begin{itemize}
  \item Potentialtopf
    \begin{equation}
    V(x)=\left\{
    \begin{aligned}
    &V_0 \ : \ x\leq -a/2 \quad (I)\\
    &0 \ : \ -a/2\leq x\leq a/2 \quad (II) \\
    &V_0 \ : \ x\geq a/2 \quad (III)
    \end{aligned}
    \right.
    \end{equation}
    Zwei Klassen von Lösungen
    \begin{itemize}
      \item $E\leq V_0$ : gebundene Zustände
      \item $E>V_0$ : ungebundene Zustände
    \end{itemize}
    \end{itemize}
    \textbf{Bindungszustand:} $0\leq E \leq V_0$
    \begin{itemize}
      \item Drei Bereiche:
        \begin{align}
          (I) \quad E<V_0: \quad &K=\sqrt{\frac{2m}{\hbar^2}(V_0-E)}\\
          &\psi_I(x)=Ae^{Kx}+A'e^{-Kx}\\
          (II) \quad E>0: \quad &k=\sqrt{\frac{2m}{\hbar^2}E}\\
          &\psi_{II}(x)=B\sin(kx)+B'\cos(kx)\\
          (III) \quad E<V_0: \quad &K=\sqrt{\frac{2m}{\hbar^2}(V_0-E)}\\
          &\psi_{III}(x)=Ce^{Kx}+C'e^{-Kx}
        \end{align}
        $\rightarrow$ 6 Integrationskonstanten
      \item 1. Normierbarkeit: Exponentiell Anwachsenden Terme für $x\rightarrow \pm \infty$ dürfen nicht beitragen, da sonst nicht normierbar.
        \begin{align}
          \hookrightarrow &A'=0 \, \quad C=0\\
          &\psi_I(x)=Ae^{Kx}\\
          &\psi_{II}(x)=B\sin(kx)+B'\cos(kx)\\
          &\psi_{III}(x)=C'e^{-Kx}
        \end{align}
      \item 2. Potential symmetrisch unter $x\rightarrow-x$
      \item Formal wird $x\rightarrow-x$ durch Paritätsoperator beschrieben.
        \begin{equation}
        \hat{\Pi}|x\rangle=|-x\rangle \quad \text{wobei} \quad \hat{\Pi}^2=\hat{1} \, , \quad \hat{\Pi}^\dagger=\hat{\Pi} \, , \quad
        \end{equation}
      \item Hamiltonian ist invariant unter Paritätstransformation
        \begin{align}
          &\hat{\Pi}^{-1}\hat{H}\hat{\Pi}=\hat{H} \quad \text{und damit} \quad [\hat{H},\hat{\Pi}]=0
        \end{align}
        $\rightarrow$ Hamiltonop. und $\hat{\Pi}$ haben simultane Eigenbasis
        \begin{align}
        &\langle x|\hat{\Pi}|E_n\rangle=\langle-x|E_n\rangle\\
        &\pm\langle x|E_n\rangle
        \end{align}
        $\rightarrow$ stat. Wellenfunktionen haben entweder gerade oder ungerade Parität
        \begin{align}
        \text{gerade:}\quad \psi_{E_n}(-x)&=\psi_{E_n}(x)\\
        \text{ungerade:}\quad \psi_{E_n}(-x)&=-\psi_{E_n}(x)
        \end{align}
    \end{itemize}
 
\begin{flushright}
Vorlesung 13 - 04.06.2013
\end{flushright}

\subsection{Bindungszustände im Potentialtopf}
(Skizze Potentialtopf symmetrisch um x=0)
\begin{align}
&I \quad &E<V_0 \quad &k=\sqrt{\frac{2m}{\hbar^2}(E-V_0)}\quad &\psi_I(x)=Ae^{kx}+A'e^{-kx}\\
&II\quad &E>0 \quad &K=\sqrt{\frac{2m}{\hbar^2}E}\quad &\psi_{II}(x)=B\sin(kx)+B'\cos(kx)\\
&III\quad &E<V_0 \quad &k=\sqrt{\frac{2m}{\hbar^2}(E-V_0)}\quad &\psi_{III}(x)=Ce^{kx}+C'e^{-kx}
\end{align}
Wegen Stetigkeitsbedingung von $\psi$, $\psi'$ und Skalierun
\begin{equation}
A'=0, \quad C=0
\end{equation}
\textbf{Paritätssymmetrie}
\begin{equation}
\hat{\Pi}|x\rangle=|-x\rangle;\quad [\hat{\Pi},\hat{H}]=0
\end{equation}
Betrachten wir zunächst die \textbf{gerade (positive) Parität}\\
$\Rightarrow$ B=0 \& A=C' (Achsensymmetrie)
\begin{equation}
\left.
\begin{aligned}
&\psi_I(x)=Ae^{kx}\\
&\psi_{II}(x)=B'\cos(Kx)\\
&\psi_{III}(x)=Ae^{-kx}
\end{aligned}
\right\} \xrightarrow{\text{Stetigkeit bei x=0}}\left\{
\begin{aligned}
B'\cos(Ka)&=Ae^{-ka}\\
-B'K\sin{Ka}&=-kAe^{-ka}\\
\rightarrow K\tan(Ka)&=k (1)
\end{aligned}
\right.
\end{equation}
\textbf{Ungerade Parität}\\
$\Rightarrow$ B'=0 \& A=-C' (Achsensymmetrie)
\begin{equation}
\left.
\begin{aligned}
&\psi_I(x)=-Ae^{kx}\\
&\psi_{II}(x)=B\sin(Kx)\\
&\psi_{III}(x)=Ae^{-kx}
\end{aligned}
\right\} \xrightarrow{\text{Stetigkeit bei x=0}}\left\{
\begin{aligned}
B\sin(ka)&=Ae^{-ka}\\
-B'K\cos{Ka}&=-kAe^{-ka}\\
\rightarrow K\cot(Ka)&=-k (2)
\end{aligned}
\right.
\end{equation}
Die beiden transzendenten Gleichungen (1), (2) führen auf ganz bestimmte Energien für die die Stetigkeitsbedingungen erfüllt sind. Die Energiequantisierung ergibt sich aus den Ausschlussbedingungen.\\
Die Bestimmung der Energieeigenwerte $E_n$ kann graphisch oder numerisch bestimmt werden.
\begin{align}
&K\tan(Ka)=k\quad &\text{gerade Parität}\\
&K\cot(Ka)=-k\quad &\text{ungerade Parität}\\
&K_n=\sqrt{\frac{2m}{\hbar^2}E_n}\quad&k_n=\sqrt{\frac{2m}{\hbar^2}(V_0-E)}
\end{align}
Spezialfall $V_0\rightarrow\infty$ unendlich hoher Potentialtopf\\
$k\rightarrow\infty$\\
Es folgt 
\begin{equation}
k_na=\frac{n\pi}{2} \quad E_n=\frac{\hbar^2}{2m}k_n^2\quad n=1,2,...
\end{equation}
\textbf{Betrachte nun Streulösungen}: $E>V_0$\\
(Skizze mit Energie größer als Potentialtopf)\\
Lösung in allen Teilbereichen durch ebene Wellen möglich
\begin{align}
&\psi_I(x)=Ae^{ikx}+A'e^{-ikx}\quad &k=\sqrt{\frac{2m}{\hbar^2}(E-V_0)}\\
&\psi_{II}(x)=Be^{ik'x}+B'e^{-ik'x}\quad &k'=\sqrt{\frac{2m}{\hbar^2}E}\\
&\psi_{III}(x)=Ce^{ikx}+C'e^{-ikx}\quad &k=\sqrt{\frac{2m}{\hbar^2}(E-V_0)}
\end{align}
Betrachte Streuproblem, d.h. man setzt eine von links einlaufende Welle vorraus.\\
$A=A_{\text{in}}$, C'=0, da keine Welle von rechts\\
$\rightarrow$ Wahl von $A_{\text{in}}$ ist Konventionssache, da Streulösungen nicht normierbar sind.
\begin{itemize}
\item Die verbleibenden Konstanten lassen sich aus 4 Stetigkeitsbedingungen bei $x=\pm a$ bestimmen.
\item Keine Energiequantisierung, jede beliebige Energie $>V_0$ ergibt Lösung.\\
Fazit:Potentialtopf
\item diskrete gebundene Zustände mit $E_n<V_0$ und kontinuierliche Streuzustände bei $E>V_0$
\item Quantisierung der Energien ist eine Konsequenz der Normierbarkeit. (wegen Randbedingungen)
\item Wellenfunktion der Bindungszustände haben exponentielle Ausläufer, die in den klassisch verbotenen Bereich tunneln.
\item Eindringtiefe: $\frac{1}{k}=\frac{1}{V_0-E}$
\end{itemize}

\subsection{Potentialbarriere}
(Skizze mit umgekehrtem Potentialtopf)
\begin{equation}
V(x)=\left\{
\begin{aligned}
0\quad &;x<0\\
V_0\quad &;0\leq x\leq a\\
0\quad &;x>a
\end{aligned}
\right.
\end{equation}
\begin{itemize}
	\item keine gebundenen Zustände
  \item 2 Typen von Streuzuständen: $E>V_0$ und $E<V_0$
\end{itemize}
In Region I und III
\begin{align}
\psi_{I}(x)&=Ae^{ikx}+A'e^{-ikx} \quad k=\sqrt{\frac{2m}{\hbar^2}E}\\
\psi_{III}(x)&=Ce^{ikx}+C'e^{-ikx}
\end{align}
Interpretation als Streuproblem\\
$A$: Amplitude der einlaufenden Welle\\
$A'$: ...reflektierten...\\
$C$: ...transmittierten...\\
$C'$: 0, da keine Welle von rechter Seite\\
Entsprechend ergeben sich die Beiträge zur Wahrscheinlichkeitsstromdichte
\begin{equation}
j(x)=\frac{\hbar}{2mi}\left[\psi^*(x)\frac{\partial\psi(x)}{\partial x}-\psi(x)\frac{\partial\psi^*(x)}{\partial x}\right]
\end{equation}
zu
\begin{align}
&\psi_{I}(x)=\frac{\hbar k}{m}\left(|A|^2-|A'|^2\right)\\
&\psi_{III}(x)=|C|^2\\
&\frac{\hbar k}{m}|A|^2\quad \text{einlaufende Wahrscheinlichkeitsdichte}\\
&\frac{\hbar k}{m}|A'|^2\quad \text{reflektierte Wahrscheinlichkeitsdichte}\\
&\frac{\hbar k}{m}|C|^2\quad \text{transmittierte Wahrscheinlichkeitsdichte}
\end{align}

\begin{flushright}
Vorlesung 14 - 06.06.2013
\end{flushright}

\begin{itemize}
	\item Aus den Verhältnissen der reflektierten bzw. transmittierten Wskstromdichte ergeben sich Reflexions- und Transmissionskoeffizient: 
    \begin{align}
      &R=\frac{|A'|^2}{|A|^2} \quad &\text{Reflexionskoeffizient}\\
      &T=\frac{|C|^2}{|A|^2} \quad &\text{Transmissionskoeffizient}
    \end{align}
  \item Aufgrund der Kontinuitätsgleichung gilt
    \begin{align}
    &R+T=1\\
    \Rightarrow &|A|^2=|C|^2+|A'|^2
    \end{align}
  \item Zur Berechnung von R und T benötigen wir die Koeffizienten A,A',C
\end{itemize}
\textbf{Fall 1}: $E<V_0$\\
  Wellenfkt. in Region II
  \begin{equation}
  \psi_{II}(x)=Be^{Kx}+B'e^{-Kx}\quad, K=\sqrt{\frac{2m}{\hbar^2}(V_0-E)}
  \end{equation}
  Die 5 Konstanten werden über die 4 Stetigkeitsbedingungen verknüpft:
  \begin{equation}
  \left.
  \begin{aligned}
  A+A'&=B+B'\\
  ik(A-A')&=K(B-B')
  \end{aligned}
  \right|x=0
  \end{equation}
  \begin{equation}
  \left.
  \begin{aligned}
  Ce^{ika}&=Be^{Ka}+B'e^{-Ka}\\
  ikCe^{ika}&=BKe^{Ka}-B'Ke^{-Ka}
  \end{aligned}
  \right|x=a
  \end{equation}
  B und B' lassen sich eliminieren und man erhält 
  \begin{equation}
  \left.
  \begin{aligned}
  \frac{A'}{A}&=\frac{(k^2+K^2)(e^{2Ka}-1)}{e^{2Ka}(k+ix)^2-(k-ix)^2}\Rightarrow\ R=\frac{|A'|^2}{|A|^2}=\left(1+\frac{4E(V_0-E)}{V_0^2\sinh^2(Ka)}\right)^{-1}\\
    \frac{C}{A}&=\frac{4ikKe^{-ika}e^{Ka}}{e^{2Ka}(k+iK)^2-(k-iK)^2}\Rightarrow\ T=\frac{|C|^2}{|A|^2}=\left(1+\frac{V_0^2\sinh^2(Ka)}{4E(V_0-E)}\right)^{-1}
    \end{aligned}
    \right\}\text{sodass $T+R=1$ gilt}
  \end{equation}
\textbf{Schlussfolgerungen}
  \begin{itemize}
    \item endliche Transmissionswahrscheinlichkeit T, obwohl die Barriere aus klassischer Sicht wegen $E<V_0$ undurchdringlich ist. Die QM erlaubt das \textbf{Tunneln.}
    \item Ist die Barriere hinreichend hoch bzw. breit gilt
      \begin{align}
      &Ka=\sqrt{\frac{2m}{\hbar^2}(V_0-E)}\cdot a \gg 1\\
      \text{so gilt}\quad &T\cong\frac{16E(V_0-E)}{V_0^2}e^{-2Ka}
      \end{align}
      $\Rightarrow$ exp. Abfall mit $Ka$.
    \item Dieser Ausdruck ist nützlich zur Konstruktion von T für allgemeinere Potentiale
  \end{itemize}
\textbf{Fall 2:} $E>V_0$
  \begin{itemize}
    \item Auch in Region II ist die Wellenfunktion als Überlagerung ebener Wellen gegeben. 
      \begin{equation}
      \psi_{II}(x)=Be^{ik'x}+B'e^{-ik'x} \quad \text{mit}\quad k'=\sqrt{\frac{2m}{\hbar^2}(E-V_0)}\neq K
      \end{equation}
    \item Verknüpfung der 5 Konstanten (A,A',B,B',C) durch Stetigkeitsbedingung bei x=0 und x=a. Auch hier bleibt die Normierung offen.
    \item Die Verhältnisse der Amplituden liefern:
      \begin{align}
      R&=\frac{|A'|^2}{|A|^2}=\left[1+\frac{4E(E-V_0)}{V_0^2\sin^2(k'a)}\right]^{-1}\\
      T&=\frac{|C|^2}{|A|^2}=\left[1+\frac{V_0^2\sin^2(k'a)}{4E(E-V_0)}\right]^{-1}
      \end{align}
      wobei $R+T=1$ gilt.
    \item Dieses Resultat gilt ebenfalls für die Streuung an einem anziehenden Potentialtopf\footnote{Kapitel \ref{topf}}. Dort gilt lediglich $k'>k$, und Breite a ist anders definiert.
    \item Überraschend ist die endliche Reflexionswahrscheinlichkeit R. Klassisch würde man für $E>V_0$ immer $T=1$ und $R=0$ erwarten. Quantenmechanisch ergibt sich aber selbst für einen anziehenden Potentialtopf eine gewisse Reflexion.
    \item Transmissionskoeffizient in Abhängigkeit der Energie E. (Skizze)
    \item Für $E/V_0<1$ zeigt sich der Tunneleffekt in Form eines endlichen T.
    \item Für $E/V_0>1$ zeigt sich die Reflexion durch eine Reduktion von T.
  \end{itemize}
 

\begin{flushright}
Vorlesung 15 - 11.06.2013
\end{flushright}
\section{Harmonischer Oszillator (HO)}
\subsection{Grundlagen}
\begin{itemize}
	\item HO eines der wenigen exakt lösbaren Probleme der QM.
  \item Ausgangspunkt: Potentielle Energie V(x) einer Masse m als Funktion der Auslenkung x aus der Ruhelage x=0.
    \begin{equation}
    V(x)=\frac{m\omega^2}{2}x^2\quad \text{$\omega$ Eigenfrequenz des HO}
    \end{equation}
    Beispiele aus klassischer Mechanik:
    \begin{align}
    \omega&=\sqrt{\frac{D}{m}}\quad \text{Federpendel}\\
    \omega&=\sqrt{\frac{a}{l}}\quad \text{Fadenpendel}
    \end{align}
    HO als Näherung für andere Potentiale in der Nähe ihres Minimums\\
    $\hookrightarrow$ Taylorentwicklung\\
    Hamiltonoperator für Teilchen der Masse m in 1-D HO-Potential
    \begin{equation}
    \hat{H}=\frac{1}{2m}\hat{p}^2+\frac{m\omega^2}{2}\hat{x}^2
    \end{equation}
    $\hookrightarrow$ besondere Symmetrie bezüglich $\hat{x}$ und $\hat{p}$ ist Grundlage für viele spezielle Eigenschaften des HO.
  \item Lösung der stationären SG- kann aufgrund der Symmetrie von $\hat{H}$ bezüglich $\hat{x}$ und $\hat{p}$ in Orts- und Impulsdarstellung erfolgen.\\
    Außerdem "`algebraische Lösung"' $\rightarrow$ Kapitel 5.3
\end{itemize}
\subsection{Eigenwertproblem in Ortsdarstellung}
\begin{itemize}
	\item Stationäre SGL in Ortsdarstellung
    \begin{equation}
    \left(-\frac{\hbar^2}{2m}\frac{\partial^2}{\partial x^2}+\frac{m\omega^2}{2}x^2\right)\psi_n(x)=E_n\psi_n(x)
    \end{equation}
    \begin{equation}
    \text{mit}\quad\psi_n(x)=\langle x|E_n\rangle \quad \text{Ortsdarstellung der Energieeigenzustände}
    \end{equation}
  \item Verwende reduzierte Einheiten für x und $E_n$ um die DGL zu lösen.
    \begin{equation}
    \left(-\frac{\partial^2}{\partial x^2}+\frac{m^2\omega^2}{\hbar^2}x^2\right)\psi_n(x)=\frac{2m}{\hbar^2}E_n\psi_n(x)
    \end{equation}
    $\hookrightarrow$ definiere charakteristische Längenskala, die sogenannte Oszillatorlänge
    \begin{equation}
    a=\sqrt{\frac{\hbar}{m\omega}}
    \end{equation}
    Definiere dimensionslose reduzierte Koordinate 
    \begin{equation}
    \xi=\frac{x}{a}
    \end{equation}
    \begin{align}
    \Rightarrow\left(-\frac{1}{a^2}\frac{\partial^2}{\partial\xi^2}+\frac{1}{a^2}\xi^2\right)\psi_n(\xi)&=\frac{2m}{\hbar^2}E_n\psi_n(\xi)\quad |\cdot a^2\\
    \left(-\frac{\partial^2}{\partial\xi^2}+\xi^2\right)\psi_n(\xi)&=\frac{2m}{\hbar^2}a^2E_n\psi_n(\xi)\\
    &=2\epsilon_n\psi_n(\xi)\\
    \text{mit}\quad \epsilon_n=\frac{m}{\hbar^2}a^2E_n&=\frac{1}{\hbar \omega}E_n \quad \text{reduzierte Energie}\\
    \hookrightarrow \frac{\partial^2}{\partial\xi^2}\psi_n(\xi)&=(\xi^2-2\epsilon_n)\psi_n(\xi)
    \end{align}
  \item Zur Lösung der DGL betrachte zunächst das asymptotische Verhalten für $\xi\rightarrow\infty$
    \begin{equation}
    \Rightarrow \frac{\partial^2}{\partial\xi^2}\psi_n(\xi)=\xi^2\psi_n(\xi) \quad \text{für}\quad \xi\rightarrow\infty
    \end{equation}
    mit den allgemeinen Lösungen
    \begin{equation}
    \psi_n(\xi)\sim e^{\pm\xi^2/2} \quad \text{für}\quad \xi\rightarrow\infty
    \end{equation}
  \item HO hat nur Bindungszustände $\rightarrow$ $e^{+\xi^2/2}$ kommt wegen Quadratintegrabilität nicht in Frage!\\
    $\hookrightarrow$ Wfkt. müssen folgendes asymptotisches Verhalten zeigen:
    \begin{equation}
    \psi_n(\xi)\sim e^{-\xi^2/2} \quad \text{für}\quad \xi\rightarrow\infty
    \end{equation}
    \item Test: setze asymptotische Lösung in komplette DGL ein
    \begin{equation}
    \frac{\partial^2}{\partial\xi^2}e^{-\xi^2/2}=-\frac{\partial}{\partial\xi}\left(\xi e^{-\xi^2/2}\right)=(\xi^2-1)e^{-\xi^2/2}=(\xi^2-2\epsilon_n)e^{-\xi^2/2}
    \end{equation}
    offenbar erfüllt
    \begin{align}
    &\psi(\xi)=ce^{-\xi^2/2}\quad\text{bzw.}\quad\psi_n(x)=ce^{-\frac{x^2}{2a^2}}\quad\text{die SGL}\\
    &\epsilon_0=\frac{1}{2}\quad\text{bzw.}\quad E_0=\frac{1}{2}\hbar\omega
    \end{align}
    $\hookrightarrow$ Wfkt. und Energieeigenwert des Grundzustandes
    %\begin{equation}
    %\Rightarrow \frac{\partial^2}{\partial\xi^2}\psi_n(\xi)=(\xi^2-2\epsilon_n)\psi_n(\xi)
    %\end{equation}
\end{itemize}
\textbf{Allgemeine Lösung}
\begin{itemize}
	\item Starte vom asymptotischen Verhalten der Wfkt., um die allgemeine Lösung der stationären SGL abzuleiten:
    \begin{equation}
    \psi_n(\xi)=e^{-\xi^2/2}\cdot H_n(\xi)
    \end{equation}
    $H_n(\xi)$ noch zu bestimmende Funktion
  \item Einsetzen in SGL.
    \begin{align}
      \frac{\partial^2}{\partial\xi^2}\left(e^{-\xi^2/2}H_n(\xi)\right)&=(\xi^2-2\epsilon_n)e^{-\xi^2/2}H_n(\xi)\\
      \Leftrightarrow e^{-\xi^2/2}\left(\frac{\partial^2}{\partial\xi^2}H_n(\xi)-2\xi\frac{\partial}{\partial\xi}H_n(\xi)+(\xi^2-1)H_n(\xi)\right)&=(\xi^2-2\epsilon_n)e^{-\xi^2/2}H_n(\xi)\\
      \Leftrightarrow \frac{\partial^2}{\partial\xi^2}H_n(\xi)-2\epsilon\frac{\partial}{\partial\xi}H_n(\xi)+(2\epsilon_n-1)H_n(\xi)&=0
    \end{align}
    \textbf{Hermite-DGL}
  \item Lösung der Hermite-DGL über Reihenansatz\\
    $\hookrightarrow$ Unterscheide Wfkt. gerader und ungerader Parität\footnote{Dies ist möglich, da $\hat{H}$ und $\hat{\Pi}$ kommutieren}
  \item \textbf{Wfkt gerader Parität}\\
    $\hookrightarrow$ Ansatz für $H(\xi)$:
    \begin{equation}
    H(\xi)=\sum^\infty_{k=0}c_k\cdot\epsilon^{2k}
    \end{equation}
    $\hookrightarrow$ Einsetzen in Hermite-DGL:
    \begin{equation}
    \sum_{k=1}^\infty 2k(2k-1)c_k\xi^{2(k-1)}+\sum_{k=0}^\infty(2\epsilon-1-4k)c_k\xi^{2k}=0
    \end{equation}
    Fasse Beiträge zu gegebener Potenz von $\xi$ zusammen
    \begin{equation}
    \Rightarrow \sum_{k=0}^\infty\left[2(k+1)(2k+1)c_{k+1}+(2\epsilon-1-4k)c_k\right]\xi^{2k}
    \end{equation}
  \item für beliebige $\xi$ muss jeder Summand einzeln verschwinden, d.h.
    \begin{align}
    &2(k+1)(2k+1)c_{k+1}+(2\epsilon-1-4k)c_k=0\\
    &\text{bzw.} \quad \frac{c_{k+1}}{c_k}=\frac{4k+1-2\epsilon}{2(k+1)(2k+1)}\quad\rightarrow\text{Rekursionsrelation für Koeffizienten $c_k$}
    \end{align}
  \item bestimme die noch unbekannten Energieeigenwerte $c_n$ über Normierbarkeit der Wellenfunktion\\
    Betrachte Rekursionsrelation für große k. Falls Reihe \textbf{nicht} in endlicher Ordnung abbricht:
    \begin{equation}
    \frac{c_{k+1}}{c_k}\sim\frac{1}{k}\quad \text{für}\quad k\gg 1
    \end{equation}
  \item Dieses Verhalten findet man auch für Funktionen der Form 
    \begin{equation}
    \xi^{2p}e^{\xi^2} \quad \text{mit p endlich}
    \end{equation}
    $\hookrightarrow$ asymptotische Wfkt.:
    \begin{equation}
    \psi(\xi)\sim\xi^{2p}e^{\xi^2/2}
    \end{equation}
    für $\xi\rightarrow\infty$ divergiert dies und ist nicht normierbar!\\
    $\hookrightarrow$ Normierbarkeit erzwingt, dass Potenzreihe in endlicher Ordnung abbricht.\\
    $H_n(\xi)$: Polynome endlicher Ordnung
  \item Annahme: höchste beitragende Potenz sei $\xi^{2N}$, d.h. $c_N\neq0,\ c_{N+1}=0$
    \begin{equation}
    0=4N+1-2\epsilon_n \quad\text{bzw.}\quad \epsilon_N=2N+\frac{1}{2}\quad,\ N=0,1,2,...
    \end{equation}
    Energieeigenwerte für gerade Wfkt.
\end{itemize}



\begin{flushright}
Vorlesung 16 - 13.06.2013
\end{flushright}
\textbf{Ungerade Parität}
\begin{align}
\text{Ansatz:} \quad H(\xi)=\sum_{k=0}^\infty d_k \xi^{2k+1}
\end{align}
\begin{align}
\text{(Analoge Herleitung)} &\rightarrow\ \text{Rekursionsrelation:}\ \frac{d_{k+1}}{d_k}=\frac{4k+3-2\epsilon}{2(k+1)(2k+3)}\\
\text{(selbes Argument)}\ &\rightarrow\ \text{für k $\gg$ 1: verhält sich wie $\frac{1}{k}$}\\
&\rightarrow\ \text{Entwicklung muss in endlicher Ordnung abbrechen}
\end{align}
\begin{equation}
\Rightarrow \epsilon_N=2N+\frac{3}{2},\quad N=0,1,2,...
\end{equation}
\textbf{insgesamt}
\begin{align}
\epsilon_n&=n+\frac{1}{2},\quad n=0,1,2,...\\
\text{bzw.} \quad E_n&=\hbar\omega(n+\frac{1}{2})
\end{align}
\begin{itemize}
  \item gerade Quantenzahl n $\rightarrow$ gerade Z.\\
    ungerade Quantenzahl n $\rightarrow$ ungerade Z.
  \item Energiespektrum ist äquidistand mit Abstand 1 bzw. $\hbar\omega$
  \item Wellenfunktion mit Normierungskonstante N:
    \begin{equation}
    \psi_n(\xi)=N_n e^{-xi^2/2}H_n(\xi)
    \end{equation}
  \item Hermite-Polynome $H_n(/xi)$
    \begin{align}
    H_0(\xi)&=1\\
    H_1(\xi)&=2\xi\\
    H_2(\xi)&=4\xi^2-2\\
    H_3(\xi)&=8\xi^2-12\xi
    \end{align}
  \item Normierung
    \begin{equation}
    N_n=\left(\frac{1}{\sqrt{\pi}2^nn!a}\right)^{1/2}=\left(\frac{m\omega}{\pi\hbar2^{2n}(n!)^2}\right)^{1/4}
    \end{equation}
  \item Orhonormierungsrelation:
    \begin{equation}
    \int_{-\infty}^\infty\text{d}x\ \psi_n^*(x)\psi_{n'}(x)=\delta_{nn'}
    \end{equation}
\end{itemize}


\subsection{Algebraische Lösung}
\begin{itemize}
  \item Elegantere Methode (nutze spezielle Form des harmonischen Oszillators)\\
    $\hookrightarrow$ funktioniert nur für den HO!
  \item Abstrakter Hamiltonoperator:
    \begin{equation}
    \hat{h}=\frac{1}{\hbar\omega}\hat{H}=\frac{a^2}{2\hbar}\hat{p}^2+\frac{1}{2a}\hat{x}^2=\left(\frac{a}{\sqrt{2}\hbar}\hat{p}\right)^2+\left(\frac{1}{\sqrt{2}a}\hat{x}\right)^2
    \end{equation}
  \item Summe quadratischer Terme
    \begin{align}
    \text{Zahlen:}\quad &a^2+b^2=(a-ib)(a+ib)\\
    \text{Operatoren:}\quad &\text{i.A. nicht kommutativ, d.h. Mischterme bleiben stehen!}
    \end{align}
  \item Definiere neue Operatoren
    \begin{align}
    \hat{a}&=\frac{1}{\sqrt{2}a}\hat{x}+i\frac{a}{\sqrt{2}\hbar}\hat{p}\\
    \hat{a}^\dagger&=\frac{1}{\sqrt{2}a}\hat{x}-i\frac{a}{\sqrt{2}\hbar}\hat{p}
    \end{align}
    $\Rightarrow$ $\hat{h}=\hat{a}^\dagger\hat{a}+\hat{c}$
  \item Bestimme Differenz von $\hat{h}$ und $\hat{a}^\dagger\hat{a}$:
    \begin{align}
    \hat{a}^\dagger\hat{a}&=\left(\frac{1}{\sqrt{2}a}\hat{x}-i\frac{a}{\sqrt{2}\hbar}\hat{p}\right)\left(\frac{1}{\sqrt{2}a}\hat{x}+i\frac{a}{\sqrt{2}\hbar}\hat{p}\right)\\
    &=\underbrace{\frac{1}{2a^2}\hat{x}^2+\frac{a^2}{2\hbar^2}\hat{p}}_{=\hat{h}}+\frac{i}{2\hbar}\underbrace{(\hat{x}\hat{p}-\hat{p}\hat{x})}_{=i\hbar\hat{1}}\\
    \Rightarrow &\hat{c}=\frac{1}{2}\hat{1}\\
    \Rightarrow &\hat{h}=\hat{a}^\dagger\hat{a}+\frac{1}{2}\hat{1}=\hat{n}+\frac{1}{2}\hat{1} \quad \text{mit}\quad \hat{n}:=\hat{a}^\dagger\hat{a}
    \end{align}
  \item Bisher nur formale Umformungen bzw. neue Definitionen\\
    Interpretiere nun die neuen Operatoren\\
    Informationsquelle: Kommutatorrelation
    \begin{equation}
    \begin{aligned}
    &[\hat{a},\hat{a}^\dagger]=\hat{1}\\
    &[\hat{n},\hat{a}]=-\hat{a}\\
    &[\hat{n},\hat{a}^\dagger]=\hat{a}^\dagger
    \end{aligned}
    \ \rightarrow\
    \begin{aligned}
    &\\
    &[\hat{h},\hat{a}]=-\hat{a}\\
    &[\hat{h},\hat{a}^\dagger]=\hat{a}^\dagger
    \end{aligned}
    \end{equation}
  \item Wirkungsweise von $\hat{a},\hat{a}^\dagger$: 
    \begin{align}
    \hat{h}|\epsilon\rangle&=\epsilon|\epsilon\rangle\\
    \hat{h}\hat{a}|\epsilon\rangle&=(\hat{a}\hat{h}+[\hat{h},\hat{a}])|\epsilon\rangle\\
    &=(\hat{a}\hat{h}-\hat{a})|\epsilon\rangle\\
    &=\hat{a}(\hat{h}-\hat{1})|\epsilon\rangle\\
    &=\hat{a}(\epsilon-1)|\epsilon\rangle\\
    &=(\epsilon -1)\hat{a}|\epsilon\rangle
    \end{align}
    $\hookrightarrow$ offenbar gilt:
    \begin{equation}
    \hat{a}|\epsilon\rangle=c_{\epsilon-1}|\epsilon-1\rangle \quad \text{mit Normierungskonstante $c_{\epsilon-1}$}
    \end{equation}
    analog:
    \begin{align}
    \hat{h}\hat{a}^\dagger|\epsilon\rangle&=(\epsilon+1)\hat{a}|\epsilon\rangle\\
    \rightarrow \hat{a}^\dagger|\epsilon\rangle&=c_{\epsilon+1}|\epsilon\rangle
    \end{align}
  \item $\hookrightarrow$ Operatoren $\hat{a}$ und $\hat{a}^\dagger$ bilden Eigenzustände$|\epsilon\rangle$ zum Eigenwert $\epsilon$ auf benachbarte Eigenzustände zu Eigenwerten $(\epsilon-1)$ und $(\epsilon+1)$ ab. Sie werden auch "`Leiteroperatoren"' bzw. Auf- und Abstiegsoperatoren genannt.
  \item Das Spektrum des HO ist nach unten beschränkt.\\
    $\hookrightarrow$ Einschränkung für $\hat{a}$:
    \begin{equation}
    \hat{a}|\epsilon_0\rangle=0 \quad \text{bzw.}\quad c_{\epsilon_0}=0 \quad \text{kein physikalischer Zustand}
    \end{equation}
  \item Komplettes Spektrum
    \begin{equation}
    \epsilon_n=n+\frac{1}{2},\quad n=0,1,2,...
    \end{equation}
  \item Interpretiere $\hat{n}$
    \begin{align}
    \hat{n}|\epsilon_n\rangle=\left(\hbar-\frac{1}{2}\right)|\epsilon_n\rangle&=\left(\epsilon_n-\frac{1}{2}\right)|\epsilon_n\rangle\\
    &=n|\epsilon_n\rangle
    \end{align}
    $\hookrightarrow$ $\hat{n}$ heißt "`Anzahloperator"' und zählt die Anzahl der Anregungsschritte
  \item Berechne $c_n,c_{n+1}$:
    \begin{align}
    \hat{a}|\epsilon_n\rangle=c_n|\epsilon_{n-1}\rangle\quad \Rightarrow \quad \langle\epsilon_n|\hat{a}^\dagger=\langle\epsilon_{n-1}|c_n^*\\
    \Rightarrow \langle \epsilon_n | \underbrace{\hat{a}^\dagger\hat{a}}_{=\hat{h}}=\underbrace{\langle\epsilon_{n-1}|\epsilon_{n-1}\rangle}_{=1}|c_1|^2\\
    \langle\epsilon_n|\hat{n}|\epsilon_n\rangle=n\underbrace{\langle\epsilon_n|\epsilon_n\rangle}{=1}\\
    \Rightarrow c_n=\sqrt{n}
    \end{align}
    analog:
    \begin{align}
    \hat{a}^\dagger|c_n\rangle&=\sqrt{n+1}|\epsilon_{n+1}\rangle\\
    \Rightarrow c_{n+1}&=\sqrt{n-1}
    \end{align}
  \item Eigenzustände in Ortsdarstellung:
    \begin{align}
    \hat{a}|\epsilon_0\rangle=0 \quad &\Rightarrow \quad \langle x|\hat{a}|\epsilon_0\rangle=0\\
    &\Rightarrow \quad \langle x|\frac{1}{\sqrt{2}a}\hat{x}+i\frac{a}{\sqrt{2}\hbar}\hat{p}|\epsilon_0\rangle=0\\
    &\Rightarrow \quad 0=\left(\frac{1}{\sqrt{2}a}x+\frac{a}{\sqrt{2}}\frac{\partial}{\partial x}\right)\underbrace{\langle x|\epsilon_0\rangle}_{=\psi_0(x)}\\
    &\Rightarrow \quad \left(a\frac{\partial}{\partial x}+\frac{x}{a}\right)\psi_0(x)=0\\
    &\quad \text{bzw.} \quad \left(\frac{\partial}{\partial \xi}+\xi\right)\psi_0(\xi)=0
    \end{align}
  \item Das ist nicht die Schrödingergleichung, sondern eine einfache DGL 1. Ordnung
    \begin{align}
    &\Rightarrow \quad \psi_0(x)=N_0e^{-x^2/2a^2} \quad \text{mit}\quad N_0=\left(\frac{1}{\sqrt{\pi}a}\right)^{1/2}\\
    &|\epsilon_n\rangle=\frac{(\hat{a}^\dagger)^n}{\sqrt{n!}}|\epsilon_0\rangle
    \end{align}
    \begin{align}
    \Rightarrow \quad \underbrace{\langle x|\epsilon_n\rangle}_{=\psi_n(x)}&=\frac{1}{\sqrt{n!}}\langle x|(\hat{a}^\dagger)^n|\epsilon_0\rangle\\
    &=\frac{1}{\sqrt{n!}}\langle x|(\frac{1}{\sqrt{2}a}\hat{x}+\frac{a}{\sqrt{2}\hbar}\hat{p})^n|\epsilon_0\rangle\\
    &=\frac{1}{\sqrt{n!}}(\frac{1}{\sqrt{2}a}x+\frac{a}{\sqrt{2}}\frac{\partial}{\partial x})^n\underbrace{\langle x|\epsilon_0\rangle}_{=\psi_0(x)}
    \end{align}
    \begin{align}
    \Rightarrow &\psi_n(x)=N_0 (\frac{1}{\sqrt{2}a}x+\frac{a}{\sqrt{2}}\frac{\partial}{\partial x})^ne^{-x^2/2a^2}\\
    \text{bzw.}\quad &\psi_n(\xi)=N_0\left(\frac{1}{\sqrt{2}}\xi-\frac{1}{\sqrt{2}}\frac{\partial}{\partial\xi}\right)^ne^{-\xi^2/2}
    \end{align}
\end{itemize}



\begin{flushright}
Vorlesung 17 - 18.06.2013
\end{flushright}
\section{Drehimpuls \& Spin}
\subsection{Bahndrehimpuls}
\begin{itemize}
	\item Wichtig bei Behandlung von 3D Problemen, insbesondere wenn Rotationssymmetrie vorliegt.
  \item QM-Formulierung des Bahndrehimpulses: Korrespondenzregeln: 
    \begin{align}
    \vec{L}&=\vec{x}\times\vec{p}\\
    &\Updownarrow \nonumber\\ 
    \hat{\vec{L}}&=\hat{\vec{x}}\times\hat{\vec{p}}
    \end{align}
    oder komponentenweise
    \begin{align}
    \hat{\vec{L}}=\left(\begin{array}{c} \hat{L}_1 \\ \hat{L}_2 \\ \hat{L}_3\end{array}\right),\quad
     \hat{\vec{x}}&=\left(\begin{array}{c} \hat{x}_1 \\ \hat{x}_2 \\ \hat{x}_3\end{array}\right), \quad
     \hat{\vec{p}}=\left(\begin{array}{c} \hat{p}_1 \\ \hat{p}_2 \\ \hat{p}_3\end{array}\right)\\
    \hat{L}_1&=\hat{x}_2\hat{p}_3-\hat{x}_3\hat{p}_2\\
    \hat{L}_2&=\hat{x}_3\hat{p}_1-\hat{x}_1\hat{p}_3\\
    \hat{L}_3&=\hat{x}_1\hat{p}_2-\hat{x}_2\hat{p}_1
    \end{align}
  \item Wichtige Eigenschaft: Kommutatorrelationen zwischen kartesischen Komponenten des Bahndrehimpulsoperators
  \item Erinnerung: Analoge Kommutatoren für Komponenten von  $\hat{\vec{x}}$ oder $\hat{\vec{p}}$
    \begin{equation}
    [\hat{p}_1,\hat{p}_2]=0,\quad  [\hat{p}_1,\hat{p}_3]=0,\ ...
    \end{equation}
  \item Berechne exemplarisch $[\hat{L}_1,\hat{L}_2]$
    \begin{align}
      [\hat{L}_1,\hat{L}_2]&=[\hat{x}_2\hat{p}_3-\hat{x}_3\hat{p}_2,\hat{x}_3\hat{p}_1-\hat{x}_1\hat{p}_3]\\
      &=\underbrace{[\hat{x}_2\hat{p}_3,\hat{x}_3\hat{p}_1]}_{\text{Für diesen Kommutator}}-[\hat{x}_2\hat{p}_3,\hat{x}_1\hat{p}_3]-[\hat{x}_3\hat{p}_2,\hat{x}_3\hat{p}_1]+[\hat{x}_3\hat{p}_2,\hat{x}_1\hat{p}_3]\\
      [\hat{x}_2\hat{p}_3,\hat{x}_3\hat{p}_1]&=\hat{x}_2[\hat{p}_3,\hat{x}_3\hat{p}_1]+[\hat{x}_2,\hat{x}_3\hat{p}_1]\hat{p}_3\\
      &=\hat{x}_2[\hat{p}_3,\hat{x}_3]\hat{p}_1\\
      &=-i\hbar\hat{x}_2\hat{p}_1 \quad \text{mit}\quad [\hat{x}_i,\hat{p}_j]=i\hbar\delta_{ij}
    \end{align}
  \item Analog für alle Beiträge, liefert 
    \begin{align}
    [\hat{L}_1,\hat{L}_2]&=i\hbar(\hat{x}_1\hat{p}_2-\hat{x}_2\hat{p}_1)\\
    &=i\hbar\hat{L}_3
    \end{align}
  \item Drehimpuls-Kommuatorrelationen, Drehimpuls-Algebra
    \begin{align}
    [\hat{L}_1,\hat{L}_2]&=i\hbar\hat{L}_3\\
    [\hat{L}_2,\hat{L}_3]&=i\hbar\hat{L}_1\\
    [\hat{L}_3,\hat{L}_1]&=i\hbar\hat{L}_2\\
    [\hat{L}_i,\hat{L}_j]&=\sum_k\epsilon_{ijk}\hat{L}_k
    \end{align}
    $\hookrightarrow$ Kommutatorrelationen sind definierende Eigenschaft jedes Drehimpulses
  \item Unterscheide:\\
    1. allgemeiner Drehimpuls $\hat{\vec{J}}$: nur Kommutatorrel. sind gegeben
    \begin{equation}
    [\hat{J}_1,\hat{J}_2]=i\hbar\hat{J}_3 \quad \text{zyklisch}
    \end{equation}
    2. Bahndrehimpuls $\hat{\vec{L}}$
    \begin{align}
    [\hat{L}_1,\hat{L}_2]&=i\hbar\hat{L}_3 \quad \text{zyklisch}\\
    \hat{\vec{L}}&=\hat{\vec{x}}\times\hat{\vec{p}}
    \end{align}
\end{itemize}

\subsection{Eigenwertproblem des allgem. Drehimpulses}
\begin{itemize}
	\item Aufgrund der nicht-Kommutation der Komponenten von $\hat{\vec{J}}$ existiert keine simultane Eigenbasis zu allen Komponenten. (anders als bei $\hat{\vec{x}},\hat{\vec{p}}$)
  \item Wähle eine Komponente, traditionell $\hat{J}_3$, und suche nach weiteren Operatoren, die mit $\hat{J}_3$ kommutieren.
    \begin{equation}
    \text{Geraten:}\quad \hat{\vec{J}}^2=\hat{J}_1^2+\hat{J}_2^2+\hat{J}_3^2
    \end{equation}
  \item Berechne Kommutator $[\hat{\vec{J}}^2,\hat{J}_3]$:
    \begin{align}
    [\hat{\vec{J}}^2,\hat{J}_3]&=0\\
    \text{allgemein:}\quad  [\hat{\vec{J}}^2,\hat{J}_i]&=0
    \end{align}
  \item Konstruiere simultane Eigenbasis von $\hat{\vec{J}}^2$ und $\hat{J}_3$\\
    $\hookrightarrow$ Drehimpuls-Eigenbasis
    \begin{align}
    \hat{\vec{J}}^2|a,b\rangle&=\hbar^2a|ab\rangle\\
    \hat{J}_3|ab\rangle&=\hbar b|ab\rangle
    \end{align}
    Aufgabe: Was sind die möglichen Eigenwerte a,b?\\
    Simultane Eigenbasis soll orthonormiert sein:
    \begin{equation}
    \langle ab|a'b'\rangle=\delta_{aa'}\delta_{bb'}
    \end{equation}
  \item Lösungsstrategie: Leiteroperatoren analog zu algebraischer Lösung des HO
  \item Auf- und Absteigeoperatoren für Drehimpuls
    \begin{align}
    \hat{J}_\pm=\hat{J}_1\pm i\hat{J}_2
    \end{align}
    Herm. Adjunktion
    \begin{align}
    (\hat{J}_+)^\dagger=\hat{J}_-,\quad (\hat{J}_-)^\dagger=\hat{J}_+
    \end{align}
    Inverse Relationen
    \begin{align}
    \hat{J}_1&=\frac{1}{2}(\hat{J}_++\hat{J}_-)\\
    \hat{J}_2&=\frac{1}{2}(\hat{J}_+-\hat{J}_-)
    \end{align}
  \item Entscheidende Eigenschaft: Kommutatoren von $\hat{J}_\pm$ mit $\hat{\vec{J}}^2,\hat{J}_3$
    \begin{align}
    [\hat{\vec{J}}^2,\hat{J}_\pm]&=0\\
    [\hat{J}_3,\hat{J}_\pm]&=\pm \hbar \hat{J}_\pm
    \end{align}
    Analogie zum HO: $\hat{J}_\pm$ hat keinen Einfluss auf $\hat{\vec{J}}^2$ Quantenzahl
  \item $\hat{J}_\pm$ wird die $\hat{J}_3$ Quantenzahl erhöhen oder erniedrigen.
\end{itemize}


\begin{flushright}
Vorlesung 18 - 20.06.2013
\end{flushright}


\begin{itemize}
	\item Betrachte die Wirkung von $\hat{J}_+$ auf $ab\rangle$
    \begin{align}
    \hat{J}_3\hat{J}_+|ab\rangle&=(\hat{J}_+\hat{J}_3+[\hat{J}_3,\hat{J}_+])|ab\rangle\\
    &=(\hat{J}_+\hat{J}_3+\hbar\hat{J}_+)|ab\rangle\\
    &=(\hat{J}_+\hbar b+\hbar\hat{J}_+)|ab\rangle\\
    &=\hbar(b+1)\hat{J}_+|ab\rangle
    \end{align}
    $\Rightarrow$ Zustand $\hat{J}_+|ab\rangle$ ist EZ zu $\hat{J}_3$ mit EV $\hbar(b+1)$.
    \begin{align}
    \hat{J}_+|ab\rangle=C_{ab}^{(+)}|a(b+1)\rangle
    \end{align}
  \item Analog für $\hat{J}_-$
    \begin{align}
    \hat{J}_-|ab\rangle=C_{ab}^{(-)}|a(b-1)\rangle
    \end{align}
  \item Zweiter Input: Beschränkung des b-Spektrums. Betrachte
    \begin{align}
    \hat{\vec{J}}^2-\hat{J}^2_3=\hat{J}^2_1+\hat{J}^2_2
    \end{align}
    ist positiver Operator.
    \begin{align}
    &\langle ab|\hat{\vec{J}}^2-\hat{J}^2_3|ab\rangle=\hbar^2a-\hbar^2b^2\geq0\\
    &\Rightarrow \quad a\geq b
    \end{align}
    $\hookrightarrow$ b-Spektrum ist für gegebenes a nach oben \& unten beschränkt!
  \item Größter b-Eigenwert sei $b_{\text{max}}$, dann
    \begin{align}
    \hat{J}_+|ab_{\text{max}}\rangle=0
    \end{align}
    Mit Identität
    \begin{align}
    &\hat{J}_+\hat{J}_-=\hat{\vec{J}}^2-\hat{J}^2_3-\hbar\hat{J}_3\\
    0&=\hat{J}_-\hat{J}_+|ab_{\text{max}}\rangle\\
    &=(\hat{\vec{J}}^2-\hat{J}_3^2-\hbar\hat{J}_3)|ab_{\text{max}}\rangle\\
    &=\hbar^2(a-b_\text{max}^2-b_\text{max})|ab_\text{max}\rangle\\
    &\Rightarrow \quad a=b_\text{max}^2+b_\text{max}
    \end{align}
  \item Analog für $b_\text{min}$ mit $\hat{J}_-$
    \begin{align}
    a=b_\text{min}(b_\text{min}-1)
    \end{align}
  \item Damit gilt auch 
    \begin{align}
    b_\text{max}=-b_\text{min}
    \end{align}
  \item Ausgehend von $|ab\rangle$ soll $|ab_\text{max}\rangle$ durch n-fache Anwendung von $\hat{J}_+$ erzeugt werden:
    \begin{align}
    b_\text{max}=b_\text{min}+n
    \end{align}
    $\Rightarrow$ Mit $b_\text{max}=-b_\text{min}$:
    \begin{align}
    b_\text{max}=\frac{n}{2} \quad n=0,1,2,3,...
    \end{align}
 \item Neue Nomenklatur für QZ
   \begin{align}
   &j=b_\text{max}\\
   &m=b
   \end{align}
   Der Eigenwert zu $\hat{\vec{J}}^2$ lautet
   \begin{align}
    a=j(j+1)
   \end{align}
  \item simultane Eigenbasis zu $\hat{\vec{J}}^2$, $\hat{J}_3$
 
  \begin{align}
   &\hat{\vec{J}}^2|jm\rangle=\hbar j(j+1)|jm\rangle\\
   &\hat{J}_3|jm\rangle=\hbar m |jm\rangle
  \end{align}
  mit QZ
  \begin{align}
   j&=0,1/2,1,3/2...\\
   m&=-j,-j+1,...,j
  \end{align}
\item Spreichweise ``Der Drehimpuls'': j, ``mag. Quantenzahl"' oder "`Drehimpulsprojektion'': m
\item Für gegebenes j sind 2j+1 m-QZ möglich
\item Orhonormierung der EZ
\begin{align}
 \langle jm|j'm'\rangle=\delta_{jj'}\delta_{mm'}
\end{align}
\item Bestimmung der Konstanten $C_{jm}^{(\pm)}$
\begin{align}
 \hat{J}_\pm|jm\rangle=C_{jm}^{(\pm)}|j(m\pm1)\rangle
\end{align}
\item Norm:
\begin{align}
\langle jm|(\hat{J}_+)^\dagger\hat{J}_+|jm\rangle&=|C_{jm}^{(\pm)}|^2\\
&=\langle jm|\hat{J}_-\hat{J}_+|jm\rangle\\
&=\langle jm|\hat{\vec{J}}^2-\hat{J}_3^2-\hbar\hat{J}_3|jm\rangle\\
&=\hbar[j(j+1)-m(m+1)]
\end{align}
  \item damit explizit:
    \begin{align}
    \hat{J}_+|jm\rangle=\hbar\sqrt{j(j+1)-m(m+1)}|j,m+1\rangle\\
    \hat{J}_-|jm\rangle=\hbar\sqrt{j(j+1)-m(m+1)}|j,m-1\rangle
    \end{align}
  \item nützliche Anwendung: Wirkung von $\hat{J}_1$ oder $\hat{J}_2$ auf $|jm\rangle$
    \begin{align}
    \hat{J}_1|jm\rangle&=\frac{1}{2}(\hat{J}_+n\hat{J}_-|jm\rangle\\
    &=\frac{\hbar}{2}\sqrt{j(j+1)-m(m+1)}|j,m+1\rangle+\frac{\hbar}{2}\sqrt{j(j+1)-m(m+1)}|j,m-1\rangle
    \end{align}
\end{itemize}



\subsection{Eigenwertproblem des Bahndrehimpulsoperators}
\begin{itemize}
	\item Bahndrehimpuls $\hat{\vec{L}}$ ist Spezialfall des allgemeinen Drehimpulses.
  \item analoge Struktur des EW-Problems
    \begin{align}
    &\hat{\vec{L}}^2|lm\rangle=\hbar^2l(l+1)|lm\rangle\\
    &\hat{L}_3|lm\rangle=\hbar m|lm\rangle\\
    &\text{mit}\quad l=0,\frac{1}{2},1,\frac{3}{2},...\quad m=-l,-l+1,...+l\nonumber
    \end{align}
  \item Zusätzlich
    \begin{align}
    &\hat{\vec{L}}=\hat{\vec{x}}\times\hat{\vec{p}}\quad\Rightarrow\quad\text{Bahndrehimpuls kann in OD betrachtet werden}\\
    &\langle\vec{x}|\hat{\vec{L}}|\alpha\rangle=-i\hbar(\vec{x}\times\vec{\nabla})\langle\vec{x}|\alpha\rangle
    \end{align}
  \item Ortsvektor lässt sich bequem in Kugelkoordinaten darstellen
    \begin{align}
    &|\vec{x}\rangle=|r\vartheta\varphi\rangle\\
    \text{mit}\quad&x=r\sin\vartheta\cos\varphi\\
    &y=r\sin\vartheta\sin\varphi\\
    &z=r\cos\vartheta
    \end{align}
  \item Für kartesische Komponenten von $\vec{L}$ gilt: Kugelkoordinaten-OD
    \begin{align}
    &\langle r\vartheta\varphi|\hat{L}_1|\alpha\rangle=-i\hbar\left(-\sin\varphi\frac{\partial}{\partial\vartheta}-\cot\vartheta\cos\varphi\frac{\partial}{\partial\varphi}\right)\langle r\vartheta\varphi|\alpha\rangle\\
    &\langle r\vartheta\varphi|\hat{L}_2|\alpha\rangle=-i\hbar\left(\cos\varphi\frac{\partial}{\partial\vartheta}-\cot\vartheta\sin\varphi\frac{\partial}{\partial\varphi}\right)\langle r\vartheta\varphi|\alpha\rangle\\
    &\langle r\vartheta\varphi|\hat{L}_3|\alpha\rangle=-i\hbar\frac{\partial}{\partial\varphi}\\
    &\langle r\vartheta\phi|\hat{\vec{L}}^2|\alpha\rangle=-\hbar^2\underbrace{\left[\frac{1}{\sin\vartheta}\frac{\partial}{\partial\vartheta}\left(\sin\vartheta\frac{\partial}{\partial\vartheta}\right)+\frac{1}{\sin^2\vartheta}\frac{\partial^2}{\partial\varphi^2}\right]}_{\text{Winkelanteil des Laplace-Operators in Kugelkoord.}}\langle r\vartheta\varphi|\alpha\rangle
    \end{align}
  \item Ziel: Lösung des EW-Problems für $\hat{\vec{L}}^2,\hat{L}_3$ in OD\\
  $\hookrightarrow$ OD von $|lm\rangle$
  \begin{equation}
  Y_{lm}(\vartheta,\varphi)=\langle\vartheta\varphi|lm\rangle
  \end{equation}
\end{itemize}


\begin{flushright}
Vorlesung 19 - 25.06.2013
\end{flushright}
\textbf{Eigenwertproblem von $\hat{L}_3$ in Ortsdarstellung}
\begin{itemize}
	\item EW-Problem
    \begin{align}
    \langle\vartheta\varphi|\hat{L}_3|lm\rangle=&\hbar m \langle\vartheta\varphi|lm\rangle\\
    =&i\hbar\frac{\partial}{\partial\varphi}Y_{lm}(\vartheta,\varphi)=\hbar m Y_{lm}(\vartheta,\varphi)
    \end{align}
  \item Separationsansatz
    \begin{align}
    &Y_{lm}(\vartheta,\varphi)=\Theta_{lm}(\vartheta)\Phi(\varphi)\\
    &-i\frac{\partial}{\partial\varphi}\Phi_{lm}(\varphi)=m\Phi_m(\varphi)
    \end{align}
  \item Lösung ist Exponentialfunktion
    \begin{align}
    \Phi_m(\varphi)=\frac{1}{\sqrt{2\pi}}e^{im\varphi}
    \end{align}
    mit Orthonormierung
    \begin{align}
    \int_0^{2\pi}\text{d}\varphi\Phi^*_m(\varphi)\Phi_{m'}(\varphi)=\delta_{mm'}
    \end{align}
  \item Man muss Eindeutigkeit der Wellenfunktion fordern: Halbzahlige m müssen für Bahndrehimpuls ausgeschlossen werden, da
    \begin{align}
    \Phi_m(\varphi)\neq\Phi_m(\varphi+2\pi)\quad\text{für}\quad m=\frac{1}{2},\frac{3}{2},\frac{5}{2},...
    \end{align}
  \item Speziell für Bahndrehimpuls
    \begin{align}
    &l=0,1,2,...\\
    &m=-l,-l+1,...,+l
    \end{align}
\end{itemize}

\noindent\textbf{Eigenwertproblem für $\hat{\vec{L}}^2$ in Ortsdarstellung}
\begin{itemize}
	\item Stand:
    \begin{align}
    Y_{lm}(\vartheta,\varphi)=\frac{1}{\sqrt{2\pi}}e^{im\varphi}\Theta_{lm}(\vartheta)
    \end{align}
\end{itemize}
\textbf{Brute Force Variante}
\begin{itemize}
	\item Lsg. des $\hat{\vec{L}}^2$ EW-Problems durch 
    \begin{align}
    \langle\vartheta\varphi|\hat{\vec{L}}^2|lm\rangle=\hbar^2l(l+1)\langle\vartheta\varphi|lm\rangle
    \end{align}
  \item Einsetzen der Ortsdarstellung von $\hat{\vec{L}}^2$ und des Ansatzes für $Y_{lm}(\vartheta,\varphi)$
    \begin{align}
    &\frac{1}{\sin\vartheta}\frac{\partial}{\partial\vartheta}\left(\sin\vartheta\frac{\partial}{\partial\vartheta}\Theta_{lm}(\vartheta)\right)+\left[l(l+1)-\frac{m^2}{\sin^2\vartheta}\right]\Theta_{lm}(\vartheta)=0\\
    &\hookrightarrow \quad \text{Legendre-Dgl.} \nonumber
    \end{align}
  \item Lsg sind assoziierte Legendre-Polynome $P^m_L(x)$
    \begin{align}
    \Theta_{lm}(\vartheta)=C_{lm}P^m_L(\cos\vartheta)
    \end{align}
  \item Ausgedrückt durch gewöhnliche Legendrepolynome
    \begin{align}
    P_L^m(x)=(1-x^2)^{|m|/2}\frac{\partial^{|m|}}{\partial x^{|m|}}P_L(x)
    \end{align}
  \item Für gewöhnliche L-Polynome gilt Rodiguez-Formel
    \begin{align}
    P_L(x)=\frac{1}{2^ll!}\frac{\partial^l}{\partial x^l}(x^2-1)^l
    \end{align}
  \item Ortsdarstellung der EZ $|lm\rangle$: Kugelflächenfunktion
    \begin{align}
    Y_{lm}(\vartheta,\varphi)&=\langle\vartheta\varphi|lm\rangle\\
    &=(-1)^m\sqrt{\frac{2l+1}{4\pi}\frac{(l-|m|)!}{(l+|m|)!}}\cdot P^m_L(\cos \vartheta)e^{im\varphi}
   \end{align}
\end{itemize}
\textbf{Elegantere Variante: Leiteroperatoren}\\
1. Abbruchbedingung
  \begin{align}
  \hat{L}_+|ll\rangle=0
  \end{align}
  $\hookrightarrow$ Ortsdarstellung von $\hat{L}_+$ liefert Dgl. für $Y_{ll}(\vartheta,\varphi)$
  \begin{align}
  Y_{ll}(\vartheta,\varphi)=\frac{C}{\sqrt{2\pi}}e^{il\varphi}\sin^l\vartheta
  \end{align}
  2. Absteigeoperatoren in Ortsdarstellung
  \begin{align}
  (\hat{L}_-)^{lm}|ll\rangle=\hbar^{l-m}\sqrt{\frac{(2l)!(l+m)!}{(l-m)!}}|lm\rangle
  \end{align}
  $\hookrightarrow$ Ortsdarstellung von $\hat{L}_-$ liefert explizite Form für $Y_{lm}(\vartheta,\varphi)$  $(m\geq0)$
  \begin{align}
  Y_{lm}(\vartheta,\varphi)=\frac{(-1)^l}{2^ll!}\sqrt{\frac{2l+1}{4\pi}}\sqrt{\frac{(l+m)!}{(l-m)!}}e^{im\varphi}\frac{1}{\sin^m\vartheta}\frac{\partial^{l-m}}{\partial(\cos\vartheta)^{l-m}}(\sin\vartheta)^{2L}
  \end{align}
  
  
  \subsection{Spin}
  \begin{itemize}
    \item Spin ist intrinsische Teilcheneigenschaft, die kein klassisches Analogon besitzt.
    \item Stern-Gerlach-Experiment: Aufspaltung von Silber-Atomstrahl in zwei Komponenten.
    \item Klassisch ist die diskrete zweiwertige Eigenschaft nicht zu erklären
    \item Hilbertraum: Betrachte ein Elektron
      \begin{align}
      \mathbb{H}_e=\mathbb{H}_\text{Ort}\otimes \mathbb{H}_\text{Spin}
      \end{align}
      $\hookrightarrow$ Tensorprodukt: Kombiniere Freiheitsgrade aus verschiedenen Hilberträumen
    \item Dimensionen
      \begin{align}
      \text{dim}\ \mathbb{H}_e=(\text{dim}\ \mathbb{H}_\text{Ort})\cdot(\text{dim}\ \mathbb{H}_\text{Spin})
      \end{align}
    \item Zustand in $\mathbb{H}_e$
      \begin{align}
      \underbrace{|\alpha\rangle}_{\in \ \mathbb{H}_e}=\underbrace{|\psi\rangle}_{\in\ \mathbb{H}_\text{Ort}}\otimes \underbrace{|\chi\rangle}_{\in\ \mathbb{H}_\text{Spin}}
      \end{align}
    \item Hypothese: Spin wird durch allgem. Drehimpulsoperator beschrieben werden, dessen Eigenschaften nur über Drehimpulsalgebra festgelegt sind
      \begin{align}
      \hat{\vec{S}}=\left(\begin{array}{c} \hat{S}_1 \\ \hat{S}_2\\\hat{S}_3 \end{array}\right)\quad\text{mit}\quad [\hat{S}_1,\hat{S}_2]=i\hbar \hat{S}_3 \quad\text{und zyklisch}
      \end{align}
    \item Eigenwertproblem für $\hat{\vec{S}}^2,\hat{S}_3$
      \begin{align}
      &\hat{\vec{S}}^2|sm_s\rangle=\hbar^2s(s+1)|sm_s\rangle\\
      &\hat{S}_3|sm_s\rangle=\hbar m_s |sm_s\rangle\\
      &\quad s=0,\frac{1}{2},1,\frac{3}{2},... \nonumber\\
      &\quad m_s=-s,-s+1,...,+s\nonumber
      \end{align}
    \item Unterscheidung:\\
      $s$: unveränderliche Teilcheneigenschaft, d.h. mit Festlegung des zu betrachtenden Teilchens ist s fest vorgegeben, analog zu Ladung oder Masse\\
      $m_s$: dynamische Variable, ein Teilchen kann in allen möglichen $m_s$-Zuständen vorkommen und kann dynamisch geändert werden 
    \item Bsp.\\
      $s$=0: Pionen $\pi^+,\pi^0,\pi^{-1}$\\
      $s=\frac{1}{2}$: Elektron, Myon, Neutrinos, Proton, Neutron, Quarks\\
      $s=1$: Photon\footnote{masselos}, W/Z-Boson, Gluonen, Rho-Meson\\
      $s=\frac{3}{2}$: Delta-Baryone
  \end{itemize}
  
  
\begin{flushright}
Vorlesung 20 - 27.06.2013
\end{flushright}

\noindent\textbf{Spin $s=\frac{1}{2}$}

\begin{itemize}
	\item Besonders wichtig, weil typische Materieteilchen ($e^-$,p,n) $s=\frac{1}{2}$-Teilchen sind
  \item Besonders einfach, weil nur $m_s=\pm\frac{1}{2}$ möglich:
    \begin{align}
    |s=\frac{1}{2},m_s=+\frac{1}{2}\rangle&=|\uparrow\rangle\\
    |s=\frac{1}{2},m_s=-\frac{1}{2}\rangle&=|\downarrow\rangle
    \end{align}
    $\hookrightarrow$ Spin $s=\frac{1}{2}$ Raum ist zweidimensional
  \item Beliebeiger Zustand in $\mathbb{H}_{s=1/2}$
    \begin{align}
    |\chi\rangle=C_\uparrow|\uparrow\rangle+C_\downarrow|\downarrow\rangle
    \end{align}
  \item Alternativ: Spinor-Schreibweise
    \begin{align}
    \left(\begin{array}{c} \langle\uparrow|\chi\rangle \\ \langle\downarrow|\chi\rangle \end{array}\right)=
    \left(\begin{array}{c} C_\uparrow \\ C_\downarrow \end{array}\right)
    \end{align}
  \item Basisdarstellung der Komponenten des Spin-Operators $\hat{S}_1,\hat{S}_2,\hat{S}_3$
    $\rightarrow$ Spin-Matrizen
  \item Matrixdarstellung von $\hat{S}_3$
    \begin{align}
     \left(\begin{matrix}
    \langle\uparrow|\hat{S}_3|\uparrow\rangle & \langle\uparrow|\hat{S}_3|\downarrow\rangle \\ 
     \langle\downarrow|\hat{S}_3|\uparrow\rangle &  \langle\downarrow|\hat{S}_3|\downarrow\rangle 
    \end{matrix}\right)=
    \frac{\hbar}{2}\left(\begin{matrix}
    1 & 0\\ 
    0 &  -1
    \end{matrix}\right)=
    \frac{\hbar}{2}\sigma_3
    \end{align}
  \item Matrixdarstellung von $\hat{S}_1,\hat{S}_2$ $\rightarrow$ Leiteroperatoren
    \begin{align}
    \hat{S}_\pm&=\hat{S}_1\pm i\hat{S}_2\\
    \hat{S}_1&=\frac{1}{2}(\hat{S}_++\hat{S}_-)\\
    \hat{S}_2&=\frac{1}{2}(\hat{S}_+-\hat{S}_-)
    \end{align}
    $\hookrightarrow$ Wirkung auf Basiszustand $|\uparrow\rangle,|\downarrow\rangle$
    \begin{align}
    \hat{S}_+|\uparrow\rangle&=0 \\ \hat{S}_-|\downarrow\rangle&=\hbar|\uparrow\rangle\\
    \hat{S}_-|\uparrow\rangle&=\hbar|\downarrow\rangle \\ \hat{S}_-|\downarrow\rangle&=0
    \end{align}
    $\hookrightarrow$ Spin-Matrizen
    \begin{align}
    \sigma_1=\left(\begin{matrix}
    0 & 1\\ 
    1 &  0
    \end{matrix}\right)\quad
    \sigma_2=\left(\begin{matrix}
    0 & -i\\ 
    i &  0
    \end{matrix}\right)
    \end{align}
  \item Aufgrund der kleinen Dimension gelten verschiedene spezielle Relationen für Operatoren im Spin-Raum\\
    Bsp.: Jeder Operator im Spin-Raum lässt sich als lineare Superposition von 
    \begin{align}
    \{\hat{1},\hat{S}_1,\hat{S}_2,\hat{S}_3\}
    \end{align}
    bzw.
    \begin{align}
    \{\hat{1},\sigma_1,\sigma_2,\sigma_3\}
    \end{align}
    schreiben.
\end{itemize}


\subsection{Drehimpulskopplung}
\begin{itemize}
	\item Drehimpulskoppeln ist für beliebige Drehimpuls-artige Operatoren möglich. Hier speziell Kopplung von Bahndrehimpuls $\hat{\vec{L}}$ und Spin $\hat{\vec{S}}$
  \item Ausgangspunkt: Kommutatorrelation
    \begin{align}
    [\hat{L}_1,\hat{L}_2]&=i\hbar\hat{L}_3,\quad\text{und zyklisch}\\
    [\hat{S}_1,\hat{S}_2]&=i\hbar\hat{S}_3
    \end{align}
    gemischte Kommutatoren
    \begin{align}
    [\hat{L}_i,\hat{S}_j]=0
    \end{align}
    weil $\hat{L}_i$ auf $\mathbb{H}_\text{Ort}$ und $\hat{S}_j$ auf $\mathbb{H}_\text{Spin}$ wirken.
\end{itemize}
    \textbf{Ungekoppelte Basis}
\begin{itemize}
  \item Betrachte Winkelanteil von $\mathbb{H}_\text{Ort}$ und Spinraum $\mathbb{H}_\text{Spin}$
  \item simultane Eigenbasis in $\mathbb{h}$ $\rightarrow$ Satz von kommutierenden Operatoren
    \begin{align}
    \{\hat{\vec{L}}^2,\hat{L}_3,\hat{\vec{S}}^2,\hat{S}_3\}
    \end{align}
    $\hookrightarrow$ simultane Eigenbasis
    \begin{align}
    \hat{\vec{L}}^2|lm_lsm_s\rangle&=\hbar^2 l(l+1)|lm_lsm_s\rangle\\
     \hat{L}_3|lm_lsm_s\rangle&=\hbar m_l|lm_lsm_s\rangle\\
    \hat{\vec{S}}^2|lm_lsm_s\rangle&=\hbar^2 s(s+1)|lm_lsm_s\rangle\\
     \hat{S}_3|lm_lsm_s\rangle&=\hbar m_s|lm_lsm_s\rangle
    \end{align}
  \item Eigenbasis $|lm_lsm_s\rangle$ heißt \textbf{ungekoppelte Basis}
\end{itemize}
\textbf{Gekoppelte Basis}
\begin{itemize}
  \item Andere Möglichkeit aus 6 Komponenten $\hat{L}_i,\hat{S_j}$ vier kommutierende Operatoren zu konstruieren.
  \item Gesamtdrehimpulsoperator:
    \begin{align}
    \hat{\vec{J}}=\hat{\vec{L}}+\hat{\vec{S}}
    \end{align}
    mit Drehimpulskommutatorrelation
    \begin{align}
    [\hat{J}_1,\hat{J}_2]=i\hbar\hat{J}_3\quad\text{und zyklisch}
    \end{align}
    $\hookrightarrow$ zwei sinnvolle Operatoren für simultane Eigenbasis: $\hat{\vec{J}}^2,\hat{J}_3$
  \item Betrachte Kommutatoren
    \begin{align}
    [\hat{\vec{J}}^2,\hat{\vec{L}}^2]&=[\hat{\vec{J}}^2,\hat{\vec{S}}^2]=0\\
    [\hat{J}_3,\hat{\vec{L}}^2]&=[\hat{J}_3,\hat{\vec{S}}^2]=0    
    \end{align}
    aber
    \begin{align}
    [\hat{\vec{J}}^2,\hat{L}_3]\neq0 \quad;\quad[\hat{\vec{J}}^2,\hat{S}_3]\neq0
    \end{align}
  \item simultane E-Basis zu $\hat{\vec{J}}^2,\hat{J}_3,\hat{\vec{L}}^2,\hat{\vec{S}}^2$:
    \begin{align}
    \hat{\vec{J}}^2|(ls)jm\rangle&=\hbar^2j(j+1)|lsjm\rangle\\
    \hat{J_3}|(ls)jm\rangle&=\hbar m|lsjm\rangle\\
    \hat{\vec{L}}^2|(ls)jm\rangle&=\hbar^2l(l+1)|lsjm\rangle\\
    \hat{\vec{S}}^2|(ls)jm\rangle&=\hbar^2s(s+1)|lsjm\rangle\\
    \end{align}
    $\hookrightarrow$ gekoppelte Basis $|(ls)jm\rangle$
\end{itemize}
\textbf{Basistransformation}
\begin{itemize}
	\item Transform. zwischen zwei vollständigen Basen in $\mathbb{H}_{\text{Ort},\Omega}\otimes\mathbb{H}_\text{Spin}$
  \item Vollständigkeitsregel:
    \begin{align}
    \hat{1}&=\sum_{lm_lsm_s}|lm_lsm_s\rangle\langle lm_lsm_s|\quad\text{(ungekoppelt)}\\
    \hat{1}&=\sum_{lsjm}|(ls)jm\rangle\langle (ls)jm|\quad\text{(gekoppelt)}
    \end{align}
  \item Bsp.: Darstellung von Zustand aus gek. Basis in ungekopp. Basis
    \begin{align}
    |(ls)jm\rangle&=\hat{1}|(ls)jm\rangle\\
    &=\sum_{l'm_ls'm_s}|l'm_ls'm_s\rangle\langle l'm_ls'm_s|(ls)jm\rangle\\
    &=\sum_{m_lm_s}|lm_lsm_s\rangle\underbrace{\langle lm_lsm_s|(ls)jm\rangle}_\text{Transformationskoeff.}
    \end{align}
  \item Clebsch-Gordan-Koeffizienten
    \begin{align}
    C\left(\begin{matrix}
    l&s\\
    m_l&m_s
    \end{matrix}\right.\left|\left.
    \begin{matrix}
    j\\
    m
    \end{matrix}\right)\right.
    :=\langle lm_lsm_s|(ls)jm\rangle
    \end{align}
    $\hookrightarrow$ Rekursionsrelation von Leiteroperatoren
  \item  Kopplung bzw. Entkopplung
    \begin{align}
    |(ls)jm\rangle= \sum_{m_l,m_s}&
    \left(\begin{matrix}
    l&s\\
    m_l&m_s
    \end{matrix}\right.\left|\left.
    \begin{matrix}
    j\\
    m
    \end{matrix}\right)\right.|lm_lsm_s\rangle
    \\
        |lm_lsm_s\rangle= \sum_{j,m}&
    \left(\begin{matrix}
    l&s\\
    m_l&m_s
    \end{matrix}\right.\left|\left.
    \begin{matrix}
    j\\
    m
    \end{matrix}\right)\right.|(ls)jm\rangle\\
    \end{align}
\end{itemize}

\begin{flushright}
Vorlesung 21 - 04.07.2013
\end{flushright}

\begin{flushright}
Vorlesung 22 - 09.07.2013
\end{flushright}

\noindent\textbf{Lösung der Relativgleichung}
\begin{itemize}
	\item Relativ Hamiltonian
    \begin{align}
    \hat{H}_\text{rel}=\frac{1}{2\mu}\hat{\vec{q}}^2+V(\hat{r})
    \end{align}
  \item Simultane Eigenbasis 
    \begin{align}
    \hat{H}_\text{rel}|nlm\rangle&=E_n"|nlm\rangle\\
    \hat{\vec{L}}^2"|nlm\rangle&=\hbar^2l(l+1)|nlm\rangle\\
    \hat{L}_3|nlm\rangle&=\hbar m|nlm\rangle
    \end{align}
  \item Ortsdarstellung der $|nlm\rangle$
    \begin{align}
    \psi_{nlm}(r,\vartheta,\varphi)&=\langle r\vartheta\varphi|nlm\rangle\\
    &=R_{nl}(r)Y_{lm}(\vartheta,\varphi)\\
    &=\frac{u_{nl}(r)}{r}Y_{lm}(\vartheta,\varphi)
    \end{align}
  \item Radial-Schrödingergleichung
    \begin{align}
    &\left\{-\frac{\hbar^2}{2\mu}\frac{\partial^2}{\partial r^2}+\frac{\hbar^2}{2\mu}\frac{l(l+1)}{r^2}+V(r)\right\}u_{nl}(r)=E_nu_{nl}(r)\\
    &\text{mit}\quad V(r)=-\frac{e^2}{r}
    \end{align}
    \textcircled{1} Asymptotisches Verhalten\\
    \textcircled{a} r $\rightarrow$ 0 Radialgleichung durch führenden $\frac{1}{r^2}$~Term bestimmt.
    \begin{align}
    -\frac{\partial^2}{\partial r^2}u_{nl}(r)+\frac{l(l+1)}{r^2}u_{nl}(r)=0
    \end{align}
    Allgemeine Lösung
    \begin{align}
    u_{nl}(r)=Ar^{l+1}+Br^{-l}
    \end{align}
    Wegen $u(r=0)=0$ bleibt nur $r^{l+1}$~Term als möglicher Beitrag übrig.\\
    \textcircled{b} r $\rightarrow~\infty$ Radialgl.
    \begin{align}
    \frac{\partial^2}{\partial r^2}u_{nl}(r)+\frac{2\mu}{\hbar^2}E_ru_{nl}(r)=0
    \end{align}
    Allgemeine Lösung
    \begin{align}
    &u_nl(r)=Ae^{-\lambda r}+Be^{+\lambda r}\\
    &\lambda=\sqrt{-\frac{2\mu}{\hbar^2}E_n}
    \end{align}
    Aus Normierbarkeit folgt, dass nur $e^{-\lambda r}$ sinnvolle Lösung ist.
  \item Ansatz für Radialwellenfunktion
    \begin{align}
    u_{nl}(r)=r^{l+1}f(r)e^{-\lambda r}
    \end{align}
    \textcircled{2} Lsg. der vollen Radialgleichung über Reihenansatz für $f(r)$
  \item Dgl. für f(r)
    \begin{align}
    \frac{\partial^2}{\partial r^2}f(r)+z\left(\frac{l+1}{r}-\lambda\right)\frac{\partial}{\partial r}f(r)+2\left(\frac{-\lambda(l+1)+\mu e^2/\hbar^2}{r}\right)f(r)=0
    \end{align}
  \item Reihenansatz für $f(r)$
    \begin{align}
    f(r)=\sum^\infty_{k=0}b_kr^k
    \end{align}
    $\Rightarrow$ Rekursionsrelation durch Einsetzen in f-Dgl
    \begin{align}
    \frac{b_k}{b_{k-1}}=\frac{2\left(\lambda(k+l)-\frac{\mu e^2}{\hbar^2}\right)}{k(k+2l+1)}
    \end{align}
  \item Rekursionsrelation für sehr große k entspricht einer analogen Rekursion für die Reihendarstellung von $e^{+2\lambda r}$\\
    $\hookrightarrow$ Rekursion muss in endlicher Ordnung abbrechen, um Normierbarkeit zu gewährleisten
  \item Erzwinge Abbruch in Ordnung $k=N$
    \begin{align}
    \frac{b_{N+1}}{b_N}=0\quad \Rightarrow\quad \lambda(N+1+l)-\frac{\mu e^2}{\hbar^2}=0
    \end{align}
  \item Über Zusammenhang von $\lambda$ und $E_n$ ergibt sich aus Abbruchbedingung eine Quantisierungsbedingung für $E_n$ 
    \begin{align}
    E_n=E_{Nl}&=-\frac{\mu e^4}{2\hbar^2}\frac{1}{(N+l+1)}\\
    &=-\frac{\mu e^4}{2\hbar^2}\frac{1}{n^2}
    \end{align}
    \begin{align}
    n&=N+l+1=1,2,3,...&\text{Hauptquantenzahl}\\
    N&=0,1,2,...&\text{Radialquantenzahl}\\
    L&=0,1,2,... &\text{Bahndrehimpuls}
    \end{align}
  \item Grundzustand für $n=1$ 
    \begin{align}
    E_1=-\frac{\mu e^4}{2\hbar^2}=-\frac{e^2}{2a_0}=-13.6~\text{eV}
    \end{align}
    Bohrradius:
    \begin{align}
    a_0=\frac{\hbar^2}{\mu e^2}=0.53~\mathring{A}
    \end{align}
  \item Spektrum (Zeichnung der Energieniveaus, l-Entartung)
  \item Spezialität 1: Unendlich viele Bindungszustände, Spektrum wird mit $n\rightarrow\infty$ immer dichter.
  \item Spezialität 2: Weitere dynamische Symmetrie, die zur Entartung der verschiedenen l-Zustände führt.\\
    klass. Mechanik: Runge-Lenz Vektor für Kepler-Problem
  \item Entartung bezüglich m (Rotationssymmetrie)
    \begin{align}
    g_{nl}=(2l+1)
    \end{align}
  \item Entartung bezüglich l (Runge-Lenz)
    \begin{align}
    g_n=\sum_{l=0}^{n-1}(2l+1)=n^2
    \end{align}
\end{itemize}


\noindent\textbf{Wellenfunktionenen}

\begin{itemize}
  \item Wellenfunktionen ergeben direkt aus den Rekursionskoeff. bei gegebenem $E_n$\\
    $\hookrightarrow$ f-Dgl. liefert die sog. assoziierten Laguerre-Polynome
    \begin{align}
    L^N_k(x)=\frac{\partial^N}{\partial x^N}L_k(x)=\frac{\partial^N}{\partial x^N}\left(e^x\frac{\partial^k}{\partial x^k}(x_k e^{-x})\right)
    \end{align}
  \item Radialwellenfunktion $R_{nl}(r)$
    \begin{align}
    R_{nl}(r)&=N_{nl}\left(\frac{2r}{na_0}\right)^le^{-r/na_0}L^{2l+1}_{n+l}\left(\frac{2r}{na_0}\right)\\
    N_{nl}&=-\frac{2}{na_0}^{3/2}\sqrt{\frac{(n-l-1)!}{2n[(n+l)!]^3}}
    \end{align}
\end{itemize}

\end{document}