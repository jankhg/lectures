\documentclass[10pt,article,colorback,accentcolor=tud9d]{tudreport}
\usepackage[utf8]{inputenc} %?%
\usepackage[T1]{fontenc}
\usepackage{lmodern} %cool beans%
\usepackage[english,ngerman]{babel} %deutsche Schriftzeichen%
\usepackage{amsmath} %für Formeln%
\usepackage{float} %Für Position von Abbildungen%
\usepackage[singlespacing]{setspace}
\usepackage{nccmath}
\usepackage[colorlinks,
pdfpagelabels,
pdfstartview = FitH,
bookmarksopen = true,
bookmarksnumbered = true,
linkcolor = black,
plainpages = false,
hypertexnames = false,
citecolor = black] {hyperref} %Für verlinktes Inhaltsverzeichnis% 
\title{Theoretische Physik IV}
\subtitle{Professor: Dr. Gernot Alber\\
Mitschrift von Philipp Dijkstal}
\subsubtitle{email: \textaccent{philipp.dijkstal@web.de}}
\institution{Fachbereich Physik}
\begin{document}
\maketitle
\tableofcontents
\newpage
Vorlesung 1 - 16.04.2013
\section{Prinzipien der Thermodynamik}
Grundbegriffe:
\begin{itemize}
\item thermodynamische Gleichgewichtszustände, leicht beschreibbar
\item thermodynamische Systeme
\end{itemize}
$\rightarrow$ Aussagen über natürliche Zustandsänderungen
\subsection{Grundkonzepte der Thermodynamik}
\subsubsection{Thermodynamische Systeme}
\begin{itemize}
\item wohldefinierte Menge einer physikalischen Substanz
\item (Idealisierung) Skalierbarkeit (beliebige Teilungs- und Vereinigungsprozesse sind möglich)
\end{itemize}
1 Atom, 1 Universum sind keine thermodynamische Systeme
\subsubsection{Thermodynamische Gleichgewichtszustände}
\textbf{Postulat I}: Makroskopische Materie kann in thermodynamischen Gleichgewichtszuständen sein\\
Eigenschaften:
\begin{itemize}
\item keine zeitlichen Änderungen (auf makroskopischen Skalen)
\item unabhängig von Präparation (Historie)
\item wenige Parameter (thermodynamische Variablen, Koordinaten)
\end{itemize}
z.B. Einfache Systeme (ungeladen, keine Oberflächeneffekte)\par \noindent \begingroup \leftskip2em{innere Energie (Gesamtenergie) U\\
Volumen V\\ 
Teilchenzahlen ${N}_i$, $i=1,...,r$} \par \endgroup \noindent
\begin{doublespace} $X=(U,V,{N}_1,...,{N}_r)$ \end{doublespace} \noindent
\textbf{1. Hauptsatz der Thermodynamik}\\ \\
Wird ein thermodynamischer Gleichgewichtszustand Y von X durch "`natürliche Prozessführung"' (adiabatische Erreichbarkeit) erreicht, so ist die am thermodynamischen System verrichtete Areit unabhängig davon, auf welchem Wege diese Arbeit zugeführt wurde, d.h. U ist eine thermodynamische Koordinate.\\
$\rightarrow$ "`Innere Energie U ist immer eine Koordinate"' Inhalt des 1. HS\\ \par
\noindent \textbf{Sonderfälle}
\begin{itemize}
\item Reservoir: U ist die einzige thermodynamische Variable
\item Mechanische Systeme im Gleichgewicht: Masse M in konsv. Schwerefeld (g - Erdbeschleunigung)
\end{itemize}
$X = (U=Mgz), \ \text{mit} \ z=\text{const}$\\ \par
\noindent \textbf{Bemerkungen:}
$U,V,{N}_1,...,{N}_i$ skalieren mit der Größe des Systems\\
$\rightarrow$ extensive thermodynamische Variablen
\subsubsection{Adiabatische Erreichbarkeit}
Max Planck \par \noindent \begingroup \leftskip2em{Y ist von X aus adiabatisch erreichbar, d.h. X<Y, wenn es möglich ist, die Zustandsänderung des betrachteten thermodynamischen Systems  von X nach Y mit Hilfe eines thermodynamischen Hilfssystems (Umgebung, z.B. Maschine) so durchzuführen, dass der einzige Effekt auf dieses Hilfssystem äquivalent einer mechanischen Energieänderung ist, wie z.B. das Heben oder Senken eines Gewichts.} \par \endgroup \noindent
\onehalfspacing{$\rightarrow$ Formulierung ohne Begriffe Wärme und Temperatur}\\ \\
Literatur:
\begin{itemize}
\item Thess: Das Entropieprinzip (Thermodynamik für Unzufriedene)
\item E.Lieb, J. Yngvason: Phys. Rep. 310/ S.1 (1999)
\end{itemize}
\textbf{mathematisch:}\\ \\$X,Y \in \Gamma$ \\Ordnungsrelation: $X<Y$ $ \Leftrightarrow S(X) \leq S(Y)$\\ \\
\textbf{Notation:}
\begin{fleqn}
\begin{equation}
\begin{aligned} \nonumber
&X<Y \wedge Y<X \Leftrightarrow X \sim Y\\
&X<Y \wedge Y\not< X \Leftrightarrow X\ll Y
\end{aligned}
\end{equation}
\end{fleqn}
Zusammengesetze Systeme:
\begin{fleqn}
\begin{equation} \nonumber
\begin{aligned}
&X=({U}_1,{V}_1,{N}_1)\\
&Y=({U}_2,{V}_2,{N}_2)\\
&(X,Y)=({U}_1,{V}_1,{N}_1,{U}_2,{V}_2,{N}_2)
\end{aligned}
\end{equation}
\end{fleqn}
\textbf{Axiome der adiabatischen Erreichbarkeit:}
\begin{itemize}
\item reflexiv: 
\begin{fleqn}
\begin{equation}
X\sim X
\end{equation}
\item transitiv: 
\begin{equation}
X<Y \wedge Y<Z \Rightarrow X<Z
\end{equation}
\item konsistent:
\begin{equation} 
\left.
\begin{aligned}
&X<X'; X,X' \in \Gamma \\
&Y<Y'; Y,Y' \in \Gamma '
\end{aligned}
\right\} 
\quad \Rightarrow (X,Y)<(X',Y')
\end{equation}
\item skalierbar:
\begin{equation}
\begin{aligned}
&X<Y \Rightarrow \lambda X< \lambda Y\\
&\lambda X :=(\lambda {U}_1, \lambda {V}_1, \lambda {N}_1)
\end{aligned}
\end{equation}
\item teilen und wiedervereinigen:
\begin{equation}
((1-\lambda)X,\lambda X) \sim X, \quad \lambda \in [0,1]
\end{equation} 
\item stabil:
\begin{equation}
\begin{aligned}
&X,Y \in \Gamma, \quad {Z}_0,{Z}_1 \in \Gamma', \quad \epsilon \rightarrow 0\\
&(X,\epsilon {Z}_0) < (Y, \epsilon {Z}_1) \Rightarrow X<Y
\end{aligned}
\end{equation}
\item Vergleichbarkeitsprinzip: \\
Für jedes Paar von Zuständen $X,Y \in \Gamma$ gilt entweder $X<Y$ oder $Y<X$ oder beides. Dies gilt auch für zusammengesetzte Systeme und Zustände der Form
\begin{equation}
\begin{aligned}
&\underbrace{((1-\lambda)X,\lambda X)}<\underbrace{((1-\lambda')Y,\lambda'Y)} \quad \text{für} \ 0\leq \lambda, \lambda' \leq 1\\
&\quad \quad := {X}_{\lambda} \quad \quad \quad \quad \quad \quad:={Y}_{\lambda'}
\end{aligned}
\end{equation}
${X}_\lambda < {Y}_{\lambda'}$ oder ${X}_\lambda > {Y}_{\lambda'}$ oder beides\\
Dazu gehören andere Zustandsräume $\Gamma_\lambda$.\\
\begin{equation}
\begin{aligned}
\Gamma_{1-\lambda} x \Gamma_\lambda \quad &X_\lambda :=((1-\lambda)X_1,\lambda X_2) \quad X_1,X_2,Y_1,Y_{-2} \in \Gamma\\
&Y_\lambda :=((1-\lambda)Y_1,\lambda Y_2)
\end{aligned}
\end{equation}

\item konvexe Kombinierbarkeit:\\
\begin{equation}
\begin{aligned}
&Z:=(1-\lambda)X +\lambda Y, \quad \lambda \in [0,1], \quad \text{Linearkombination von} \ \lambda X \ \text{und} \ (1-\lambda)X\\
&((1-\lambda)X,\lambda Y)<Z
\end{aligned}
\end{equation}
Mischung möglich!
\end{fleqn}
\end{itemize}
Vorlesung 2 - 18.04.2013\\
$\Rightarrow$ \textbf{lokales Entropieprinzip} (für 1 thermodynamisches System, $\Gamma$)\\
$\exists \ S_\Gamma(X)$ mit folgenden Eigenschaften: 
\begin{itemize}
\item monoton unter $<$
\begin{equation}
X<Y \quad ; \quad X,Y \in \Gamma \Leftrightarrow S_\Gamma(X) \leq S_\Gamma(Y)
\end{equation}
\item additiv
\begin{equation}
S_{\Gamma \times \Gamma}((X,Y))=S_\Gamma(X)+S_\Gamma(X)+S_\Gamma(Y)
\end{equation}
\item skalierbar 
\begin{equation}
S_{\Gamma_\lambda}(\lambda X)=\lambda S_\Gamma(X)
\end{equation}
\item konkav
\begin{equation}
\lambda S_\Gamma(X)+(1-\lambda)S_\Gamma(Y) \leq S_\Gamma(\lambda X +(1-\lambda)Y)
\end{equation}
\end{itemize}
\textbf{Konstruktion EINRÜCKEN}:\\
wähle $X_0 \in \Gamma \ll X_1 \in \Gamma$;\\
sei $X \in \Gamma$\\
$\text{sup}_\lambda \{\lambda: \ ((1-\lambda)X_0,\lambda X_1) < X\}:=S_\Gamma(X)$\\
$\Leftrightarrow \quad X \sim ((1-\underbrace{S_\Gamma(X)})X_0,\underbrace{S_\Gamma(X)}X_1)\\
\in R \quad \in R$\\
Alternative Definition von $((1-\lambda)X_0,\lambda X_1)<X$:\\
\begin{itemize}
\item $0 \leq \lambda \leq 1$
\item $\lambda < 0: \quad (1-\lambda)X_0<(X,-\lambda X_1)$
\item $(1-\lambda)<0: \quad \lambda X_1<((\lambda-1)X_0,X)$
\end{itemize} 
\textbf{Eindeutigkeit:} $\overline{S}\sim_\Gamma(X)=aS_\Gamma(X)+b$\\
Für jede Substanz die Freiheit, das b zu wählen. a ist nach einmaligem Festlegen eine Konstante.\\
Mischung: 6-dim-Raum zu 3-dim Raum\\
EINRÜCKEN ENDE\\
\textbf{Es gilt:}
\begin{itemize}
\item Betrachtung zusammengesetzter Systeme (ohne Mischung und chemische Reaktionen)
$\rightarrow \exists S(X) = a_\Gamma S_\Gamma(X) + b_\Gamma$ mit $a_\Gamma$ kann auf $a_{\Gamma^0}$ zurückgeführt werden. $b_\Gamma$ beliebig.\\
d.h. durch die Wahl der Entropiefunktion für eine Substanz, z.B. $H_2O$, sind alle multiplikativen Konstanten festgelegt; additive Konstanten sind frei wählbar.\\
Referenzsubstanz: z.B. $H_2O$, d.h. $X_0 \ll X_1, \quad X_0,X_1 \in \Gamma \Rightarrow S_\Gamma(X)$ festgelegt\\
betrachte 2. Substanz $Y_0 \ll Y_1, \quad Y_0,Y_1 \in \Gamma' \Rightarrow S_{\Gamma'}(Y)$\\
\begin{fleqn}
\begin{equation}
\begin{aligned}
\text{bestimmt mit} \quad &(Y_0,tX_1) \sim (Y_1,tX_0)
&\Rightarrow S_{\Gamma'}(Y_0)+tS_\Gamma(X_1)=S_{\Gamma'}(Y_0)+tS_\Gamma(X_0) \Leftrightarrow S_{\Gamma'}(Y_1)-S_{\Gamma'}(Y_0)=t(S_\Gamma
\end{aligned}
\end{equation}
\item Betrachtung von Mischungen und chemischen Prozessen\\
$\Rightarrow$ alle additiven Konstanten sind durch die Wahl eines einzigen a bestimmt.
\end{fleqn}
\end{itemize}
$\Rightarrow$ \textbf{globales Entropieprinzip}\\
$\exists$ globale Entropiefunktion $S(X)$; eindeutig bis auf die Wahl einer multiplikativen und einer additiven Konstante.\\
monton, additiv, skalierbar, konkav in Bezug auf adiabatische Erreichbarkeit und für alle Materialien
\begin{fleqn}
\begin{equation}
\begin{aligned}
&X<Y \Leftrightarrow S(X) \leq S(Y)\\
&X<Y \wedge Y<X \Leftrightarrow X \sim Y\\
&X<Y \wedge Y\not{<}X \Leftrightarrow X \ll Y
\end{aligned}
\end{equation}
\end{fleqn}
\textbf{Postulat II}\\
Die globale Entropiefunktion ist:
\begin{itemize}
\item stetig differenzierbar in Bezug auf alle extensiven thermodynamischen Variablen
\item streng monoton wachsend in Bezug auf U
\end{itemize}
\subsection{Kontrolle thermodynamischer Gleichgewichtszustände}
 Extensive Parameter $(\underbrace{(U,V,N)})$ sind kontrollierbar durch ``Wände''.\\
 $\quad \quad \quad:=X$\\
 Es gibt \textbf{adiabatische Wände} mit der Eigenschaft:\\
 Energieänderungen haben ihre Ursache in Arbeit
 \subsection{Grundproblem der Thermodynamik}
 1. Ausgangspunkt Zusammengesetztes, abgeschlossenes, thermodynamisches System. Wände müssen Eeigenschaften stabilisieren.\\
 Beseitigt man die Wände zwischen  2 Systemen, gilt Energieerhaltung, aber man weiß nicht, wie sich diese Energie verteilt.
 2. neuer thermodynamischer Gleichgewichtszustand, 5-dim statt 6-dim\\
 $\Rightarrow$ \textbf{Frage:} Was bestimmt $X$? Welcher neuerr thermodynamischer Zusstand ist adiabatisch erreichbar?\\
 \textbf{Postulat III:}\\
 Der neue adiabatisch erreichbarer thermmodynamischer Gleichgewichtszustand ist durch mmaximale Entropie charakterisiert,
 d.h. $S(X_0) \quad \rightarrow$  max.;\\
 falls $X_0$ eindeuutig $\rightarrow$  stabil\\
 \textbf{Tthermodynamischess Fundamentalrelation}\\
 $S(X) \quad \rightarrow$ $dS(U,V,N)= \frac{\partial S}{\partial U}\left.\right|_{(V,N)} dU  + \frac{\partial S}{\partial V}\left.\right|_{(U,N)} + \frac{\partial  S}{\partial N}\left.\right|_{(U,V)} dN =0$

\end{document}